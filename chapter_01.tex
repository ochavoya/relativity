For the space-time of physics we need four coordinates, the time $t$ and three 
space coordinates $x$, $y$, $z$. We put 
\[
 t = x^0, x = x^1, y = x^2, z = x^3,
\]
so that the four coordinates may be written $x^\mu$, where the suffix $\mu$ 
takes on the values 0, 1, 2, 3. The suffix is written in the upper position so 
that we may maintain a ``balancing'' of the suffixes in all the general 
equations of the theory. The precise meaning of ``balancing'' will become clear 
a little later.

Let us take a point close to the point that we originally considered and let 
its coordinates be $x^\mu + dx^\mu$. The four quantities $dx^\mu$ which form 
the displacement may be considered as the components of a vector. The laws of 
special relativity allow us to make linear nonhomogeneous transformations of the 
coordinates, resulting in linear homogeneous transformations of the $dx^\mu$. 
These are such that, if we choose units of distance and of time such that the 
velocity of light is unity,
\begin{equation}
\label{ecuacion 1.1}
 \left( dx^0 \right)^2 - \left( dx^1 \right)^2 - \left( dx^2 \right)^2 - \left( dx^3 \right)^2 = 0 
\end{equation}
is invariant.

Any set of quantities $A^\mu$ that transform under a change of coordinates in 
the same way as the $dx^\mu$ form what is called a \emph{contravariant vector}. 
The invariant quantity 
\begin{equation}
 \label{ecuacion 1.2}
 \left(A^0 \right)^2 -  \left(A^1 \right)^2 -  \left(A^2 \right)^2 - \left(A^3 \right)^2 = \left( A , A \right) 
\end{equation}
may be called the squared length of the vector. With a second contravariant 
vector we have the scalar product invariant
\begin{equation}
 \label{ecuacion 1.3}
 A^0 B^0  - A^1 B^1 - A^2 B^2 - A^3 B^3 = \left( A, B \right)
\end{equation}

In order to get a convenient way of writing such invariants we introduce the 
device of lowering suffixes. Define
\begin{equation}
 \label{ecuacion 1.4}
 A_0 = A^0,\, A_1 = -A^1,\, A_2 = -A^2,\, A_3 = -A^3 .
\end{equation}
Then the expression on the left side of (\ref{ecuacion 1.2}) may be written 
$A_\mu A^\mu$, in which it is understood that a summation is to be taken over 
the four values of $\mu$. With the same notation we can write (\ref{ecuacion 
1.3}) as $A_\mu B^\mu$ or else $A^\mu B_\mu$.

The four quantities $A\mu$ introduced by equation (\ref{ecuacion 1.4}) ma also 
be considered as the compoenents of a vector. Their transformation laws under a 
change of coordinates are somewhat different from those of $A^\mu$, because of 
the diferences in sign, and the vector is called a \emph{covariant 
vector}.(\footnote{
In fact, if we set $A_\mu = g_{\mu\nu} A^\nu$ where
\[
  g_{\mu\nu} = 
  g^{\mu\nu} =
  \left(
    \begin{array}{cccc}
        1 &  0 &  0 &  0 \\
        0 & -1 &  0 &  0 \\
        0 &  0 & -1 &  0 \\
        0 &  0 &  0 & -1
    \end{array}
  \right)
\]
we can see that the coefficients of the transformation ${A'_{\mu}} = 
{\Lambda_\mu}^\nu A_\nu$ are linearly related to the coefficients of the 
transformation ${A'}^\mu = {\Lambda^\mu}_\nu A^\nu$.
})

From two contravariant vectors $A^\mu$ and $B^\mu$ we may form the sixteen 
quantities $A^\mu B^\nu$. The suffix $\nu$, like all the Greek suffixes 
appearing in this work, also takes on the four values 0, 1, 2, 3. Those sixteen 
quantities form the components of a tensor of the second rank. It is sometimes 
called the outer product of the vectors $A^\mu$ and $B^\nu$, as distinct of the 
scalar product (\ref{ecuacion 1.3}) which is called the inner product.

The tensor $A^\mu B^\nu$ is rather a special tensor because there are special 
relations between its components. But we can add together several tensors 
constructed in this way to get a general tensor of the second rank; say
\begin{equation}
 \label{ecuacion 1.5}
 T^{\mu\nu} = A^\mu B^\nu + {A'}^\mu {B'}^\nu + {A''}^\mu {B''}^\nu + \cdots
\end{equation}
The important thing about the general tensor is that under a transformation of 
coordinates its components transform in the same way as the components of $A^\mu 
B^\nu$.

We may lower one of the suffixes in $T^{\mu\nu}$ by applying the lowering 
process to each of the terms on the right hand side of (\ref{ecuacion 1.5}). 
Thus we may form ${T_\mu}^\nu$ or ${T^\mu}_\nu$.

In ${T_\mu}^\nu$ we may set $\nu=\mu$ and get ${T_\mu}^\mu$. This is to be 
summed over the four values of $\mu$. A summation is always implied over a 
suffix that occurs twice in a term. Thus ${T_\mu}^\mu$ is scalar. It is equal to 
${T^\mu}_\mu$.

We may continue this process and multiply more than two vectors together, 
taking care that their suffixes are all different. In this way we can construct 
tensors of higher rank. If the vectors are all contravariant, we get a vector 
with all its suffixes upstairs. We may then lower any of the suffixes and so get 
a general tensor with an number of suffixes uppstairs and any number downstairs.

We may set a downstsir suffix equal to an upstairs one. We then have to sum 
over all values of this suffix. The suffix becomes a dummy. We are left with a 
tensor having two fewer effective suffixes than the original one. This process 
is called \emph{contraction}. Thus, if we start with the fourth rank tensor 
${{T_\mu}_{\nu\rho}}^\sigma$, one way of contracting it is to put $\sigma = 
\rho$, which gives the second rank tensor ${{T_\mu}_{\nu\rho}}^\rho$, having 
only sixteen components, arising from the four values of $\mu$ and $\nu$. We 
could contract again to get the scalar ${{T_\mu}_{\mu\rho}}^\rho$, with just one 
component.

At this stage one can appreciate the balancing of suffixes. Any effective 
suffix occurring in an equation appears once and only once in each term of the 
equation, and always upstairs or only downstairs. A suffix appearing twice in a 
term is a dummy, and it must occur once upstairs and once downstairs. It may be 
replaced by any other Greek letter not already mentioned in the term. Thus 
${{T_\mu}_{\nu\rho}}^\rho = {{T_\mu}_{\nu\alpha}}^\alpha$. A suffix must never 
occur more than twice in a term.
