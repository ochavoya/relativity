Before passing to the formalism of general relativity it is convenient to 
consider an intermediate formalism--special relativity referred to oblique 
rectilinear axes.

If we make a transformation to oblique axes, each of the $dx^\mu$ mentioned in 
(\ref{ecuacion 1.1}) becomes a linear function the new $dx^\mu$ and the 
quadratic form (\ref{ecuacion 1.1}) becomes a general quadratic form in the new 
$dx^\mu$. We may write it
\begin{equation}
 \label{ecuacion 2.1}
 g_{\mu\nu} dx^\mu dx^\nu
\end{equation}
with summation understood over both $\mu$ and $\nu$. The coefficients 
$g_{\mu\nu}$ appearing here depend on the system of oblique axes. Of course we 
take $g_{\mu\nu} = g_{\nu\mu}$ because any difference of $g_{\mu\nu}$ and 
$g_{\nu\mu}$ would not show up in the quadratic form (\ref{ecuacion 2.1}). There 
are thus ten independent coefficients $g_{\mu\nu}$.\footnote{In general the 
number of independent components in a symmetric matrix of $n\times n$ dimensions 
is equal to the number of unordered pairs of $n$ elements. From the total number 
of ordered pairs we subtract $n$ pairs with identical elements, and we are left 
with $n^2 - n$. We divide the last number by two to obtain the number of 
unordered pairs with distinct elements. The total number of uordered pairs is 
then \[ T_n = \frac{1}{2} n \cdot (n+1)\] which is the $n$th triangular number. 
Then we notice that $T_{n-1}$ is the number of independent components of an 
antisymmetric matrix of $n\times n $ dimensions. In the case of four dimensions 
this number is $T_3 = 6$.}

A general contravariant vector has four components $A^\mu$ which transform like 
the $dx^\mu$ under any transformation of the oblique axes. Thus 
\[
 g_{\mu\nu} A^\mu A^\nu
\]
is invariant. It is the squared length of the vector $A^\mu$. 

Let $B^\mu$ be a second contravariant vector; then $A^\mu + \lambda B^\mu$ is 
still another, for any value of the number $\lambda$. It's squared length is
\[
 g_{\mu\nu}
 \left( A^\mu + \lambda B^\mu \right)
 \left( A^\nu + \lambda B^\nu \right)
 = g_{\mu\nu} A^\mu A^\nu 
 + \lambda \left( g_{\mu\nu} A^\mu B^\nu + g_{\mu\nu} A^\nu B^\mu \right)
 + \lambda^2 g_{\mu\nu} B^\mu B^\nu .
\]
This must be an invariant for all values of $\lambda$. It follows that the term 
independent of $\lambda$ and the coefficients of $\lambda$ and $\lambda^2$ must 
separately be invariants. The coefficient of $\lambda$ is
\[
 g_{\mu\nu} A^\mu B^\nu + g_{\mu\nu} A^\nu B^\mu = 2 g_{\mu\nu} A^\mu B^\nu
\]
since in the second term in the left we can interchange $\mu$ and $\nu$ and 
then set $g_{\mu\nu} = g_{\nu\mu}$. Thus we find that $g_{\mu\nu} A^\mu B^\nu$ 
is an invariant. It is the scalar product of $A^\mu$ and $B^\mu$.

Let $g$ be the determinant of $g_{\mu\nu}$. It must not vanish; otherwise the 
four axes would not provide independent directions in space-time and would no be 
suitable as axes. For the orthogonal axes of the preceding section the diagonal 
elements of $g_{\mu\nu}$ are $1$, $-1$, $-1$, $-1$ and the nondiagonal elements 
are zero. Thus $g=-1$. With oblique axes $g$ must still be negative, because the 
oblique axes can be obtained from the orthogonal ones by a continuous process, 
resulting in $g$ varying continuously, and $g$ cannot pass through the value 
zero.

Define the covariant vector $A_\mu$, with a downstairs suffix, by
\begin{equation}
 \label{ecuacion 2.2}
 A_\mu = g_{\mu\nu} A^\nu.
\end{equation}
 Since the determinant $g$ does not vanish, these equations can be solved for 
$A^\nu$ in terms of the $A_\mu$. Let the result be
\begin{equation}
 \label{ecuacion 2.3}
 A^\nu = g^{\mu\nu} A_\mu.
\end{equation}
Each $g^{\mu\nu}$ equals the cofactor of the corresponding $g_{\mu\nu}$, 
divided by the determinant itself. It follows that $g^{\mu\nu} = g^{\nu\mu}$.

Let us substitute for the $A^\nu$ in (\ref{ecuacion 2.2}) their values given by 
(\ref{ecuacion 2.3}). We must replace the dummy $\mu$ in (\ref{ecuacion 2.3}) by 
some other Greek letter, say $\rho$, in order not to have three $\mu$'s in the 
same term. We get
\[
 A_\mu = g_{\mu\nu} g^{\nu\rho}A_\rho.
\]
Since this equation must hold for any four quantities $A_\mu$, we can infer 
\begin{equation}
 \label{ecuacion 2.4}
 g_{\mu\nu} g^{\nu\rho} = g_{\mu}^{\rho},
\end{equation}
where
\begin{equation}
 \label{ecuacion 2.5}
 \begin{array}{rclr}
    g_{\mu}^{\rho} & = & 1 & \mbox{ for } \mu = \rho,\\
                   & = & 0 & \mbox{ for } \mu \ne \rho.
 \end{array}
\end{equation}

The formula (\ref{ecuacion 2.2}) may be used to lower any upper suffix 
occurring in a tensor. Similarly, (\ref{ecuacion 2.3}) can be used to raise any 
downstairs suffix. If a suffix is lowered and raised agian, he result is the 
same as the original tensor, on account of (\ref{ecuacion 2.4}) and 
\ref{ecuacion 2.5}. Note that $g_{\mu}^{\nu}$ just produces a substitution of 
$\rho$ for $\mu$ or of $\mu$ for $\rho$,
\[
 g_{\mu}^{\rho} A^{\mu} = A^{\rho},
\]
and of $\mu$ for $\rho$,
\[
 g_{\mu}^{\rho} A_{\rho} = A_{\mu}.
\]

if we apply the rule to raising a suffix to the $\mu$ in $g_{\mu\rho}$ we get
\[
 {g^{\alpha}}_{\nu} = g^{\alpha\mu} g_{\mu\nu}.
\]
This agrees with (\ref{ecuacion 2.4}) if we take into accoun that in 
${g^{\alpha}}_{\nu}$ we may write the suffixes one above the other because of 
the symmetry of $g_{\mu\nu}$. Further we may raise the suffix $\nu$ by the same 
rule and get
\[
 g^{\alpha\beta} = g^{\nu\beta} g^{\alpha}_{\nu},
\]
a result which follows immediately form (\ref{ecuacion 2.5}). The rules for 
raising and lowering suffixes apply to all the suffixes in $g_{\mu\nu}$, 
$g^{\mu}_{\nu}$, $g^{\mu\nu}$.
