We now pass on to a system of curvilinear coordinates. We shall deal with 
quantities which are located at a point in space. Such a  quantity may have 
various components, which are then referred to the axes at that point. There may 
be a quantity of the same nature at all points in space. It then becomes a field 
quantity.

If we take such quantity $Q$ (or one of its components if it has several), we 
can differentiate it with respect to any of the four coordinates. We write the 
result
\[
 \frac{\partial Q}{\partial x^\mu} = Q_{,\mu}
\]
A downstairs suffix preceded by a comma will always denote a derivative in this 
way. We put the suffix $\mu$ downstairs in order to balance the upstairs $\mu$ 
in the denominator on the left. We can see that the suffixes balance by noting 
that the change in $Q$, when we pass from a point $x^\mu$ to the neighboring 
point $x^\mu + \delta x^\mu$ is
\begin{equation}
 \label{ecuacion 3.1}
 \delta Q = Q_{,\mu} \delta x^\mu .
\end{equation}

We shall have vectors and tensors located at a point, with various components 
referring to the axes at that point. When we change our system of coordinates, 
the components will change according to the same laws as in the preceding 
section, depending on the change of axes at the point concerned. We shall have a 
$g_{\mu\nu}$ and a $g^{\mu\nu}$ to lower and raise suffixes, as before. But 
\emph{they are no longer constants}. They vary from point to point. They are 
field quantities.

Let us see the effect of a particular change in the coordinate system. Take new 
curvilinear coordinates ${x'}^{\mu}$, each a function of the four 
$x^{\mu}$'s. They may be written more conveniently $x^{\mu'}$, with the prime 
attached to the suffix rather than the main symbol.

Making a small variation in the $x^{\mu}$, we get the four quantities $\delta 
x^{\mu}$ forming a contravariant vector. Referred to the new axes, this vector 
has the components
\[
    \delta {x^{\mu'}} 
    = \frac{\partial x^{\mu'}}{\partial x^{\nu}} \delta x^{\nu}
    = x^{\mu'}_{,\nu} \delta x^\nu
\]
with the notation of (\ref{ecuacion 3.1}). This gives the law for he 
transformation of any contravariant vector $A^{\nu}$; namely,
\begin{equation}
 \label{ecuacion 3.2}
 A^{\mu'} = x^{\mu'}_{,\nu} A^\nu
\end{equation}
Interchanging the two systems of axes and changing the suffixes, we get
\begin{equation}
 \label{ecuacion 3.3}
 A^{\lambda} = x^{\lambda}_{,\mu'} A^{\mu'}
\end{equation}

We know from the laws of partial differentiation that
\begin{equation}
 \label{ecuacion 3.4}
 \frac{\partial x^{\lambda}}{\partial x^{\mu'}} 
\frac{\partial x^{\mu'}}{\partial x^{\nu}} = g^{\lambda}_{\nu}.
\end{equation}

To see how a covariant vector $B_{\mu}$ transforms, we use the condition that 
$A^{\mu} B_{\mu}$ is invariant. Thus with the help of (\ref{ecuacion 3.3})
\[
    A^{\mu'} B_{\mu'} 
    = A^{\lambda} B_{\lambda} 
    = x^{\lambda}_{,\mu'}A^{\mu'} B_{\lambda}
\]
This result must hold for all values of the four $A^{\mu'}$; therefore we can 
equate the coefficients of $A^{\mu'}$ and get
\begin{equation}
 \label{ecuacion 3.5}
 B_{\mu'} = x^{\lambda}_{,\mu'} B_{\lambda}
\end{equation}

We can now use the formulas (\ref{ecuacion 3.2}) and (\ref{ecuacion 3.5}) to 
transform any tensor with any upstairs and downstairs suffixes. We just have to 
use coefficients like $x^{\mu'}_{,\nu}$ for each upstairs suffix and 
$x^{\lambda}_{,\nu'}$ for each downstairs suffix and make all the suffixes 
balance. For example
\begin{equation}
 \label{ecuacion 3.6}
 {T^{\alpha'\beta'}}_{\gamma'} =
    x^{\alpha'}_{,\lambda}x^{\beta'}_{,\mu}x^{\nu}_{,\gamma'} 
    {T^{\lambda\mu}}_{\nu}.
\end{equation}
Any quantity that transforms according to this law is a tensor. This may be 
taken as the definition of a tensor.

It should be noted that it has a meaning for a tensor to be symmetrical or 
antisymmetrical between two suffixes like $\lambda$ and $\mu$, because this 
property of symmetry is preserved with the change of coordinates.(\footnote{
In fact if $T^{\mu\nu}$ is tensor and ($T^{\mu\nu} = \pm T^{\nu\mu}$). By 
the transformation law 
\[T^{\mu'\nu'} = x^{\mu'}_{,\alpha} x^{\nu'}_{,\beta}T^{\alpha\beta}.\]
Using the property of symmetry
\[T^{\mu'\nu'} = \pm x^{\mu'}_{,\alpha} x^{\nu'}_{,\beta}T^{\beta\alpha}
               = \pm x^{\nu'}_{,\beta}  x^{\mu'}_{,\alpha}T^{\beta\alpha}\]
and reordering the dummy suffixes
\[T^{\mu'\nu'} = \pm x^{\nu'}_{,\alpha} x^{\mu'}_{,\beta}T^{\alpha\beta}
               = \pm T^{\nu'\mu'}.\]
})

The formula (\ref{ecuacion 3.4}) may be written
\[
    x^{\lambda}_{,\alpha'}x^{\beta'}_{,\nu} 
    g^{\alpha'}_{\beta'} = g^{\lambda}_{\nu}
\]
It just shows that $g^{\lambda}_{\nu}$ is a tensor. We have also, for any 
vectors $A^\mu$, $B^\nu$,
\[
    g_{\alpha'\beta'} A^{\alpha'} A^{\beta'} = g_{\mu\nu} A^\mu B^\nu
    = g_{\mu\nu} x^{\mu}_{,\alpha'} x^{\nu}_{,\beta'} A^{\alpha'} A^{\beta'}.
\]
Since this holds for all values of $A^{\alpha'}$, $B^{\beta'}$, we can infer
\begin{equation}
 \label{ecuacion 3.7}
 g_{\alpha'\beta'} = x^{\mu}_{,\alpha'} x^{\nu}_{,\beta'} g_{\mu\nu}.
\end{equation}
This shows that $g_{\mu\nu}$ is a tensor. Similarly, $g^{\mu\nu}$ is a tensor. 
They are called the \emph{fundamental tensors}.

If $S$ is any scalar field quantity, it can be considered as a function of the 
four $x^\mu$ or the four $x^{\mu'}$. From the laws of partial differentiation 
\[
    S_{,\mu'} = S_{,\lambda} x^{\lambda}_{,\mu'}.
\]
Hence the $S_{,\lambda}$ transform like the $B_{\lambda}$ of equations 
(\ref{ecuacion 3.5}) and thus \emph{the derivative of a scalar field is a 
covariant vector field.}(\footnote{As a matter of fact, a covariant 
(cogradient) vector can be defined as an array of four quantities $A_{\mu}$ that 
transform like the components of the gradient of a scalar.  A contravariant 
(contragradient) vector can be defined as an array of four quantities that 
transform like the differentials $dx^\mu$.})


