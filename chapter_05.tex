One can easily imagine a curved two-dimensional space as a surface immersed in Euclidean three-dimensional space. In the same way, one can have a curved four-dimensional space immersed in a flat space of a larger number of dimensions. Such a curved space is called a Riemann space. A small region of it is approximately flat.

Einstein assumed that physical space is of this nature and thereby laid the foundation for his theory of gravitation.

For dealing with curved space one cannot introduce a rectilinear system of axes. One has to use curvilinear coordinates, such as those dealt with in Section 3. The whole formalism of that section can be applied to curved space, because all the equations are local ones which are not disturbed by the curvature.

The invariant distance $ds$ between a point $x^\mu$ and a neighboring point $x^\mu + dx^\mu$ is given by 
\[
 ds^2 = g_{\mu\nu}dx^\mu dx^\nu
\]
like (\ref{ecuacion 2.1}). $ds$ is real for a timelike interval and imaginary for a spacelike interval.

With a network of curvilinear coordinates the $g_{\mu\nu}$, given as functions of the coordinates, fix all the elements of distance; so they fix the metric. They determine both the coordinate system and the curvature of the space.