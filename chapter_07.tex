By differentiating (\ref{ecuacion 6.3}) we get (\textbf{omitting the second 
comma with two differentiations})
\begin{equation}
 \label{ecuacion 7.1}
 \begin{array}{rcl}
g_{\mu\nu,\sigma} & = & y^n_{,\mu\sigma} y_{n,\nu} +
                    y^n_{,\mu} y_{n,\nu\sigma} \\
                  & = & y_{n,\mu\sigma} y^n_{,\nu} +
                     y_{n,\nu\sigma} y^n_{,\mu}
\end{array}
\end{equation}

since we can move the suffix $n$ freely up and down, on account of the 
constancy of the $h_{mn}$. Interchanging $\mu$ and $\sigma$ in (\ref{ecuacion 
7.1}) we get 
\begin{equation}
 \label{ecuacion 7.2}
 g_{\sigma\nu,\mu} = y_{n,\mu\sigma} y^n_{,\nu} + y_{n,\nu\mu} y^n_{,\sigma} 
\end{equation}

Interchanging $\nu$ and $\sigma$ in (\ref{ecuacion 7.1})
\begin{equation}
 \label{ecuacion 7.3}
 g_{\mu\sigma,\nu} = y_{n,\mu\nu} y^n_{,\sigma} + y_{n,\sigma\nu} y^n_{,\mu} 
\end{equation}
Now take (\ref{ecuacion 7.1})$+$(\ref{ecuacion 7.3})$-$(\ref{ecuacion 7.2}) and 
divide by $2$. The result is
\begin{equation}
 \label{ecuacion 7.4}
 \frac{1}{2} \left( g_{\mu\nu,\sigma} + g_{\mu\sigma,\nu} - g_{\nu\sigma,\mu}  
\right) = y_{n,\nu\sigma}y^n_{,\mu}.
\end{equation}

Put
\begin{equation}
 \label{ecuacion 7.5}
 \Gamma_{\mu\nu\sigma} = \frac{1}{2}\left(
    g_{\mu\nu,\sigma} + g_{\mu\sigma,\nu} - g_{\nu \sigma,\mu}
 \right)
\end{equation}
It is called a Christoffel symbol of the first kind. It is symmetrical with 
respect to the last two suffixes. It is a nontensor. A simple consequence of 
(\ref{ecuacion 7.5}) is
\begin{equation}
 \label{ecuacion 7.6}
 \Gamma_{\mu\nu\sigma} + \Gamma_{\nu\mu\sigma} = g_{\mu\nu,\sigma}.
\end{equation}
We see now that (\ref{ecuacion 6.7}) can be written as
\begin{equation}
 \label{ecuacion 7.7}
 d A_{\nu} = A^\mu \Gamma_{\mu\nu\sigma} dx^\sigma.
\end{equation}
Al reference to the $N$-dimensional space has now disappeared, as the 
Christoffel symbol involves only the metric $g_{\mu\nu}$ of physical space.

We can infer that the length of a vector is unchanged by parallel displacement. 
We have
\begin{equation}
 \label{ecuacion 7.8}
 \begin{array}{rcl}
 d\left( g^{\mu\nu} A_\mu A_\nu \right) & = & 
    g^{\mu\nu} A_\nu d A_\mu + g^{\mu\nu} A_\mu d A_\nu + A_\mu A_\nu  
    {g^{\mu\nu}}_{,\sigma} dx^\sigma \\
    & = & A^\mu d A_\mu + A^\nu d A_\nu 
      + A_\mu A_\nu {g^{\mu\nu}}_{,\sigma} dx^\sigma \\
    & = & A^\mu A^\nu \Gamma_{\nu\mu\sigma} dx^\sigma 
    + A^\nu A^\mu \Gamma_{\mu\nu\sigma} dx^\sigma
    + A_\mu A_\nu {g^{\mu\nu}}_{,\sigma} dx^\sigma \\
    & = & A^\mu A^\nu g_{\mu\nu,\sigma} dx^\sigma
    + A_\mu A_\nu {g^{\mu\nu}}_{,\sigma} dx^\sigma
 \end{array}
\end{equation}

Now ${g^{\alpha\mu}}_{,\sigma} g_{\mu\nu} + g^{\alpha\mu} g_{\mu\nu,\sigma} = 
\left(g^{\alpha\mu}g_{\mu\nu}\right)_{,\sigma} = g^{\alpha}_{\nu,\sigma} = 0$. 
Multiplying by $g^{\beta\nu}$, we get
\begin{equation}
 \label{ecuacion 7.9}
 {g^{\alpha\beta}}_{,\sigma} = - g^{\alpha\mu} g^{\beta\nu}g_{\mu\nu,\sigma}.
\end{equation}
This is a useful formula giving the derivative of $g^{\alpha\beta}$ in terms 
of the derivative of $g_{\mu\nu}$. It allows us to infer
\[
 A_\alpha A_{\beta} {g^{\alpha\beta}}_{,\sigma} = - A^\mu A^\nu 
g_{\mu\nu,\sigma}
\]
and so the expression (\ref{ecuacion 7.8}) vanishes. Thus the length of a 
vector is constant. In particular, a null vector (i.e. a vector of zero length) 
remains a null vector under parallel displacement.

The constancy of the length of the vector follows also from geometrical 
arguments. When we split up the vector $A^\mu$ into tangential and normal parts 
according to (\ref{ecuacion 6.5}), the normal part is infinitesimal and is 
orthogonal to the tangential part. It follows that, to the first order, the 
length of the whole vector equals that of its tangential part.

The constancy of the length of any vector requires the constancy of the scalar 
product $g^{\mu\nu} A_\mu B_\nu$ of any two vectors $A$ and $B$. This can be 
inferred from the constancy of the length of $A + \lambda B$ for any value of 
the parameter $\lambda$.

It is frequently useful to raise the first suffix of the Christoffel symbol so 
as to form
\[
 \Gamma^\mu_{\nu\sigma} = g^{\mu\lambda}\Gamma_{\lambda\mu\sigma}.
\]
It is then called a Christoffel symbol of the second kind. It is symmetrical 
between its two lower suffixes. As explained in Section 4, this raising is 
quite permissible, even for a nontensor.

The formula (\ref{ecuacion 7.7}) may be written
\begin{equation}
 \label{ecuacion 7.10}
 dA_{\nu} = \Gamma^\mu_{\nu\sigma} A_\mu dx^\sigma
\end{equation}
It is the standard formula referring to covariant components. For a second 
vector $B^\nu$ we have
\[
\begin{array}{rcl}
d(A_\nu B^\nu) & = & 0 \\
A_\nu d B^\nu  & = & - B^\nu d A_\nu 
                 = - B^\nu \Gamma^\mu_{\nu\sigma} A_\mu dx^\sigma \\
               & = & - B^\mu \Gamma^\nu_{\mu\sigma} A_\nu dx^\sigma
\end{array}
\]
This must hold for any $A_\nu$, so we get
\begin{equation}
 \label{ecuacion 7.11}
 d B^\nu = - \Gamma^\nu_{\mu\sigma} B^\mu dx^\sigma.
\end{equation}
This is the standard formula for parallel displacement referring to 
contravariant components.









