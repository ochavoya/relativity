Take a point with coordinates $z^\mu$ and suppose it moves along a track; we 
then have a function of some parameter $\tau$. Put $dz^\mu / d\tau = 
u^\mu$.

There is a vector $u^\mu$ at each point of the track. Suppose that as we go 
along the track the vector $u^\mu$ gets shifted by parallel displacement. Then 
the whole track is determined if we are given the initial point and the initial 
value of the vector $u^\mu$. We just have to shift the initial point from 
$z^\mu$ to $z^\mu + u^\mu d\tau$, then shift the vector $u^\mu$ to this new 
point by parallel displacement, then shift the point again in the direction 
fixed by the new  $u^\mu$, and so on. Not only is the track determined, but 
also the parameter $\tau$ along it. A track produced this way is called a 
geodesic.

If the vector $u^\mu$ is a null vector, it always remains a null vector and the 
track is called a null geodesic. If the vector $u^\mu$ is initially timelike 
(i.e., $u^\mu u_\mu > 0$), it is always timelike and we have a timelike 
geodesic. Similarly if $u^\mu$ is initially spacelike ($u^\mu u_\mu < 0$), it 
is always spacelike and we have a spacelike geodesic.

We get the equations of a geodesic by applying (\ref{ecuacion 7.11}) with 
$B^\nu = u^\nu$ and $dx^\sigma = dz^\sigma$. Thus 
\begin{equation}
 \label{ecuacion 8.1}
 \frac{d u^\nu}{d\tau} 
       + \Gamma^\nu_{\mu\sigma} u^\mu \frac{dz^\sigma}{d\tau} = 0
\end{equation}
or
\begin{equation}
 \label{ecuacion 8.2}
 \frac{d^2 z }{d \tau^2} =
      + \Gamma^\nu_{\mu\sigma} \frac{dz^\mu}{d\tau} \frac{dz^\sigma}{d\tau}  
      = 0
\end{equation}

For a timelike geodesic we may multiply the initial $u^\nu$ by a factor so as 
to make its length unity. This merely requires a change in the scale of $\tau$. 
The vector $u^\mu$ now always has the length unity. It is just the velocity 
vector $v^\mu = d z^\mu / ds$, and the parameter $\tau$ has become the proper 
time $s$.

Equation (\ref{ecuacion 8.1}) then becomes
\begin{equation}
 \label{ecuacion 8.3}
 \frac{dv^\mu}{ds} + \Gamma^\mu_{\nu\sigma} u^\nu u^\sigma = 0.
\end{equation}
Equation (\ref{ecuacion 8.2}) becomes 
\begin{equation}
 \label{ecuacion 8.4}
 \frac{d^2 z^\mu}{d s^2} + 
      \Gamma^\mu_{\nu\sigma}\frac{dz^\nu}{ds}\frac{dz^\sigma}{ds} = 0.
\end{equation}

We make the physical assumption that the world line of a particle not acted on 
by any forces, except gravitational, is a timelike geodesic. This replaces 
Newton's first law of motion. Equation (\ref{ecuacion 8.4}) fixes the 
acceleration and provides the equations of motion.

We also make the assumption that the path of a ray of light is a null geodesic. 
It is fixed by equation (\ref{ecuacion 8.2}) referring to some parameter along 
the path. The proper time $s$ cannot now be used because $ds$ vanishes.

