A gepdesic that is not a null geodesic has the property that $\int ds$, taken 
along a sectio of the track with the end points $P$ and $Q$, is stationary if 
one makes a small variation of the tack keeping the end points fixed.

Let us suppose that each point of the track, with the coordinates $z^\mu$ is 
shifted so that its coordinates become $z^\mu + \delta z^\mu$. If $dx^\mu$ 
denotes an element along the track,
\[
 ds^2 = g_{ \mu \nu } dx^\mu dx^\nu
\]
Thus
\[
\begin{array}{rcl}
 2 ds \delta ds & = & d x ^\mu dx^\nu \delta g_{\mu\nu} 
    + g_{\mu\nu} dx^\mu \delta dx^\nu
    + g_{\mu\nu} dx^\nu \delta dx^\mu  \\
    & = & d x ^\mu dx^\nu \delta g_{\mu\nu}  
    + 2 g_{\mu\lambda}dx^\mu \delta dx^\lambda .
\end{array}
\]
Now
\[
\delta dx^\lambda = d (\delta x^\lambda).
\]
Thus, with the help of $dx^\mu = v^\mu ds$,
\[
    \delta(ds) = \left(
    \frac{1}{2} g_{\mu\nu,\lambda}v^\mu v^\nu \delta x^\lambda  
    + g_{\mu\lambda}v^\mu \frac{d\delta x^\lambda}{ds}
    \right) ds
\]
Hence
\[
\delta \int ds = \int{\delta (ds)}
 = \int \left[
    \frac{1}{2} g_{\mu\nu,\lambda}v^\mu v^\nu \delta x^\lambda  
    + g_{\mu\lambda}v^\mu \frac{d\delta x^\lambda}{ds}
    \right] ds
\]
By partial integration (of the second term), using the condition that $\delta 
x^\lambda = 0$ at the end points $P$ and $Q$, we get
\begin{equation}
 \label{ecuacion 9.1}
 \delta \int ds 
 = \int \left[
    \frac{1}{2} g_{\mu\nu,\lambda}v^\mu v^\nu 
    -   \frac{d}{ds} \left(g_{\mu\lambda}v^\mu\right)
    \right] \delta x^\lambda   ds
\end{equation}
The condition for this to vanish with arbitrary $\delta x^\lambda$ is
\begin{equation}
 \label{ecuacion 9.2}
 \frac{d}{ds}\left(g_{\mu\lambda} v^\mu\right) 
 - \frac{1}{2}g_{\mu\nu,\lambda} = 0
\end{equation}
Now
\[
\begin{array}{rcl}
 \frac{d}{ds}\left(g_{\mu\lambda} v^\mu\right) & = &  
  g_{\mu\lambda}\frac{dv^\mu}{ds} + g_{\mu\lambda,\nu}v^\mu v^\nu \\
  & = & g_{\mu\lambda}\frac{dv^\mu}{ds} + 
  \frac{1}{2}\left( g_{\lambda\mu,\nu} + g_{\lambda\nu,\mu} \right) v^\mu v^\nu 
.
\end{array}
\]
Thus the condition (\ref{ecuacion 9.2}) becomes
\[
 g_{\mu\lambda}\frac{dv^\mu}{ds} + \Gamma_{\lambda\mu\nu}v^\mu v^\nu = 0.
\]
Multiplying this by $g^{\lambda\sigma}$, it becomes
\[
 \frac{dv^\sigma}{ds} + \Gamma^\sigma_{\mu\nu}v^\mu v^\nu = 0,
\]
which is just the condition (\ref{ecuacion 8.3}) for the geodesic.

This work shows that for a geodesic, (\ref{ecuacion 9.1}) vanishes and $\int 
ds$ is stationary. Conversely, if we assume that $\int ds$ is stationary, we 
can infer that the track is a geodesic. Thus we may use the stationary 
condition as the definition of a geodesic, except in the case of a null 
geodesic.
