Let $S$ be a scalar field. Its derivative $S,y$ is a covariant vector, as we 
saw in Section 3. Now let $A_\mu$ be a vector field. Is its derivative 
$A_{\mu,\nu}$ a tensor?

We must examine how $A_{\nu,\nu}$ transforms under a change of coordinate 
system. With the notation in Section 3, $A_\mu$ transforms to
\[
 A_{\mu'} = A_\rho x^\rho _{,\mu'}
\]
like equation (\ref{ecuacion 3.5}), and hence
\[
 \begin{array}{rcl}
  A_{\mu',\nu'} & = & \left( A_\rho x^\rho _{,\mu'} \right)_{,\nu'}\\
                & = & A_{\rho,\sigma} x^\sigma_{,\nu'} x^\rho_{,\mu'}
                + A_\rho x^\rho_{,\mu'\nu'} .
 \end{array}
\]
The last term should not be here if we were to have the correct transformation 
law for a tensor. Thus $A_{\mu,\nu}$ is not a tensor.

We can, however, modify the process of differentiation so as to get a tensor. 
Let us take the vector $A_\mu$ at the point $x$ and shift it to $x+dx$ by 
parallel displacement. It is still a vector. We may subtract it from the vector 
$A_\mu$ at $x+dx$ and the difference will be a vector. It is, to the first order
\[
 A_{\mu}(x + dx) - \left[ A_{\mu}(x) + \Gamma^\alpha_{\mu\nu}A_\alpha 
dx^\nu\right] = \left(A_{\mu,\nu} - \Gamma^\alpha_{\mu\nu} A_\alpha 
\right)dx^\nu .
\]
This quantity is a vector, for any vector $dx^\nu$; hence, by the quotient 
theorem of Section 4, the coefficient
\[
 A_{\mu\nu} - \Gamma^\alpha_{\mu\nu} A_\alpha
\]
is a tensor. One can easily verify directly that it transforms correctly under 
a change of coordinate system.

It is called the covariant derivative of $A_\mu$ and written
\begin{equation}
 \label{ecuacion 10.1}
 A_{\mu:\nu} = A_{\mu,\nu} - \Gamma^\alpha_{\mu\nu}A_\alpha. 
\end{equation}

The sign $:$ before a lower suffix will always denote a covariant derivative, 
just as the comma denotes an ordinary derivative.

Let $B_{\nu}$ be a second vector. We define the outer product $A_{\mu}B_{\nu}$ 
to have the covariant derivative(\footnote{Notice that the definition of the  
covariant derivative of a second rank tensor is not motivated because the 
notion of \emph{parallel displacement of tensor} doesn't make sense. })

\begin{equation}
 \label{ecuacion 10.2}
 \left(A_{\mu} B_{\nu} \right)_{\sigma} =
     A_{\mu:\sigma}B_{\nu} + A_{\mu} B_{\nu:\sigma}
\end{equation}

Evidently it is a tensor with three suffixes. It has the value
\[
  \left(A_{\mu} B_{\nu} \right)_{\sigma} =
  \left(A_{\mu,\sigma} - \Gamma^{\alpha}_{\mu\sigma} A_{\alpha}\right)B_{\nu} 
+ A_{\mu} \left(B_{\nu,\sigma} - \Gamma^{\alpha}_{\nu\sigma}B_{\alpha}\right)
\]

Let $T_{\mu\nu}$ be a tensor with two suffixes. It is expressible as a sum of 
terms like $A_{\mu}B_{\nu}$, so, its covariant derivative is
\begin{equation}
 \label{ecuacion 10.3}
 T_{\mu\nu:\sigma} = T_{\mu\nu,\sigma}
 - \Gamma^{\alpha}_{\mu\sigma} T_{\alpha\nu}
 - \Gamma^{\alpha}_{\nu\sigma} T_{\mu\alpha}.
\end{equation}


The rule can be extended to the covariant derivative of a tensor 
$Y_{\mu\nu\cdots}$ with any number of suffixes downstairs
\begin{equation}
 \label{ecuacion 10.4}
 Y_{\mu\nu\cdots:\sigma} = Y_{\mu\nu\cdots,\sigma} - \mbox{ a }
 \Gamma \mbox{ term for each suffix}.
\end{equation}
In each of these $\Gamma$ terms we must make the suffixes balance, which is 
sufficient to fix how the suffixes go.

The case of a scalar is included in the general formula (\ref{ecuacion 10.4}) 
with the number of suffixes in $Y$ zero.
\begin{equation}
 \label{ecuacion 10.5}
 Y_{:\sigma} = Y_{,\sigma} .
\end{equation}

Let's apply (\ref{ecuacion 10.3}) to the fundamental tensor $g_{\mu\nu}$. It 
gives
\[
\begin{array}{rcl}
    g_{\mu\nu:\sigma} & = & g_{\mu\nu,\sigma} 
   - \Gamma^{\alpha}_{\mu\sigma} g_{\alpha,\nu}
   - \Gamma^{\alpha}_{\nu\sigma} g_{\mu\alpha} \\
   & = & g_{\mu\nu,\sigma} - \Gamma_{\nu\mu\sigma} - \Gamma_{\nu\mu\sigma} = 0
\end{array}
\]
from (\ref{ecuacion 7.6}). Thus the $g_{\mu\nu}$, count as constants under 
covariant differentiation.

