Let $S$ be a scalar field. Its derivative $S,y$ is a covariant vector, as we 
saw in Section 3. Now let $A_\mu$ be a vector field. Is its derivative 
$A_{\mu,\nu}$ a tensor?

We must examine how $A_{\nu,\nu}$ transforms under a change of coordinate 
system. With the notation in Section 3, $A_\mu$ transforms to
\[
 A_{\mu'} = A_\rho x^\rho _{,\mu'}
\]
like equation (\ref{ecuacion 3.5}), and hence
\[
 \begin{array}{rcl}
  A_{\mu',\nu'} & = & \left( A_\rho x^\rho _{,\mu'} \right)_{,\nu'}\\
                & = & A_{\rho,\sigma} x^\sigma_{,\nu'} x^\rho_{,\mu'}
                + A_\rho x^\rho_{,\mu'\nu'} .
 \end{array}
\]
The last term should not be here if we were to have the correct transformation 
law for a tensor. Thus $A_{\mu,\nu}$ is not a tensor.

We can, however, modify the process of differentiation so as to get a tensor. 
Let us take the vector $A_\mu$ at the point $x$ and shift it to $x+dx$ by 
parallel displacement. It is still a vector. We may subtract it from the vector 
$A_\mu$ at $x+dx$ and the difference will be a vector. It is, to the first order
\[
 A_{\mu}(x + dx) - \left[ A_{\mu}(x) + \Gamma^\alpha_{\mu\nu}A_\alpha 
dx^\nu\right] = \left(A_{\mu,\nu} - \Gamma^\alpha_{\mu\nu} A_\alpha 
\right)dx^\nu .
\]
This quantity is a vector, for any vector $dx^\nu$; hence, by the quotient 
theorem of Section 4, the coefficient
\[
 A_{\mu\nu} - \Gamma^\alpha_{\mu\nu} A_\alpha
\]
is a tensor. One can easily verify directly that it transforms correctly under 
a change of coordinate system.

It is called the covariant derivative of $A_\mu$ and written
\begin{equation}
 \label{ecuacion 10.1}
 A_{\mu:\nu} = A_{\mu,\nu} - \Gamma^\alpha_{\mu\nu}A_\alpha. 
\end{equation}

The sign $:$ before a lower suffix will always denote a covariant derivative, 
just as the comma denotes an ordinary derivative.

Let $B_{\nu}$ be a second vector. We define the outer product $A_{\mu}B_{\nu}$ 
to have the covariant derivative(\footnote{Notice that the definition of the  
covariant derivative of a second rank tensor is not motivated because the 
notion of \emph{parallel displacement of tensor} doesn't make sense. })

\begin{equation}
 \label{ecuacion 10.2}
 \left(A_{\mu} B_{\nu} \right)_{\sigma} =
     A_{\mu:\sigma}B_{\nu} + A_{\mu} B_{\nu:\sigma}
\end{equation}

Evidently it is a tensor with three suffixes. It has the value
\[
  \left(A_{\mu} B_{\nu} \right)_{\sigma} =
  \left(A_{\mu,\sigma} - \Gamma^{\alpha}_{\mu\sigma} A_{\alpha}\right)B_{\nu} 
+ A_{\mu} \left(B_{\nu,\sigma} - \Gamma^{\alpha}_{\nu\sigma}B_{\alpha}\right)
\]

Let $T_{\mu\nu}$ be a tensor with two suffixes. It is expressible as a sum of 
terms like $A_{\mu}B_{\nu}$, so, its covariant derivative is
\begin{equation}
 \label{ecuacion 10.3}
 T_{\mu\nu:\sigma} = T_{\mu\nu,\sigma}
 - \Gamma^{\alpha}_{\mu\sigma} T_{\alpha\nu}
 - \Gamma^{\alpha}_{\nu\sigma} T_{\mu\alpha}.
\end{equation}


The rule can be extended to the covariant derivative of a tensor 
$Y_{\mu\nu\cdots}$ with any number of suffixes downstairs
\begin{equation}
 \label{ecuacion 10.4}
 Y_{\mu\nu\cdots:\sigma} = Y_{\mu\nu\cdots,\sigma} - \mbox{ a }
 \Gamma \mbox{ term for each suffix}.
\end{equation}
In each of these $\Gamma$ terms we must make the suffixes balance, which is 
sufficient to fix how the suffixes go.

The case of a scalar is included in the general formula (\ref{ecuacion 10.4}) 
with the number of suffixes in $Y$ zero.
\begin{equation}
 \label{ecuacion 10.5}
 Y_{:\sigma} = Y_{,\sigma} .
\end{equation}

Let's apply (\ref{ecuacion 10.3}) to the fundamental tensor $g_{\mu\nu}$. It 
gives
\[
\begin{array}{rcl}
    g_{\mu\nu:\sigma} & = & g_{\mu\nu,\sigma} 
   - \Gamma^{\alpha}_{\mu\sigma} g_{\alpha,\nu}
   - \Gamma^{\alpha}_{\nu\sigma} g_{\mu\alpha} \\
   & = & g_{\mu\nu,\sigma} - \Gamma_{\nu\mu\sigma} - \Gamma_{\nu\mu\sigma} = 0
\end{array}
\]
from (\ref{ecuacion 7.6}). Thus the $g_{\mu\nu}$, count as constants under 
covariant differentiation.

Formula (\ref{ecuacion 10.2}) is the usual rule that one uses for 
differentiating a product. We assume this usual rule holds also for the 
covariant derivative of the scalar product of two vectors. Thus
\[
 \left( A^{\mu} B_{\mu}\right):{\sigma} =  
 A^{\mu}_{:\sigma}B_{\mu} +  A^{\mu} B_{\mu:\sigma}.
\]
We get, according to (\ref{ecuacion 10.5}) and (\ref{ecuacion 10.1}),
\[
 \left( A^{\mu} B_{\mu}\right){:\sigma} =  
 A^{\mu}_{:\sigma}B_{\mu} 
 + A^{\mu} \left( B_{\mu,\sigma} -\Gamma^{\alpha}_{\mu\sigma} B_{\alpha} 
\right).
\]
and hence 
\[
 A^{\mu}_{,\sigma} B_{\mu} = A^{\mu}_{:\sigma}B_{\mu} 
 - A^{\alpha}\Gamma^{\mu}_{\alpha\sigma} B_{\mu}.
\]
Since this holds for any $B_{\mu}$, we get
\begin{equation}
 \label{ecuacion 10.7}
   A^{\mu}_{:\sigma}B_{\mu} = A^{\mu}_{,\sigma} B_{\mu}  
 + A^{\alpha}\Gamma^{\mu}_{\alpha\sigma} B_{\mu},
\end{equation}
which is the basic formula for the covariant derivative of a contravariant 
vector. The same Christoffel symbol occurs as  in the basic formula 
(\ref{ecuacion 10.1}) for a covariant vector, but now there is a $+$ sign. The 
arrangement of suffixes is completely determined by the balancing requirement.

We can extend the formalism so as to include the covariant derivative of any 
tensor with any number of upstairs or downstairs suffixes. A $\Gamma$ terms 
appears for each suffix, with a $+$ sign if the suffix is upstairs an a $-$ 
sign if it is downstairs. If we contract two suffixes in the tensor, the 
corresponding $\Gamma$ terms cancel.

The formula for the covariant derivative of a product,
\begin{equation}
 \label{ecuacion 10.8}
 \left(X Y\right)_{:\sigma} = X_{:\sigma} Y + X Y_{:\sigma},
\end{equation}
holds quite generally, with $X$ and $Y$ any kind of tensor quantities. On 
account of the $g_{\mu\nu}$ counting as constants, we can shift suffixes up or 
down before covariant differentiation and the result is the same as if we 
shifted them afterwards.

The covariant derivative of a nontensor has no meaning.

The laws of physics must be valid in all systems of coordinates. Thus must thus 
be expressible as tensor equations. Whenever they involve the derivative of a 
field quantity, it must be a covariant derivative. The field equations of 
physics must all be rewritten with the ordinary derivatives replaced by 
covariant derivatives. For example, the d'Alembert equation $\Box V = 0$ for a 
scalar $V$ becomes, in covariant form
\[
 g^{\mu\nu} V_{:\mu:\nu} = 0.
\]
This gives, from (\ref{ecuacion 10.1}) and (\ref{ecuacion 10.5}),
\begin{equation}
 \label{ecuacion 10.9}
 g^{\mu\nu}\left( V_{,\mu\nu} - \Gamma^{\alpha}_{\mu\nu}V_{\alpha} = 0.\right)
\end{equation}

Even if one is working with flat space (which means neglecting the 
gravitational field) and one is using curvilinear coordinates, one must write 
one's equations in terms of covariant derivatives if one wants them to hold in 
all systems of coordinates.
