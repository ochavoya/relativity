With the product law (\ref{ecuacion 10.8}) we see that covariant 
differentiation is very similar to ordinary differentiation. But there is an 
important property of ordinary differentiation, that if we perform two 
differentiations in succession their order does not matter, which does not, in 
general, hold for covariant differentiation.

Let us first consider a scalar field $S$. We have from the formula 
(\ref{ecuacion 10.1}),
\begin{equation}
 \label{ecuacion 11.1}
 \begin{array}{rcl}
 S_{:\mu:\nu} & = & S_{:\mu,\nu} - \Gamma^{\alpha}_{\mu\nu} S_{\alpha} \\
      & = & S_{,\mu\nu} - \Gamma^{\alpha}_{\mu\nu} S_{,\alpha}.
 \end{array}
\end{equation}
\textbf{This is symmetrical between $\mu$ and $\nu$, so in this 
case the order of the covariant differentiation does not matter.}

Now let us take a vector $A_{\nu}$ and apply two covariant differentiations to 
it. From the formula (\ref{ecuacion 10.3}) with $A_{\nu:\rho}$ for 
$T_{\nu\rho}$ we get
\[
 \begin{array}{rcl}
  A_{\nu:\rho:\sigma} & = & A_{\nu:\rho,\sigma} 
  - \Gamma^{\alpha}_{\nu\sigma} A_{\alpha:\rho}
  - \Gamma^{\alpha}_{\rho\sigma} A_{\nu:\alpha} \\
  & = &  \left(A_{\nu,\rho} - 
              \Gamma^{\alpha}_{\nu\rho} A_{\alpha} \right)_{,\sigma}
  - \Gamma^{\alpha}_{\nu\sigma} \left( 
      A_{\alpha,\rho} - \Gamma^{\beta}_{\alpha\rho} A_{\beta}
  \right)
  - \Gamma^{\alpha}_{\rho\sigma}  \left( 
      A_{\nu,\alpha} - \Gamma^{\beta}_{\nu\alpha} A_{\beta}
  \right)\\
  & = & A_{\nu,\rho\sigma} 
      - \Gamma^{\alpha}_{\nu\rho} A_{\alpha,\sigma}
      - \Gamma^{\alpha}_{\nu\sigma} A_{\alpha,\rho}
      - \Gamma^{\alpha}_{\rho\sigma} A_{\nu,\alpha}\\
      & & 
      -A_{\beta}\left(
        \Gamma^{\beta}_{\nu\rho,\sigma}
      - \Gamma^{\alpha}_{\nu\sigma}\Gamma^{\beta}_{\alpha\rho}
      - \Gamma^{\alpha}_{\rho\sigma}\Gamma^{\beta}_{\nu\alpha}
      \right)
 \end{array}
\]
Interchange $\rho$ and $\sigma$ here and subtract from the previous expression. 
The result is
\begin{equation}
 \label{ecuacion 11.2}
 A_{\nu:\rho:\sigma} - A_{\nu:\sigma:\rho} =
 A_{\beta}R^{\beta}_{\nu\rho\sigma}
\end{equation}
where
\begin{equation}
 \label{ecuacion 11.3}
 R^{\beta}_{\nu\rho\sigma} = 
 \Gamma^{\beta}_{\nu\sigma,\rho} - \Gamma^{\beta}_{\nu\rho,\sigma}
 + \Gamma^{\alpha}_{\nu\sigma}\Gamma^{\beta}_{\alpha\rho}
 - \Gamma^{\alpha}_{\nu\rho}\Gamma^{\beta}_{\alpha\sigma}
\end{equation}

The left hand of (\ref{ecuacion 11.2}) is a tensor. It follows that the 
right-hand side of (\ref{ecuacion 11.2}) is a tensor. This holds for any vector 
$A_{\beta}$; therefore, by the quotient theorem in Section 4, 
$R^{\beta}_{\nu\rho\sigma}$ is a tensor. It is called the Riemann-Christoffel 
tensor or the curvature tensor.

It has the obvious property 
\begin{equation}
 \label{ecuacion 11.4}
 R^{\beta}_{\nu\rho\sigma} = - R^{\beta}_{\nu\sigma\rho}
\end{equation}
Also we easily see from (\ref{ecuacion 11.3}) that
\begin{equation}
 \label{ecuacion 11.5}
   R^{\beta}_{\nu\rho\sigma} 
 + R^{\beta}_{\rho\sigma\nu} 
 + R^{\beta}_{\sigma\nu\rho} = 0
\end{equation}

Let us lower the suffix $\beta$ and put it as the first suffix. We get
\[
 R_{\mu\nu\sigma\rho} = g_{\mu\beta} R^{\beta}_{\nu\sigma\rho} =
 g_{\mu\beta} \Gamma^{\beta}_{\nu\sigma,\rho} 
 + \Gamma^{\alpha}_{\nu\alpha} \Gamma_{\mu\alpha\rho} - \langle \rho\sigma 
\rangle,
\]
where the symbol $\langle \rho \sigma \rangle$ is used to denote the preceding 
terms with $\rho$ and $\sigma$ interchanged. Thus
\[
 \begin{array}{rcl}
  R_{\mu\nu\sigma\rho} & = & R_{\mu\nu\sigma,\rho} - 
g_{\mu\beta,\rho}\Gamma^{\beta}_{\nu\sigma} + 
\Gamma_{\mu\beta\rho}\Gamma^{\beta}_{\nu\sigma} - \langle \rho\sigma \rangle \\
& = & \Gamma_{\mu\nu\sigma,\rho} - 
\Gamma_{\beta\mu\rho}\Gamma^{\beta}_{\nu\sigma} - \langle \rho\sigma \rangle
 \end{array}
\]
from (\ref{ecuacion 7.6}). So from (\ref{ecuacion 7.5})
\begin{equation}
 \label{ecuacion 11.6}
 R_{\mu\nu\sigma\rho} = \frac{1}{2}
 \left(
       g_{\mu\sigma,\nu\rho} 
     - g_{\nu\sigma,\mu\rho} 
     - g_{\mu\rho,\nu\sigma}
     - g_{\nu\rho,\nu\sigma}
 \right)
 + \Gamma_{\beta\mu\sigma}\Gamma^{\beta}_{\nu\rho}
 - \Gamma_{\beta\mu\rho}\Gamma^{\beta}_{\nu\sigma}
\end{equation}

Some further symmetries now show up; namely,
\begin{equation}
 \label{ecuacion 11.7}
 R_{\mu\nu\rho\sigma} = - R_{\nu\mu\rho\sigma}
\end{equation}
and
\begin{equation}
 \label{ecuacion 11.8}
 R_{\mu\nu\rho\sigma} = R_{\rho\sigma\mu\nu} = R_{\sigma\rho\nu\mu}
\end{equation}
The result of all these symmetries is that, of the 256 components of 
$R_{\mu\nu\rho\sigma}$, only 20 are independent.


