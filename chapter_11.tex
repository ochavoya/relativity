With the product law (\ref{ecuacion 10.8}) we see that covariant 
differentiation is very similar to ordinary differentiation. But there is an 
important property of ordinary differentiation, that if we perform two 
differentiations in succession their order does not matter, which does not, in 
general, hold for covariant differentiation.

Let us first consider a scalar field $S$. We have from the formula 
(\ref{ecuacion 10.1}),
\begin{equation}
 \label{ecuacion 11.1}
 \begin{array}{rcl}
 S_{:\mu:\nu} & = & S_{:u,n} - \Gamma^{\alpha}_{\mu\nu} S_{\alpha} \\
      & = & S_{,\mu\nu} - \Gamma^{\alpha}_{\mu\nu} S_{,\alpha}.
 \end{array}
\end{equation}
\textbf{This is symmetrical between $\mu$ and $\nu$, so in this 
case the order of the covariant differentiation does not matter.}

Now let us take a vector $A_{\nu}$ and apply two covariant differentiations to 
it. From the formula (\ref{ecuacion 10.3}) with $A_{\nu:\rho}$ for 
$T_{\nu\rho}$ we get
\[
 \begin{array}{rcl}
  A_{\nu:\rho:\sigma} & = & A_{\nu:\rho,\sigma} 
  - \Gamma^{\alpha}_{\nu\sigma} A_{\alpha:\rho}
  - \Gamma^{\alpha}_{\rho\sigma} A_{\nu:\alpha} \\
 \end{array}
\]

