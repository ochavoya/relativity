If space is flat, we may choose a system of coordinates that is rectilinear and 
then the $g_{\mu\nu}$ are constant. The tensor $R_{\mu\nu\rho\sigma}$ then 
vanishes.

Conversely, if $R_{\mu\nu\rho\sigma}$ vanishes, one can prove that the space is 
flat. Let us take a vector $A_{\mu}$ situated at the point $x$ and shift it by 
parallel displacement to the point $x+dx$. Then shift it by parallel 
displacement to the point $x+dx+\delta x$. If $R_{\mu\nu\rho\sigma}$ vanishes, 
he result must be the same as if we had shifted it first from $x$ to $x+\delta 
x$, then to $x+\delta x+ dx$. Thus we can shift the vector to a distant point 
and the result we get is independent of the path to the distant point. 
Therefore, if we shift the original vector $A_{\mu}$ at $x$ to all points by 
parallel displacement, we get a vector field that satisfies $A_{\mu:\nu}=0$, or
\begin{equation}
 \label{ecuacion 12.1}
 A_{\mu,\nu} = \Gamma^{\alpha}_{\mu\nu}.
\end{equation}

Can such a vector field be the graident of a scalar? Let us put $A_{\mu} = 
S_{\mu}$ in (\ref{ecuacion 12.1}). We get 
\begin{equation}
 \ref{ecuacion 12.2}
 S_{,\mu\nu} = \Gamma^{\alpha}_{\mu\nu} S_{\alpha}
\end{equation}
By virtue of the symmetry of $\Gamma^{\alpha}_{\mu\nu}$ in the lower suffixes, 
we have the same value for $S_{,\mu\nu}$ as $S_{,\nu\mu}$ and the equations 
(\ref{ecuacion 12.2}) are integrable.

Let us take four independent scalars satisfying (\ref{ecuacion 12.2}) and let 
them to be the coordinates $x^{\alpha'}$ of a new system of coordinates.
