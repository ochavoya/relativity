Let us contract two of the suffixes in $R_{\mu\nu\rho\sigma}$. If we take two with respect to which it is 
antisymmetrical, we get zero, of course. If we take any other two we get the same result, apart from the sign because 
of the symmetries (\ref{ecuacion 11.4}), (\ref{ecuacion 11.7}), and (\ref{ecuacion 11.8}). Let us take the first and 
last and put
\[
R^{\alpha}_{\nu\rho\alpha} = R_{\nu\rho}.
\]
It is called the Ricci tensor.

By multiplying (\ref{ecuacion 11.8}) by $g^{\mu\nu}$ we get 
\begin{equation}
 \label{ecuacion 14.1}
 R_{\nu\rho} = R_{\rho\nu}.
\end{equation}
The Ricci tensor is symmetrical 

