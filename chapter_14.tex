Let us contract two of the suffixes in $R_{\mu\nu\rho\sigma}$. If we take two with respect to which it is 
antisymmetrical, we get zero, of course. If we take any other two we get the same result, apart from the sign because 
of the symmetries (\ref{ecuacion 11.4}), (\ref{ecuacion 11.7}), and (\ref{ecuacion 11.8}). Let us take the first and 
last and put
\[
R^{\alpha}_{\nu\rho\alpha} = R_{\nu\rho}.
\]
It is called the Ricci tensor.

By multiplying (\ref{ecuacion 11.8}) by $g^{\mu\nu}$ we get 
\begin{equation}
 \label{ecuacion 14.1}
 R_{\nu\rho} = R_{\rho\nu}.
\end{equation}
The Ricci tensor is symmetrical.

We may contract again and form
\[
g^{\nu\rho}R_{\nu\rho} = R^{\nu}_{\nu} = R,
\]
say. This $R$ is a scalar and is called the scalar curvature or total curvature. It is defined in such way that it is 
positive for the surface of a sphere in three dimensions, as one can check by a straightforward calculation.

The Bianci relation (\ref{ecuacion 13.4}) involves five suffixes. Let us contract it twice and get a relation with one 
nondummy suffix. Put $\tau = \alpha$ and multiply by $g^{\mu\rho}$. The result is
\[
g^{\mu\rho}\left(
R^{\alpha}_{\mu\rho\sigma:\alpha} +
R^{\alpha}_{\mu\sigma\alpha:\rho} +
R^{\alpha}_{\mu\alpha\rho:\sigma}
\right) = 0
\]
or
\begin{equation}
 \label{ecuacion 14.2}
 \left( g^{\mu\rho} R^{\alpha}_{\mu\rho\sigma}\right)_{:\alpha} +
 \left( g^{\mu\rho} R^{\alpha}_{\mu\sigma\alpha}\right)_{:\rho} +
 \left( g^{\mu\rho} R^{\alpha}_{\mu\alpha\rho}\right)_{:\sigma}
 = 0 .
\end{equation}
Now
\[
 g^{\mu\rho} R^{\alpha}_{\mu\rho\sigma} =
  g^{\mu\rho}g^{\alpha\beta}R_{\beta\mu\rho\sigma} =
  g^{\mu\rho}g^{\alpha\beta}R_{\mu\beta\sigma\rho} =
  g^{\alpha\beta}R_{\beta\sigma} = R^{\alpha}_{\sigma}.
\]
One can write $R^{\alpha}_{\sigma}$ with the suffixes one over the other on account of $R_{\alpha\sigma}$ being 
symmetrical. Equation (\ref{ecuacion 14.2}) now becomes
\[
 R^{\alpha}_{\sigma:\alpha} + \left( g^{\mu\rho} R_{\mu\sigma} \right)_{:\rho} - R_{:\sigma} = 0
\]
or
\[
 2 R^{\alpha}_{\sigma:\alpha} - R_{:\sigma} = 0
\]
which is the Bianci relation for the Ricci tensor. If we raise the suffix $\sigma$, we get
\begin{equation}
 \label{ecuacion 14.3}
 \left( R^{\sigma \alpha} - \frac{1}{2} g^{\sigma \alpha} R\right)_{:\alpha} = 0
\end{equation}

The explicit expression for the Ricci tensor is, from (\ref{ecuacion 11.3})
\begin{equation}
 \label{ecuacion 14.4}
 R_{\mu\nu} = \Gamma^{\alpha}_{\mu\alpha,\nu} - \Gamma^{\alpha}_{\mu\nu,\alpha}
 + \Gamma^{\alpha}_{\mu\beta}\Gamma^{\beta}_{\nu\alpha} - \Gamma^{\alpha}_{\mu\nu}\Gamma^{\beta}_{\alpha\beta}
\end{equation}
The first term here does not appear to be symmetrical in $\mu$ and $\nu$, although the other three terms evidently are. 
To establish that the first term really is symmetrical we need a little calculation.

To differentiate the determinant $g$ we must differentiate each element $g_{\lambda\mu}$ and then multiply for the 
cofactor $g g^{\lambda\mu}$. Thus
\begin{equation}
 \label{ecuacion 14.5}
 g_{,\nu} = g g^{\lambda\mu} g_{\lambda\mu,\nu}.
\end{equation}
Hence 
\begin{equation}
 \label{ecuacion 14.6}
 \begin{array}{rcl}
    \Gamma^{\mu}_{\lambda\mu} & = & g^{\lambda\mu}\Gamma_{\lambda\nu\mu} =
    \frac{1}{2}\left(
        g_{\lambda\nu,\mu} + g_{\lambda\mu,\nu} - g_{\mu\nu,\lambda}
    \right)\\
    & = & \frac{1}{2} g^{\lambda\mu} g_{\lambda\mu,\nu} = \frac{1}{2} g^{-1}g_{\nu}
    = \frac{1}{2} \left( log(g) \right)_{,\nu}.
 \end{array}
\end{equation}
This makes it evident that the first term of (\ref{ecuacion 14.4}) is symmetrical.


