Let us contract two of the suffixes in $R_{\mu\nu\rho\sigma}$. If we take two with respect to which it is 
antisymmetrical, we get zero, of course. If we take any other two we get the same result, apart from the sign because 
of the symmetries (\ref{ecuacion 11.4}), (\ref{ecuacion 11.7}), and (\ref{ecuacion 11.8}). Let us take the first and 
last and put
\[
R^{\alpha}_{\nu\rho\alpha} = R_{\nu\rho}.
\]
It is called the Ricci tensor.

By multiplying (\ref{ecuacion 11.8}) by $g^{\mu\nu}$ we get 
\begin{equation}
 \label{ecuacion 14.1}
 R_{\nu\rho} = R_{\rho\nu}.
\end{equation}
The Ricci tensor is symmetrical.

We may contract again and form
\[
g^{\nu\rho}R_{\nu\rho} = R^{\nu}_{\nu} = R,
\]
say. This $R$ is a scalar and is called the scalar curvature or total curvature. It is defined in such way that it is 
positive for the surface of a sphere in three dimensions, as one can check by a straightforward calculation.

The Bianci relation (\ref{ecuacion 13.4}) involves five suffixes. Let us contract it twice and get a relation with one 
nondummy suffix. Put $\tau = \alpha$ and multiply by $g^{\mu\rho}$. The result is
\[
g^{\mu\rho}\left(
R^{\alpha}_{\mu\rho\sigma:\alpha} +
R^{\alpha}_{\mu\sigma\alpha:\rho} +
R^{\alpha}_{\mu\alpha\rho:\sigma}
\right) = 0
\]
or
\begin{equation}
 \label{ecuacion 14.2}
 \left( g^{\mu\rho} R^{\alpha}_{\mu\rho\sigma}\right)_{:\alpha} +
 \left( g^{\mu\rho} R^{\alpha}_{\mu\sigma\alpha}\right)_{:\rho} +
 \left( g^{\mu\rho} R^{\alpha}_{\mu\alpha\rho}\right)_{:\sigma}
 = 0 .
\end{equation}

