Up to the present our work has all been pure mathematics (apart from the physical assumption that the track of a 
particle is a geodesic). It was done mainly in the last century and applies to curved spaces in any number of 
dimensions. The only place where the number of dimensions would appear in the formalism is in the equation 
\[g^{\mu}_{\nu} = \mbox{ number of dimensions}\]

    Einstein made the assumption that in empty spaces
\begin{equation}
    \label{ecuacion 15.1}
    R_{\mu\nu} = 0.
\end{equation}
It constitutes his law of gravitation. ``Empty'' here means that there is no matter present and no physical fields 
except the gravitational field. The gravitational field does not disturb the emptyness. Other fields do.(\footnote{If 
the gravitational field does not disturb the \emph{emptyness} of space, then the gravitational field is nothing, which 
begs the questions: ``What is a quantum of \emph{nothing}>'', ``How \emph{nothing} can be the substrate of 
gravitational waves?'', etcetera}). The conditions for empty space hold in a good approximation for the space between 
the planets in the solar system, and equation (15.1) applies there. 

Flat space obviously satisfies (\ref{ecuacion 15.1}). The geodesics are then straight lines and so particles move along 
straight lines. Where space is not flat, Einstein's law puts restrictions on the curvature. Combined with the 
assumption that the planets move along geodesics, it gives some information about their motion.

At first sight, Einstein's law of gravitation does not look anything like Newton's. To see a similarity, we must look 
on the $g_{\mu\nu}$ as \emph{potentials} describing the gravitational field. There are ten of them, instead of just the 
one potential of the Newtonian theory. They describe not only the gravitational field, but also the system of 
coordinates. The gravitational field and the system of coordinates are inextricable mixed up in Einstein's theory, and 
one cannot describe one without the other.

Looking upon the $g_{\mu\nu}$ as potentials, we find that (\ref{ecuacion 15.1}) appears as field equations. They are 
like the usual equations of physics in that they are of the second order, because second order derivatives appear in 
(\ref{ecuacion 14.4}), as the Christoffel symbols involve first derivatives. They are unlike the usual field equations 
in that they are not linear; far from it. The nonlinearity means that the equations are complicated and it is difficult 
to get accurate solutions.



