Up to the present our workhas all been pure mathematics (apart from the physical assumption that the track of a 
particle is a geodesic). It was done mainly in the last entury and applies to curved spaces in any number of 
dimensions. The only place where the number of dimensions would appear in the formalism is in the equation 
\[g^{\mu}_{\nu} = \mbox{ number of dimensions}\]

    Einstein made the assumption that in empty spaces
\begin{equation}
    \label{ecuacion 15.1}
    R_{\mu\nu} = 0.
\end{equation}
It constitutes his law of gravitation. ``Empty'' here means that htere is no matter present and no physical fields 
except the gravitational field. The gravitational field does not disturb the emptyness. Other fields do.(\footnote{If 
the gravitational field does not disturb the \emph{emptyness} of space, then the gravitational field is nothing, which 
begs the questions: ``What is a quantum of \emph{nothing}>'', ``How \emph{nothing} can be the substrate of 
gravitational waves?'', etcetera}). The conditions for empty space hold in a good approximation for the space between 
ghe planets in the solar system, and equation (15.1) applies there. 

Flat space obviously satisfies (\ref{ecuacion 15.1}). The geodesics are then straight lines and so particles move along 
straight lines. 

