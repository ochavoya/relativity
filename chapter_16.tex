Let us consider a static gravitational field and refer it to a static coordinate system. The $g_{\mu\nu}$ are then 
constant in time, $g_{\mu\nu,0} = 0$. Further, we must have 
\[
 g_{m0} = 0, \, \left( m=1,\cdots,3 \right) .
\]
This leads to
\[
 g^{m0} = 0, \, g^{00} = (g_{00})^{-1},
\]
and $g^{mn}$ is the reciprocal matrix to $g_{mn}$. Roman suffixes like $m$ and $n$ always take on the values 1, 2, 3. 
We find that $\Gamma_{m0n} = 0$ and hence also $\Gamma^{m}_{0n}=0$.

Let us take a particle that is moving slowly, compared with the velocity of light. Then $v^m$ is a small quantity, of 
the first order. With neglect of second order quantities,
\begin{equation}
 \label{ecuacion 16.1}
 g_{00} {v^{0}}^2 = 1.
\end{equation}
