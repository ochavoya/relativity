Let us consider a static gravitational field and refer it to a static coordinate system. The $g_{\mu\nu}$ are then 
constant in time, $g_{\mu\nu,0} = 0$. Further, we must have 
\[
 g_{m0} = 0, \, \left( m=1,\cdots,3 \right) .
\]
This leads to
\[
 g^{m0} = 0, \, g^{00} = (g_{00})^{-1},
\]
and $g^{mn}$ is the reciprocal matrix to $g_{mn}$. Roman suffixes like $m$ and $n$ always take on the values 1, 2, 3. 
We find that $\Gamma_{m0n} = 0$ and hence also $\Gamma^{m}_{0n}=0$.

Let us take a particle that is moving slowly, compared with the velocity of light. Then $v^m$ is a small quantity, of 
the first order(\footnote{compared to $v^0$.}). With neglect of second order quantities,
\begin{equation}
 \label{ecuacion 16.1}
 g_{00} {v^{0}}^2 = 1.
\end{equation}
The particle will move along a geodesic. With neglect of second-order quantities, the equation (\ref{ecuacion 8.3}) 
gives
\[
\begin{array}{rcl}
 \frac{dv^m}{ds} & = & -\Gamma^m_{00}{v^0}^2 = -g^{mn} \Gamma_{n00}{v^0}^2\\
                 & = & = \frac{1}{2} g^{mn}g_{00,n} {v^0}^2. 
\end{array}
\]
Now
\[
 \frac{dv^m}{ds} = \frac{dv^m}{dx^0} \frac{dx^0}{ds} = frac{dv^m}{dx^0}v^0
\]
to the first order. Thus
\begin{equation}
 \label{ecuacion 16.2}
 \frac{dv^m}{dx^0} = \frac{1}{2} g^{mn}g_{00,n}v^0 = g^{mn}\left( {g_{00}}^{1/2} \right)_{,n}
\end{equation}
with the help of (\ref{ecuacion 16.1}). Since the $g_{\mu\nu}$ are independent of $x^0$, we may lower the suffix $m$ 
here and get
\begin{equation}
 \label{ecuacion 16.3}
 \frac{dv^m}{dx^0} = \left( {g_{00}}^{1/2} \right)_{,m} .
\end{equation}

We see that the particle moves as thought it were under the influence of a potential $\left({g_{00}}^{1/2}\right)$. We 
have not used Einstein's law to obtain this result. We now use Einstein's law to obtain a condition for the potential, 
so that it can be compared with Newton's.

Let us suppose that the gravitational field is weak, so that the curvature of space is small. Then we may choose our 
coordinate system so that the curvature of the coordinate lines (each with three $x$'s constant) is small. Under those 
conditions the $g_{\mu\nu}$ are approximately constant, and $g_{\mu\nu,\sigma}$ and all the Christoffel symbols are 
small. If we count them of the first order and neglect second-order quantities, Einstein's law (\ref{ecuacion 15.1}) 
becomes, from (\ref{ecuacion 14.4})
\[
 \Gamma^{\alpha}_{\mu\alpha,\nu} - \Gamma^{\alpha}_{\mu\nu,\alpha} = 0.
\]
We can evaluate this most conveniently by contracting (\ref{ecuacion 11.6}) with $\rho$ and $\mu$ interchanged and 
neglecting second-order terms. The result is
\begin{equation}
 \label{ecuacion 16.4}
 g^{\rho\sigma} \left( 
      g_{\rho\sigma,\mu\nu} 
    - g_{\nu\sigma,\mu\rho} 
    - g_{\mu\rho,\nu\sigma} 
    + g_{\mu\nu,\rho\sigma}  \right) = 0.
\end{equation}


