Let us take again a static gravitational field and consider an atom at rest emitting monochromatic radiation. The 
wavelength of the light will correspond to a definite $\Delta s$. Since the atom is at rest we have, for a static 
system of coordinates such as we used in Section 16,
\[
    {\Delta s}^2 = g_{00}{\Delta x^0}^2,
\]
where $\Delta x^0$ is the period, that is, the time between successive crests referred to our static coordinate system.

If the light travels to another place, $Delta x^0$ will remain constant. This $Delta x^0$ will not be the same as the 
period of the same spectral line emitted by a local atom, which would be $\Delta s$ again. The period is thus dependent 
on the gravitational potential $g_{00}$ at the place where the light was emitted:
\[
    \Delta x^0 :: {g_{00}}^{-1/2}.
\]
The spectral line will be shifted by this factor ${g_{00}}^{-1/2}$

If we use the Newtonian approximation (\ref{ecuacion 16.6}), we have 
\[
    \Delta x^0 :: 1 - V.
\]

$V$ will be negative at a place with a strong gravitational field, such as the surface of the sun, so, light emitted 
there will be red-shifted when compared with the corresponding light emitted on earth. The effect can be observed with 
the sun's light but it is rather masked by other physical effects, such as the Doppler effect arising from the motion 
of the emitting atoms. It can be better observed in light emitted from a white dwarf star, where the high density of 
the matter in the star gives rise to a much stronger gravitational potential at its surface.(\footnote{At the surface 
of a white dwarf star, precisely because of the high density of the stellar matter, the \emph{small curvature} 
approximation---and the concept of gravitational potential energy---are not applicable.}) 
