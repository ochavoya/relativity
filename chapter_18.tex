The Einstein equations for empty space are nonlinear and are therefore very complicated, and it is difficult to get 
accurate solutions of them. There is, however, a special case which can be solved without too much trouble; namely the 
static spherically symmetric field produced by a spherically symmetric body at rest.

The static condition means that, with a static coordinate system, the $g_{\mu\nu}$ are independent of the time $x^0$ or 
$t$ and also $g_{0m}$(\footnote{Making the assumption that all intervals of the form $(\delta x^0,0,0,0)$ are temporal 
and all interval of the form $(0,\delta x^1, \delta x^2, \delta x^3)$ are spatial.}). The spatial coordinates may be 
taken to be spherical polar coordinates $x^1 = r,\,x^2=\theta,\,x^3=\phi$. The most general form of $ds^2$ compatible 
with spherical symmetry is
\[
 ds^2 = U dt^2 - V dr^2 -W r^2(d\theta^2 + \sin^2\theta d\phi^2),
\]
where $U$, $V$, and $W$ are functions of $r$ only. We may replace $r$ by any function of $r$ without disturbing the 
spherical symmetry. We use this freedom to simplify things as much as possible, and the most convenient arrangement is 
to have $W=1$. The expression for $ds^2$ may then be written as 
\begin{equation}
 \label{ecuacion 18.1}
 ds^2 = e^{2\nu} dt^2 - e^{2\lambda} dr^2 - r^2 d\theta^2 - r^2 \sin^2 \theta d\phi^2,
\end{equation}
with $\nu$ and $\lambda$ functions of $r$ only. They must be chosen to satisfy the Einstein equations. We can read off 
the values of the $g_{\mu\nu}$ from (\ref{ecuacion 18.1}), namely,
\[
\begin{array}{cccc}
 g_00 = e^{2\nu}, & g_{11} = e^{-2\lambda}, & g_{22} = -r^2, & g_{33} = - r^2 \sin^2\theta,
\end{array}
\]
and 
\[
g_{\mu\nu} = 0 \mbox{ for } \mu \ne \nu .
\]
We find
\[
   \begin{array}{cccc}
   g^{00} = e^{-2\nu}, & g_{11} = e^{-2\lambda}, & g^{22} = -r^2, &  g^{33} = -r^{-2} \sin^{-2} 
\theta,
   \end{array}
\]
and
\[
 g^{\mu\nu} = 0 \mbox{ for } \mu \ne \nu .
\]

It is now necessary to calculate all the Christoffel symbols $\Gamma^\sigma_{\mu\nu}$. Many of them vanish. The ones 
that do not are(\footnote{With primes denoting differentiation with respect to $r$.}):
\[
 \begin{array}{ll}
   \Gamma^1_{00} = \nu' e^{2\nu - 2 \lambda}       & \Gamma^0_{10} =  \nu' \\
   \Gamma^1_{11} = \lambda'                        & \Gamma^2_{12} = \Gamma^3_{13} = r^{-1}\\
   \Gamma^1_{22} = - r e^{-2\lambda}               & \Gamma^2_{23} = \cot \theta \\
   \Gamma^1_{33} = - r \sin^2 \theta e^{-2\lambda} & \Gamma^2_{33} = - \sin \theta \cos \theta
 \end{array}
\]
These expressions are to be substituted in (\ref{ecuacion 14.4}). The results are
\begin{equation}
 \label{ecuacion 18.2}
 R_{00} = \left( - \nu'' + \lambda' \nu' - {\nu'}^2 - \frac{2 \nu'}{r} \right) e^{2 \nu - 2 \lambda},
\end{equation}
\begin{equation}
 \label{ecuacion 18.3}
 R_{11} = \nu'' - \lambda' \nu' + {\nu'}^2 - \frac{2 \lambda'}{r},
\end{equation}
\begin{equation}
 \label{ecuacion 18.4}
 R_{22} = (1 + r \nu' - r \lambda' ) e^{-2\lambda} -1,
\end{equation}
\[
 R_{33} = R_{22} \sin^2 \theta,
\]
with the other components of $R_{\mu\nu}$ vanishing.

Einstein's law of gravitation requires these expressions to vanish. The vanishing of (\ref{ecuacion 18.2}) and 
(\ref{ecuacion 18.3}) leads to
\[
 \lambda' + \nu' = 0. 
\]
For large values of $r$ the space must approximate to being flat, so that $\lambda$ and $\nu$ both tend to zero as 
$r\rightarrow \infty$. It follows that
\[
 \lambda + \nu = 0.
\]
The vanishing of (\ref{ecuacion 18.4}) now gives
\[
 (1 + 2 r \nu') e^{2\nu} = 1
\]
or
\[
 \left( r e^{2\nu}\right)' = 1.
\]
Thus 
\[
 r e^{2\nu} = r - 2m,
\]
where $m$ is a constant of integration. This also makes (\ref{ecuacion 18.2}) and (\ref{ecuacion 18.3}) vanish. We now 
get
\begin{equation}
 \label{ecuacion 18.5}
 g_{00} = 1 - \frac{2 m}{r}
\end{equation}

The Newtonian approximation must hold for large values of $r$. Comparing (\ref{ecuacion 18.5}) with \ref{ecuacion 
16.6}), we see that the constant of integration that has appeared in (\ref{ecuacion 18.5}) is just the mass of the 
central body which is producing the gravitational field.

Te complete solution is
\begin{equation}
\label{ecuacion 18.6}
 ds^2 = \left( 1 - \frac{2 m }{r} \right)
 -  \left( 1 - \frac{2 m }{r} \right)^{-1} dr^2 - r^2 d\theta^2 - r^2 d\theta^2 - r^2 \sin^2 \theta d\phi^2.
\end{equation}
It is known as the Schwarzschild solution. It holds outside the surface of the body that is producing the field, where 
there is no matter. Thus it holds fairly accurately outside the surface of a star.

The solution (\ref{ecuacion 18.6}) leads to small corrections in the Newtonian theory for the motions of the planets 
around the Sun. These corrections are appreciable only in the case of Mercury, the nearest planet, and they explain the 
discrepancy of the motion of this planet with the Newtonian theory. Thus they provide a striking confirmation of the 
Einstein theory.\footnote{As we know that the mass of the Sun is not constant and this fact must have its effect on the 
displacement of the perihelion of the orbit of Mercury. I have attempted to compute this but the methods of classical 
perturbation theory cannot be applied because, when we consider a non constant solar mass, the resulting dynamical 
system stops being autonomous. Also, the displacement of the orbit of Mercury is small $39''$ per century. } 






