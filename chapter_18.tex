The Einstein equations for empty space are nonlinear and are therefore very complicated, and it is difficult to get 
accurate solutions of them. There is, however, a special case which can be solved without too much trouble; namely the 
static spherically symmetric field produced by a spherically symmetric body at rest.

The static condition means that, with a static coordinate system, the $g_{\mu\nu}$ are independent of the time $x^0$ or 
$t$ and also $g_{0m}$(\footnote{Making the assumption that all intervals of the form $(\delta x^0,0,0,0)$ are temporal 
and all interval of the form $(0,\delta x^1, \delta x^2, \delta x^3)$ are spatial.}). The spatial coordinates may be 
taken to be spherical polar coordinates $x^1 = r,\,x^2=\theta,\,x^3=\phi$. The most general form of $ds^2$ compatible 
with spherical symmetry is
\[
 ds^2 = U dt^2 - V dr^2 -W r^2(d\theta^2 + \sin^2\theta d\phi^2),
\]
where $U$, $V$, and $W$ are functions of $r$ only. We may replace $r$ by any function of $r$ without disturbing the 
spherical symmetry. We use this freedom to simplify things as much as possible, and the most convenient arrangement is 
to have $W=1$. The expression for $ds^2$ may then be written as 
\begin{equation}
 \label{ecuacion 18.1}
 ds^2 = e^{2\nu} dt^2 - e^{2\lambda} dr^2 - r^2 d\theta^2 - r^2 \sin^2 \theta d\phi^2,
\end{equation}
with $\nu$ and $\lambda$ functions of $r$ only. They must be chosen to satisfy the Einstein equations. We can read off 
the values of the $g_{\mu\nu}$ from (\ref{ecuacion 18.1}), namely,
\[
\begin{array}{cccc}
 g_00 = e^{2\nu}, & g_{11} = e^{-2\lambda}, & g_{22} = -r^2, & g_{33} = - r^2 \sin^2\theta,
\end{array}
\]
and 
\[
g_{\mu\nu} = 0 \mbox{ for } \mu \ne \nu .
\]
