The solution (\ref{ecuacion 18.6}) becomes singular at $r = 2 m$, because then $g_{00}=0$ and $g_{11}=-\infty$. It 
would seem that $r=2m$ gives a minimum radius for a body of mass $m$. But a close investigation shows that this is not 
so.

Consider a particle falling into the central body and let its velocity vector be $v^\mu = d z^\mu / ds$. Let us suppose 
that it falls in radially, so that $v^2=v^3=0$. The motion is determined by the geodesic equation (\ref{ecuacion 8.3}):
\[
 \begin{array}{rcl}
  \frac{dv^0}{ds} & = & - \Gamma^0_{\mu\mu} v^\mu v^\nu = -g^{00} \Gamma_{0\mu\nu} v^\mu v^\nu\\
         & = & -g^{00} g_{00,1} v^0 v^1 = - g^{00} \frac{dg_{00}}{ds} v^0 .
 \end{array}
\]

Now $g^{00}= \frac{1}{g_{00}}$, so we get
\[
  g_{00} \frac{dv^0}{ds} + \frac{dg_{00}}{ds} v^0 = 0.
\]
This integrates to
\[
 g_{00} v^0 = k,
\]
with $k$ a constant. It is the value of $g_{00}$ where the particle starts to fall.

Again, we have
\[
 1 = g_{\mu\nu} v^\mu v^\nu = g_{00} {v^0}^2 + g_{11} {v^1}^2 .
\]
Multiplying this equation by $g_{00}$ and using $g_{00} g_{11} = -1$, which we obtained in the last section, we find
\[
 k^2 - {v^1}^2 = g_{00} = 1 - \frac{2 m }{r}.
\]
For a falling body $v^1<0$ and hence
\[
 v^1 = - \left( k^2 - 1 + \frac{2m}{r} \right)^{\frac{1}{2}}.
\]

Now
\[
 \frac{dt}{dr} = \frac{v^0}{v^1} = - k \left( 1 - \frac{2 m}{r} \right)^{-1}
 \left( k^2 - 1 + \frac{2m}{r} \right)^{-\frac{1}{2}}.
\]
Let us suppose the particle is close to the critical radius, so $r = 2 m + \epsilon$ with $\epsilon$ small, and let us 
neglect $\epsilon^2$. Then 
\[
 \frac{dt}{dr} \approx - \frac{2 m }{\epsilon} = - \frac{2 m}{r - 2m}.
\]
This integrates to
\[
 t = - 2 m \log (r - 2 m) + \mbox{ constant}.
\]
Thus, as $r\rightarrow 2m$ $t\rightarrow \infty$. The particle takes an infinite time to reach the critical radius $r = 
2 m$.

Let us suppose the particle is emitting light of a certain spectral line, and is being observed by someone at a large 
value of $r$. The light is red-shifted by a factor of ${g_{00}}^{-1/2}$. This factor becomes infinite as the particle 
approaches the critical radius. All physical processes on the particle will be observed to be going more and more 
slowly as it approaches $r=2m$.

Now consider an observer traveling with the particle. His time scale is measured by $ds$. Now 
\[
 \frac{ds}{dr} = \frac{1}{v^1} = - \left( k^2 - 1 + \frac{2 m }{r} \right) ^{-1/2},
\]
and this tends to $-k^{-1}$ as $r$ tends to $2m$. Thus, the particle reaches $r = 2m$ after a lapse of finite proper 
time for the observer. The traveling observer has aged only a finite amount when he reaches $r = 2m$. What will happen 
to him afterwards? He may continue sailing through empty space into smaller values of $r$.

To examine the continuation of the Schwarzschild solution for values of $r < 2 m$, it is necessary to use a non static 
system of coordinates, so that we have  the $g_{\mu\nu}$ varying with the time coordinate. We keep the coordinates 
$\theta$ and $\phi$ unchanged, but instead of $t$ and $r$ we use $\tau$ and $\rho$, defined by
\begin{equation}
 \label{ecuacion 19.1}
 \tau = t + f(r)\, \rho = t + g(r),
\end{equation}
where the functions $f$ and $g$ are at our disposal.

We have, using the prime again to denote derivative with respect to $r$,
\begin{equation}
 \label{ecuacion 19.2}
 \begin{array}{rcl}
   d\tau^2 - \frac{2 m}{3} d\rho^2 & = & (d t + f' dr)^2 - \frac{2 m}{r}(dt + g' dr)^2 \\
   & = & \left( 1 - \frac{2m}{r} \right)dt^2 + 2 \left( f' - \frac{2m}{r} g' \right) dt dr   
   + \left( f'^2 - \frac{2m}{r} g'^2 \right)dr^2\\
   & = & \left(1 - \frac{2m}{r}\right)dt^2 - \left(1 - \frac{2m}{r}\right)^{-1} dr^2
 \end{array}
\end{equation}
provided we choose the functions $f$ and $g$ to satisfy
\begin{equation}
 \label{ecuacion 19.3}
 f' = \frac{2 m}{r} g'
\end{equation}
and
\begin{equation}
 \label{ecuacion 19.4}
 \frac{2m}{r}g'^2 - f'^2 = \left(1 - \frac{2m}{r}\right)^{-1}.
\end{equation}
Elimination of $f$ from these equations gives
\begin{equation}
 \label{ecuacion 19.5}
 g' = \left( \frac{r}{2 m} \right)^{1/2} \left(
     1 = \frac{2 m}{r}
 \right)^{-1}
\end{equation}
To integrate this equation, put $r = y^2$ and $2 m = a^2$. With $r > 2m$ we have $y > a$. We now have
\[
  \frac{dg}{dy} = 2 y \frac{dg}{dr} = \frac{2 y^4}{a} \frac{1}{y^2 - a^2},
\]
\begin{equation}
 \label{ecuacion 19.6}
 g = \frac{2}{3a} y^3 + 2 a y - a^2 \log \frac{y + 1}{y - a}.
\end{equation}
Finally we get from (\ref{ecuacion 19.3}) and (\ref{ecuacion 19.5})
\[
g'-f' = \left( 1 - \frac{2 m }{r} \right) g' = \left(\frac{r}{2 m}\right)^{1/2},
\]
which integrates to 
\begin{equation}
 \label{ecuacion 19.7}
 \frac{2}{3} \frac{1}{\sqrt{2m}} r^{3/2} = g - f = \rho - \tau .
\end{equation}
Thus
\begin{equation}
 \label{ecuacion 19.8}
 r = \mu (\rho - \tau)^{2/3},
\end{equation}
with 
\[
\mu = \left(\frac{3}{2} \sqrt{2 m}\right)^{2/3}.
\]

In this way we see that we can satisfy the conditions (\ref{ecuacion 19.3}) and (\ref{ecuacion 19.4}) and so we can use 
(\ref{ecuacion 19.2}). Substituting into the Schwarzschild solution (\ref{ecuacion 18.6}), we get
\begin{equation}
 \label{ecuacion 19.9}
 ds^2 = d\tau^2 - \frac{2 m}{\mu (\rho - \tau)^{2/3}} d\rho^2 - \mu^2 (\rho - \tau)^{4/3}
 \left( d\theta^2 + \sin^2 \theta d\phi^2 \right).
\end{equation}
The critical value $r = 2 m$ corresponds, from (\ref{ecuacion 19.7}), to $\rho - \tau = 4 m/3$. There is no singularity 
here in the metric (\ref{ecuacion 19.9}).

We know that the metric (\ref{ecuacion 19.9}) satisfies the Einstein equations for empty space in the region $r > 2m$, 
because it can be transformed to the Schwarzschild solution by a mere change of coordinates. We can  infer that it 
satisfies the Einstein equations for $r \le 2m $ from analytic continuity, because it does not involve any singularity 
at $r=2m$. It may continue to hold down to $r=0$ or $\rho - \tau = 0$. The singularity appears in the connection 
between the new coordinates and the original ones, equation (\ref{ecuacion 19.1}). But once we have established the 
coordinate system we can disregard the previous one and the singularity no longer appears.

We see that the Schwarzschild solution for empty space can be extended to the region $r<2m$. But this region cannot 
communicate with the space for which $r>m$. Any signal, even a light signal, will take an infinite time to reach the 
boundary $r=2m$, as we can easily check. Thus we cannot have direct observational knowledge of the region $r < 2m$. 
Such a region is called a black hole, because things can fall into it (taking and infinite time by our clocks, to do 
so) but nothing can come out.

The question arises whether such a region can actually exist. All we can say is that definitely the Einstein equations 
allow it. A massive stellar object can collapse to a very small radius and the gravitational forces then become so 
strong that no physical forces can hold them in check and prevent further collapse. It would seem that it would have to 
collapse into a black hole. It would take an infinite time to do so by our clocks, but only a finite time relative to 
the collapsing matter itself.






