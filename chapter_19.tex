The solution (\ref{ecuacion 18.6}) becomes singular at $r = 2 m$, because then $g_{00}=0$ and $g_{11}=-\infty$. It 
would seem that $r=2m$ gives a minimum radius for a body of mass $m$. But a close investigation shows that this is not 
so.

Consider a particle falling into the central body and let its velocity vector be $v^\mu = d z^\mu / ds$. Let us suppose 
that it falls in radially, so that $v^2=v^3=0$. The motion is determined by the geodesic equation (\ref{ecuacion 8.3}):
\[
 \begin{array}{rcl}
  \frac{dv^0}{ds} & = & - \Gamma^0_{\mu\mu} v^\mu v^\nu = -g^{00} \Gamma_{0\mu\nu} v^\mu v^\nu\\
         & = & -g^{00} g_{00,1} v^0 v^1 = - g^{00} \frac{dg_{00}}{ds} v^0 .
 \end{array}
\]

Now $g^{00}= \frac{1}{g_{00}}$, so we get
\[
  g_{00} \frac{dv^0}{ds} + \frac{dg_{00}}{ds} v^0 = 0.
\]
This integrates to
\[
 g_{00} v^0 = k,
\]
with $k$ a constant. It is the value of $g_{00}$ where the particle starts to fall.




