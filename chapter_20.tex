With a transformation of coordinates, an element of four-dimensional volume transforms according to the law
\begin{equation}
 \label{ecuacion 20.1}
 {dx^0}'{dx^1}'{dx^2}'{dx^3}' = {dx^0}{dx^1}{dx^2}{dx^3} J,
\end{equation}
where $J$ is the Jacobian
\[
 J = \frac{\partial({x^0}',{x^1}',{x^2}',{x^3}')}{\partial({x^0},{x^1},{x^2},{x^3})}
 = \mbox{ determinant of } x^{\mu'}_{,\nu}
\]
We may write (20.1)
\begin{equation}
 \label{ecuacion 20.2}
 d^4 x' = J d^4 x
\end{equation}
for brevity.

Now
\[
   g_{\alpha\beta} = x^{\mu'}_{,\alpha} g_{\mu'\nu'} x^{\nu'}_{,\beta}.
\]
We can look upon the right side as the product of three matrices, the first matrix having its rows specified by 
$\alpha$ and columns specified by $\mu'$, the second having its rows specified by $\mu'$ and columns by $\nu'$ and the 
third having its rows specified by $\nu'$ and columns by $\beta$. This product equals the matrix $g_{\alpha\beta}$ on 
the left. The corresponding equation must hold between the determinants; therefore
\[
 g = J g' J
\]
or
\[
 g = J^2 g'.
\]

Now, $g$ is a negative quantity, so we may form $\sqrt{-g}$, taking the positive value for he square root. Thus
\begin{equation}
 \label{ecuacion 20.3}
 \sqrt{-g} = J \sqrt{- g'}.
\end{equation}

Suppose $S$ is a scalar field quantity, $S = S'$. Then
\[
\int{S \sqrt{-g} d^4 x} = \int{S \sqrt{-g'} J d^4 x} = \int{S' \sqrt{-g'} d^4 x'},
\]
if the region of integration for the $x'$ corresponds to that for the $x$. Thus
\begin{equation}
 \label{ecuacion 20.4}
 \int{S \sqrt{-g} d^4 x} = \mbox{ invariant.}
\end{equation}
We call $S \sqrt{-g}$ a scalar density, meaning a quantity whose integral is invariant.

Similarly, for any tensor field $T^{\mu\nu\cdots}$ we may call $T^{\mu\nu\cdots}\sqrt{-g}$ a tensor density. The 
integral
\[
   \int{T^{\mu\nu\cdots}\sqrt{-g} d^4 x}
\]
is a tensor if the domain of integration is small. It is not a tensor if the domain of integration is not small, 
because it then consists of a sum of tensors located at different points and it does not transform in any simple way in 
a transformation of coordinates.

The quantity $\sqrt{-g}$ will be very much used in the future. For brevity we shall write it simply as $\sqrt{}$. We 
have (\footnote{This follows from the identity $\log \sqrt{-g } = \frac{1}{2} \log (-g) $.})
\[
 g^{-1} g_{,\nu} = 2 {\sqrt{}}^{-1} {\sqrt{}}_{,\nu}, 
\]
Thus the formula (\ref{ecuacion 14.5}) gives 
\begin{equation}
 \label{ecuacion 20.5}
 \sqrt{}_{,\nu} = \frac{1}{2} \sqrt{} g^{\lambda\mu}g_{\lambda\mu,\nu}
\end{equation}
and the formula (\ref{ecuacion 14.6}) may be written
\begin{equation}
 \label{ecuacion 20.6}
 \Gamma^\mu_{\nu\mu}\sqrt{} = \sqrt{}_{,\nu}
\end{equation}
