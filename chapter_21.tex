The vector $A^\mu$ has the covariant divergence $A^\mu_{;\mu}$, which is a scalar. We have
\[
    {A^\mu}_{;\mu} = A^\mu_{,\mu} + \Gamma^\mu_{\nu\mu} A^\nu = A^\mu_{,\mu}+\sqrt{}^{-1}\sqrt{}_{,\nu}A^\nu.
\]
Thus
\begin{equation}
 \label{ecuacion 21.1}
 A^\mu_{;\mu} \sqrt{} = \left( A^\mu \sqrt{} \right)_{,\mu}.
\end{equation}

We can put $A^\mu_{;\mu}\sqrt{}$ for $S$ in (\ref{ecuacion 20.4}), and we get the invariant
\[
 \int {A^\mu_{;\mu}\sqrt{} d^4 x}
  = \int { \left(A^\mu \sqrt{}\right)_{,\mu} d^4 x}.
\]
If the integral is taken over a finite (four-dimensional) volume, the right hand side can be converted by Gauss's 
theorem to an integral over the boundary surface (three-dimensional) of that volume.

If $A^\mu_{;\mu}=0$, we have 
\begin{equation}
 \label{ecuacion 21.2}
 \left( A^\mu \sqrt{}\right)_{,\mu} = 0
\end{equation}
and this gives us a conservation law; namely, the conservation of a fluid whose density is $A^0\sqrt{}$ and whose flow 
is given by the three-dimensional vector $A^m\sqrt{}(m=1,2,3)$. We may integrate (\ref{ecuacion 21.2}) over a 
three-dimensional volume $V$ lying at a fixed time $x^0$(\footnote{I sense here an error of method because Dirac makes 
the assumption that $x^0$ is strictly temporal.}). The result is
\[
\begin{array}{rcl}
    \left(\int{A^0\sqrt{}d^3x}\right)_{,0} & = & - \int{\left(A^m\right)_{,m}d^3 x} \\
                               & = & \mbox{ surface integral over boundary of } V
\end{array} 
\]

