The vector $A^\mu$ has the covariant divergence $A^\mu_{;\mu}$, which is a scalar. We have
\[
    {A^\mu}_{;\mu} = A^\mu_{,\mu} + \Gamma^\mu_{\nu\mu} A^\nu = A^\mu_{,\mu}+\sqrt{}^{-1}\sqrt{}_{,\nu}A^\nu.
\]
Thus
\begin{equation}
 \label{ecuacion 21.1}
 A^\mu_{;\mu} \sqrt{} = \left( A^\mu \sqrt{} \right)_{,\mu}.
\end{equation}

We can put $A^\mu_{;\mu}\sqrt{}$ for $S$ in (\ref{ecuacion 20.4}), and we get the invariant
\[
 \int {A^\mu_{;\mu}\sqrt{} d^4 x}
  = \int { \left(A^\mu \sqrt{}\right)_{,\mu} d^4 x}.
\]
If the integral is taken over a finite (four-dimensional) volume, the right hand side can be converted by Gauss's 
theorem to an integral over the boundary surface (three-dimensional) of that volume.

If $A^\mu_{;\mu}=0$, we have 
\begin{equation}
 \label{ecuacion 21.2}
 \left( A^\mu \sqrt{}\right)_{,\mu} = 0
\end{equation}
and this gives us a conservation law; namely, the conservation of a fluid whose density is $A^0\sqrt{}$ and whose flow 
is given by the three-dimensional vector $A^m\sqrt{}(m=1,2,3)$. We may integrate (\ref{ecuacion 21.2}) over a 
three-dimensional volume $V$ lying at a fixed time $x^0$(\footnote{I sense here an error of method because Dirac makes 
the assumption that $x^0$ is strictly temporal.}). The result is
\[
\begin{array}{rcl}
    \left(\int{A^0\sqrt{}d^3x}\right)_{,0} & = & - \int{\left(A^m\right)_{,m}d^3 x} \\
                               & = & \mbox{ surface integral over boundary of } V
\end{array} 
\]
If there is no current in the boundary of $V$, $\int{A^0\sqrt{}d^3x}$ is constant. 

Those results for a vector $A^\mu$ cannot be taken over to a a tensor with more than on suffix, in general. Take a two 
suffix tensor $Y^{\mu\nu}$. In flat space we can use Gauss's theorem to express $\int{{Y^{\mu\nu}}_{,\nu}dx^4}$ as a 
surface integral, but in curved space we cannot in general express $\int{{Y^{\mu\nu}}_{;\nu}\sqrt{}dx^4}$ as a surface 
integral. An exception occurs for an antisymmetrical tensor $F^{\mu\nu} = -F^{\nu\mu}$.

In this case we have
\[
 {F^{\mu\nu}}_{;\sigma} = {F^{\mu\nu}}_{,\sigma} +
 \Gamma^{\mu}_{\sigma\rho} F^{\rho\nu} +
 \Gamma^{\nu}_{\sigma\rho} F^{\mu\rho}
\]
so
\[
 \begin{array}{rcl}
  {F^{\mu\nu}}_{;\nu} & = & {F^{\mu\nu}}_{,\nu} +
 \Gamma^{\mu}_{\nu\rho} F^{\rho\nu} +
 \Gamma^{\nu}_{\nu\rho} F^{\mu\rho} \\
 & = & {F^{\mu\nu}}_{,\nu} + \sqrt{}^{-1} \sqrt{}_{,\rho} F^{\mu\rho}
 \end{array}
\]
from (\ref{ecuacion 20.6}) (\footnote{And the fact that $\Gamma^{\mu}_{\nu\rho} F^{\rho\nu} \equiv 0$ because 
$\Gamma^{\mu}_{\nu\rho} = \Gamma^{\mu}_{\rho\nu}$.}). Thus 
\begin{equation}
 \label{ecuacion 21.3}
 {F^{\mu\nu}}_{;\nu} \sqrt{} = \left( {F^{\mu\nu} \sqrt{}} \right)_{,\nu} 
\end{equation}
Hence $\int{{F^{\mu\nu}}_{;\nu} \sqrt{}d^4x} = \mbox{ a surface integral}$, and if ${F^{\mu\nu}}_{;\nu} = 0$ we have a 
conservation law.

In the symmetrical case $Y^{\mu\nu} = Y^{\nu\mu}$ we can get a corresponding equation with an extra term, provided we 
put one of the suffixes downstairs and deal with ${{Y_\mu}^\nu}_{;\nu}$
