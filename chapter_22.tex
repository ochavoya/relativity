The d'Alembert equation for a scalar $V$, namely $\Box V = 0$, gives, from (\ref{ecuacion 10.9}),
\begin{equation}
 \label{ecuacion 22.1}
 g^{\mu\nu}\left(
    V_{\mu\nu} - \Gamma^\alpha_{\mu\nu} V_{,\alpha} = 0.
 \right)
\end{equation}
If we are using rectilinear axes in flat space, each of the four coordinates $x^\lambda$ satisfies $\Box x^\lambda = 
0$. We might substitute $x^\lambda$ for $V$ in (\ref{ecuacion 22.1}). The result, of course, is not a tensor equation, 
because $x^\lambda$ is not a scalar like $V$, so it holds only in certain coordinate systems. It imposes a restriction 
on the coordinates.

If we substitute $x^\lambda$ for $V$, then for $V_{,\alpha}$ we must substitute $x^\lambda_{,\alpha} = 
g^\lambda_{\alpha}$(\footnote{Which is a constant and therefore $x^\lambda_{,\alpha\beta}\equiv 0$.}). The equation 
(\ref{ecuacion 22.1}) becomes
\begin{equation}
 \label{ecuacion 22.2}
 g^{\mu\nu} \Gamma^\lambda_{\mu\nu} = 0.
\end{equation}

Coordinates that satisfy this condition are called \emph{harmonic coordinates}. They provide the closest approximation 
to rectilinear coordinates that we can have in curve space. They provide the closes approximation to rectilinear 
coordinates that we can have in curved space. We may use them in any problem if we wish to, but very often they are not 
worthwhile because the tensor formalism with general coordinates is really quite convenient. For the discussion of 
gravitational waves, however, harmonic coordinates are very useful.

We have in general coordinates, from (\ref{ecuacion 7.9}) and (\ref{ecuacion 7.6}),
\begin{equation}
 \label{ecuacion 22.3}
 \begin{array}{rcl}
  {g^{\mu\nu}}_{,\sigma} & = & - g^{\mu\alpha}g^{\nu\beta}\left(
    \Gamma_{\alpha\beta\sigma} + \Gamma_{\beta\alpha\sigma}
  \right)\\
  & = & - g^{\nu\beta} \Gamma^\mu_{\beta\sigma} - g^{\mu\alpha} \Gamma^\nu_{\alpha\sigma}.
 \end{array}
\end{equation}
Thus, with the help of (\ref{ecuacion 20.6}),
\begin{equation}
 \label{ecuacion 22.4}
 \left(g^{\mu\nu}\sqrt{}\right)_{,\sigma} = \left( 
 - g^{\nu\beta}  \Gamma^{\mu}_{\beta\sigma}
 - g^{\mu\alpha} \Gamma^{\nu}_{\alpha\sigma}
 + g^{\mu\nu} \Gamma^\beta_{\sigma\beta}
 \right) \sqrt{}.
\end{equation}
Contracting by putting $\sigma = \nu$, we get
\begin{equation}
 \label{ecuacion 22.5}
 \left(g^{\mu\nu} \sqrt{} \right)_{,\nu} = - g^{\nu\beta}\Gamma^\mu_{\beta\nu}\sqrt{}.
\end{equation}

We see now that an alternative form for the harmonic condition is
\begin{equation}
 \label{ecuacion 22.6}
 \left(
 - g^{\mu\nu} \sqrt{}
 \right)_{,\nu} = 0.
\end{equation}


