The d'Alembert equation for a scalar $V$, namely $\Box V = 0$, gives, from (\ref{ecuacion 10.9}),
\begin{equation}
 \label{ecuacion 22.1}
 g^{\mu\nu}\left(
    V_{\mu\nu} - \Gamma^\alpha_{\mu\nu} V_{,\alpha} = 0.
 \right)
\end{equation}
If we are using rectilinear axes in flat space, each of the four coordinates $x^\lambda$ satisfies $\Box x^\lambda = 
0$. We might substitute $x^\lambda$ for $V$ in (\ref{ecuacion 22.1}). The result, of course, is not a tensor equation, 
because $x^\lambda$ is not a scalar like $V$, so it holds only in certain coordinate systems. It imposes a restriction 
on the coordinates.

If we substitute $x^\lambda$ for $V$, then for $V_{,\alpha}$ we must substitute $x^\lambda_{,\alpha} = 
g^\lambda_{\alpha}$(\footnote{Which is a constant and therefore $x^\lambda_{,\alpha\beta}\equiv 0$.}). The equation 
(\ref{ecuacion 22.1}) becomes
\begin{equation}
 \label{ecuacion 22.2}
 g^{\mu\nu} \Gamma^\lambda_{\mu\nu} = 0.
\end{equation}

