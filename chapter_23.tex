Maxwell's equations, as ordinarily written, are (\footnote{Equations (\ref{ecuacion 22.3}) and (\ref{ecuacion 
22.4}) are logical consequences of (\ref{ecuacion 22.1}) and (\ref{ecuacion 22.2}). As a consequence they can be 
omitted.}):
\begin{equation}
 \label{ecuacion 23.1}
 E = - \frac{1}{c}\frac{\partial A}{\partial t} - \mbox{grad }\phi,
\end{equation}
\begin{equation}
 \label{ecuacion 23.2}
 H = \mbox{curl } A,
\end{equation}
\begin{equation}
 \label{ecuacion 23.3}
 \frac{1}{c}\frac{\partial H}{\partial t} = - \mbox{curl } E,
\end{equation}
\begin{equation}
 \label{ecuacion 23.4}
 \mbox{div } H = 0,
\end{equation}
\begin{equation}
 \label{ecuacion 23.5}
 \frac{1}{c}\frac{\partial E}{\partial t} = \mbox{curl } H - 4 \pi j,
\end{equation}
\begin{equation}
 \label{ecuacion 23.6}
 \mbox{div } E = 4 \pi \rho.
\end{equation}

We must first put them in four-dimensional form for special relativity. The potentials $A$ and $\phi$ form a 
four-vector $\kappa^\mu$ in accordance with 
\[
k^0 = \phi \mbox{, } \kappa^m = A^m\mbox{, }(m=1,2,3).
\]

Define 
\begin{equation}
\label{ecuacion 23.7}
 F_{\mu\nu} = \kappa_{\mu,\nu} - \kappa_{\nu,\mu}
\end{equation}
Then from (\ref{ecuacion 23.1})(\footnote{\[
    F^{10} = g^{1\mu}g^{0\nu} F_{\mu\nu} = g^{11}g^{00} F_{10} = - F^{10}.
\]})
\[
 E^1 = - \frac{\partial \kappa^1}{\partial x^0} - \frac{\partial k^0}{\partial x^1}
     =   \frac{\partial \kappa_1}{\partial x^0} - \frac{\partial k_0}{\partial x^1}
     =   F_{10}
     = - F^{10}
\]
and from (\ref{ecuacion 23.2})
\[
    H^1 =   \frac{\partial \kappa^3}{\partial x^2} - \frac{\partial \kappa^2}{\partial x^3}
        = - \frac{\partial \kappa_3}{\partial x^2} + \frac{\partial \kappa_2}{\partial x^3}
        = F_{23} = F^{23}
\]
Thus the six components of the antisymmetric tensor $F_{\mu\nu}$ determine the field quantities $E$ and $H$.\footnote{
\[
\begin{array}{cc}
F_{\alpha\beta} = 
\left(
    \begin{array}{cccc}
        0    &  -E_x  & -E_y &  -E_z \\
        E_x  &    0   &  H_z &  -H_y \\
        E_y  &  -H_z  &   0  &   H_x \\
        E_z  &   H_y  & -H_x &    0
    \end{array}
\right)
&
F^{\alpha\beta} = 
\left(
    \begin{array}{cccc}
         0    &   E_x  &  E_y &   E_z \\2
        -E_x  &    0   &  H_z &  -H_y \\
        -E_y  &  -H_z  &   0  &   H_x \\
        -E_z  &   H_y  & -H_x &    0
    \end{array}
\right)
\end{array}
\]
}

From the definition (\ref{ecuacion 23.7})
\begin{equation}
 \label{ecuacion 23.8}
 F_{\mu\nu,\sigma} + F_{\nu\sigma,\mu} + F_{\sigma\mu,\nu} = 0.
\end{equation}
This gives the Maxwell Equations (\ref{ecuacion 23.3}) and (\ref{ecuacion 23.4}). We have
\begin{equation}
 \label{ecuacion 23.9}
 {F^{0\nu}}_{,\nu} = {F^{0m}}_{,m} = -{F^{m0}}_{,m} = \mbox{div } E = 4 \pi \rho
\end{equation}
from (\ref{ecuacion 23.6}). Again 
\begin{equation}
 \label{ecuacion 23.10}
 \begin{array}{rcl}
 {F^{1\nu}}_{,\nu} & = & {F^{10}}_{0} + {F^{12}}_{2} + {F^{13}}_{3} 
 = -\frac{\partial E^1}{\partial x^0} + \frac{\partial H^{3}}{\partial x^2} - \frac{\partial H^2}{\partial x^3} \\
 & = & 4 \pi j^1.
 \end{array}
\end{equation}
from (\ref{ecuacion 23.5}). The density of charge $\rho$ and current $j^\mu$ form a four-vector $J^\mu$ in accordance 
with
\[
 J^0 = \rho \, J^m = j^m.
\]
Thus (\ref{ecuacion 23.9}) and (\ref{ecuacion 23.10}) combine into
\begin{equation}
 \label{ecuacion 23.11}
 {F^{\mu\nu}}_{,\nu} = 4 \pi J^\mu.
\end{equation}
In this form the Maxwell equation are put into the four-dimensional form required by special relativity.

To pass to general relativity we must write the equations in covariant form. On account of (\ref{ecuacion 21.5}) we can 
write (\ref{ecuacion 23.7}) immediately as 
\[
 F_{\mu\nu} = \kappa_{\mu;\nu} - \kappa_{\nu;\mu}.
\]
This gives us a covariant definition of the field quantities $F_{\mu\nu}$. We have further
\[
 F_{\mu\nu;\sigma} = F_{\mu\nu,\sigma} - \Gamma^\alpha_{\mu\sigma} F_{\alpha\nu} 
                                       - \Gamma^\alpha_{\nu\sigma} F_{\mu\alpha} 
\]
Making cyclic permutations of $\mu$, $\nu$, and $\sigma$ and adding the three equations so obtained, we get
\begin{equation}
 \label{ecuacion 23.12}
   F_{\mu\nu;\sigma} + F_{\nu\sigma;\mu} + F_{\sigma\mu;\nu} 
 = F_{\mu\nu,\sigma} + F_{\nu\sigma,\mu} + F_{\sigma\mu,\nu} = 0,
\end{equation}
from (\ref{ecuacion 23.8}). So, this Maxwell equation goes over immediately to the covariant form.

Finally we must deal with the equations (\ref{ecuacion 23.11}). This is not a valid equation in general relativity and 
must be replaced with the covariant form
\begin{equation}
 \label{ecuacion 23.13}
 {F^{\mu\nu}}_{;\nu} = 4\pi J^\mu
\end{equation}
From (\ref{ecuacion 21.3}), which applies to any antisymmetric two-suffix tensor, we get
\[
 \left( F^{\mu\nu} \sqrt{} \right)_{,\nu} = 4 \pi J^\mu \sqrt{}.
\]
This leads immediately to
\[
 \left( J^\mu \sqrt{} \right)_{\mu} = (4\pi)^{-1} \left( F^{\mu\nu} \sqrt{} \right)_{,\mu\nu} = 0.
\]
So we have an equation like (\ref{ecuacion 21.2}), giving us the law of conservation of electricity. The conservation 
of electricity holds accurately, undisturbed by the curvature of space.


