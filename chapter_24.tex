The Einstein equations in the absence of matter are 
\begin{equation}
 \label{ecuacion 24.1}
 R^{\mu\nu} = 0 .
\end{equation}
They lead to 
\[
 R =  0.
\]
and hence
\begin{equation}
 \label{ecuacion 24.2}
 R^{\mu\nu} - \frac{1}{2}g^{\mu\nu} R = 0.
\end{equation}
If we start with equation (\ref{ecuacion 24.2}), we get by contraction
\[
R - 2 R = 0
\]
and so we can get back to (\ref{ecuacion 24.1}). We may either use (\ref{ecuacion 24.1}) or (\ref{ecuacion 24.2}) as 
the basic equations for empty space.

In the presence of matter these equations must be modified. Let us suppose (\ref{ecuacion 24.1}) is changed to 
\begin{equation}
 \label{ecuacion 24.3}
  R^{\mu\nu} = X^{\mu\nu} 
\end{equation}
and (\ref{ecuacion 24.2}) to
\begin{equation}
 \label{ecuacion 24.4}
  R^{\mu\nu} -\frac{1}{2} g^{\mu\nu} R = Y^{\mu\nu}.
\end{equation}
Here $X^{\mu\nu}$ and $Y^{\mu\nu}$ are symmetric two-index tensors indicating the presence of matter.

We see now that (\ref{ecuacion 24.4}) is the most convenient form to work with, because we have the Bianci relation 
(\ref{ecuacion 14.3}), which tells us that
\[
\left(R^{\mu\nu} - \frac{1}{2}g^{\mu\nu}  R \right)_{;\nu} = 0
\]
Hence (\ref{ecuacion 24.4}) requires
\begin{equation}
 \label{ecuacion 24.5}
 {Y^{\mu\nu}}_{;\nu} = 0.
\end{equation}
Any tensor $Y^{\mu\nu}$ produced by matter must satisfy this condition; otherwise the equation (\ref{ecuacion 24.4}) 
would not be consistent.

It is convenient to bring in the coefficient $-8\pi$ and to rewrite equations (\ref{ecuacion 24.4}) as
\begin{equation}
 \label{ecuacion 24.6}
  R^{\mu\nu} -\frac{1}{2} g^{\mu\nu} R = - 8 \pi Y^{\mu\nu}.
\end{equation}
We shall find that the tensor $Y^{\mu\nu}$ with this coefficient is to be interpreted as the density and flux of (non 
gravitational) energy and momentum. $Y^{\mu 0}$ is the density and $Y^{\mu\nu}$ is the flux.

In flat space equation (\ref{ecuacion 24.5}) would become
\[
{Y^{\mu\nu}}_{,\nu} = 0
\]
and would then give conservation of energy and momentum. In curved space the conservation of energy and momentum is 
only approximate. The error is to be ascribed to the gravitational field working on the matter and having itself some 
energy and momentum.
