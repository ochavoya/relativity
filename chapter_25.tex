Suppose we have a distribution of matter whose velocity varies continuously from one point to a neighboring one. If 
$z^\mu$ denotes the coordinates of an element of the matter, we can introduce the velocity vector $v^\mu = dz^\mu / d 
s$, which will be a continuous function of the $x$'s, like a field function. It has the properties
\begin{equation}
\label{ecuacion 25.1}
 g_{\mu\nu} v^\mu v^\nu = 1
\end{equation}
\[
0 = (g_{\mu\nu} v^\mu v^\nu)_{;\sigma} 
  = g_{\mu\nu}\left(v^\mu {v^\nu}_{;\sigma} + {v^\mu}_{;\sigma} v^\nu \right) = 2 g_{\mu\nu} v^\mu {v^\nu}_{;\sigma}.
\]
Thus
\begin{equation}
 \label{ecuacion 25.2}
 v_\nu {v^\nu}_{;\sigma} = 0.
\end{equation}

We may introduce a scalar field $\rho$ such that the vector $\rho v^\mu$ determines the density and flow of the matter, 
just like $J^\nu$ determines the density and flow of electricity; that is $\rho v^0 \sqrt{}$ is the density and $\rho 
v^m \sqrt{}$ is the flow. The condition for the conservation of matter is 
\[
\left( \rho v^\mu \sqrt{} \right)_{,\mu} = 0
\]
or
\begin{equation}
 \label{ecuacion 25.3}
 \left( \rho v^\mu \right)_{;\mu} = 0.
\end{equation}

The matter that we are considering will have an energy density $\rho v^0 v^0\sqrt{}$ and energy flux $\rho v^0 v^m 
\sqrt{}$, and similarly a momentum density $\rho v^n v^0 \sqrt{}$ and momentum flux $\rho v^n v^m \sqrt{}$. Put 
\begin{equation}
 \label{ecuacion 25.4}
 T^{\mu\nu} = \rho v^\mu v^\nu.
\end{equation}
Then $T^{\mu\nu}\sqrt{}$ gives the density and flux of energy and momentum. $T^{\mu\nu}$ is called the material energy 
tensor. It is, of course, symmetric.

