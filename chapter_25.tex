Suppose we have a distribution of matter whose velocity varies continuously from one point to a neighboring one. If 
$z^\mu$ denotes the coordinates of an element of the matter, we can introduce the velocity vector $v^\mu = dz^\mu / d 
s$, which will be a continuous function of the $x$'s, like a field function. It has the properties
\begin{equation}
\label{ecuacion 25.1}
 g_{\mu\nu} v^\mu v^\nu = 1
\end{equation}
\[
0 = (g_{\mu\nu} v^\mu v^\nu)_{;\sigma} 
  = g_{\mu\nu}\left(v^\mu {v^\nu}_{;\sigma} + {v^\mu}_{;\sigma} v^\nu \right) = 2 g_{\mu\nu} v^\mu {v^\nu}_{;\sigma}.
\]
Thus
\begin{equation}
 \label{ecuacion 25.2}
 v_\nu {v^\nu}_{;\sigma} = 0.
\end{equation}

We may introduce a scalar field $\rho$ such that the vector $\rho v^\mu$ determines the density and flow of the matter, 
just like $J^\nu$ determines the density and flow of electricity; that is $\rho v^0 \sqrt{}$ is the density and $\rho 
v^m \sqrt{}$ is the flow. The condition for the conservation of matter is 
\[
\left( \rho v^\mu \sqrt{} \right)_{,\mu} = 0
\]
or
\begin{equation}
 \label{ecuacion 25.3}
 \left( \rho v^\mu \right)_{;\mu} = 0.
\end{equation}

The matter that we are considering will have an energy density $\rho v^0 v^0\sqrt{}$ and energy flux $\rho v^0 v^m 
\sqrt{}$, and similarly a momentum density $\rho v^n v^0 \sqrt{}$ and momentum flux $\rho v^n v^m \sqrt{}$. Put 
\begin{equation}
 \label{ecuacion 25.4}
 T^{\mu\nu} = \rho v^\mu v^\nu.
\end{equation}
Then $T^{\mu\nu}\sqrt{}$ gives the density and flux of energy and momentum. $T^{\mu\nu}$ is called the material energy 
tensor. It is, of course, symmetric.

Can we use $T^{\mu\nu}$ for the matter term in the right-hand side of the Einstein equation (\ref{ecuacion 24.6})? For 
this purpose we require ${T^{\mu\nu}}_{;\nu} = 0$. We have from the definition (\ref{ecuacion 25.4})
\[
    {T^{\mu\nu}}_{;\nu} = \left( \rho v^\mu v^\nu \right)_{;\nu} 
                        = v^\mu \left( \rho v^\nu \right)_{;\nu} + \rho v^\nu {v^\mu}_{;\nu}
\]
The first term here vanishes from the condition for conservation of mass (\ref{ecuacion 25.3}). The second term 
vanishes if matter moves along geodesics for, if $v^\mu$ is define as a continuous field function instead of having a 
meaning only on one world line, we have
\[
 \frac{d v^\mu}{ds} = {v^\mu}_{;\nu}v^\nu.
\]
So (\ref{ecuacion 8.3}) becomes
\[
 \left( {v^\mu}_{,\nu} + \Gamma^\mu_{\nu\sigma}v^\sigma \right) v^\nu = 0
\]
or
\begin{equation}
 \label{ecuacion 25.5}
 {v^\mu}_{;\nu} v^\nu = 0.
\end{equation}

We see now that we can substitute the material energy tensor (\ref{ecuacion 25.4}), with a suitable numerical 
coefficient $k$, into the Einstein equation (24.4). We get
\begin{equation}
 \label{ecuacion 25.6}
 R_{\mu\nu} - \frac{1}{2} g^{\mu\nu} R = k \rho v^\mu v^\nu.
\end{equation}

We shall now determine the value of the coefficient $k$. We go over to the Newtonian approximation, following the 
method of Section 16. We note first that, contracting (\ref{ecuacion 25.6}), we get
\[
 - R = k \rho .
\]
So (\ref{ecuacion 25.6}) may be written 
\[
 R^{\mu\nu} = k \rho \left( v^\mu v^\nu - \frac{1}{2} g^{\mu\nu} \right).
\]
 
With the weak field approximation we get, corresponding to (16.4),
\[
 \frac{1}{2} g^{\rho\sigma} \left(
     g_{\rho\sigma,\mu\nu} - g_{\nu\sigma,\mu\rho} - g_{\mu\rho,\nu\sigma} + g_{\mu\nu,\rho\sigma}
 \right)
 = k \rho \left(
     v_\mu v_\nu - \frac{1}{2}g_{\mu\nu}
 \right).
\]
We now take a static field and a static distribution of matter, so that $v_0 = 1$, $v_m = 0$. Putting $\mu = \nu = 0$ 
and neglecting second order quantities, we find 
\[
 -\frac{1}{2} \nabla^2 g_{00} = \frac{1}{2} k \rho
\]
or from (\ref{ecuacion 16.6})
\[
 \nabla^2 V = -\frac{1}{2} k \rho
\]
to agree with the Poisson equation we must take $k = - 8 \pi$.

The Einstein equation for the presence of a distribution of matter with a velocity field thus reads
\begin{equation}
 \label{ecuacion 25.7}
 R^{\mu\nu} - \frac{1}{2} g^{\mu\nu} R = - 8\pi \rho v^\mu v^\nu.
\end{equation}
Thus $T^{\mu\nu}$, given by (\ref{ecuacion 25.4}), is precisely the $Y^{\mu\nu}$ of equation (\ref{ecuacion 24.6}).

The condition for conservation of mass (\ref{ecuacion 25.3}) gives
\[
 \rho_{;\mu} v^{\mu} + \rho {v^\mu}_{;\mu};
\]
hence
\begin{equation}
 \label{ecuacion 25.8}
 \frac{d\rho}{ds} = \frac{\partial \rho}{\partial x^\mu}v^\mu = - \rho {v^\mu}_{;\mu}.
\end{equation}
This is a condition that fixes how $\rho$ varies along the world line of an element of matter. It allows $\rho$ to vary 
arbitrarily from the world line of one element to that of a neighboring element. Thus we may take $\rho$ to vanish 
except for a packet of world lines forming a tube in space-time. Such a packet would compose a particle of matter of a 
finite size. Outside the particle  we heave $\rho = 0$, and Einstein's field equation for empty space holds.

It should be noted that, if one assumes the general field equation (\ref{ecuacion 25.7}), one can deduce from it two 
things: (a) the mass is conserved and  (b) the mass moves along geodesics. To do this we note that $(\mbox{left-hand 
side})_{;\nu}$ vanishes from Bianci's relation, so the equation gives
\[
\left( \rho v^\mu v^\nu \right)_{;\nu} = 0,
\]
or
\begin{equation}
 \label{ecuacion 25.9}
 v^\mu \left( \rho v^\nu \right)_{;\nu} + \rho v^\nu {v^\mu}_{;\nu} = 0.
\end{equation}
multiply this equation by $v_{\mu}$. The second term gives zero from (\ref{ecuacion 25.2}) and we are left with 
$\left( 
    \rho v^\nu
\right)_{;\nu} = 0$
which is just the conservation equation (\ref{ecuacion 25.3}). Equation (\ref{ecuacion 25.9}) now reduces to $v^\nu 
{v^\mu}_{;\nu} = 0$ which is the geodesic equation. It is thus not necessary to make the separate assumption that a 
particle moves along a geodesic. With a small particle the motion is constrained to lie along a geodesic by the 
application of Einstein's equations for empty space to the space around the particle.





