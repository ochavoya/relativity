Introduce the scalar
\begin{equation}
 \label{ecuacion 26.1}
 I = \int{R \sqrt{} d^4 x}
\end{equation}
integrated over a certain four-dimensional volume. Make small variations $\delta g_{\mu\nu}$ in the $g_{\mu\nu}$, 
keeping the $g_{\mu\nu}$ and their first derivatives constant on the boundary. We shall find that putting $\delta I = 
0$ for arbitrary $\delta g_{\mu\nu}$ gives Einstein's vacuum equations.

We have from (\ref{ecuacion 14.4})
\[
 R = g^{\mu\nu} R_{\mu\nu} = R^{*} - L,
\]
where
\begin{equation}
 \label{ecuacion 26.2}
 R^{*} = g^{\mu\nu} \left( \Gamma^\sigma_{\mu\sigma,\nu} - \Gamma^\sigma_{\mu\nu,\sigma} \right).
\end{equation}
and
\begin{equation}
 \label{ecuacion 26.3}
 L = g^{\mu\nu} \left( 
    \Gamma^\sigma_{\mu\nu}\Gamma^\rho_{\sigma\rho} - \Gamma^\rho_{\mu\sigma}\Gamma^\sigma_{\nu\rho}
 \right).
\end{equation}
$I$ involves second derivatives of $g_{\mu\nu}$, since these second derivatives occur in $R^{*}$. But they occur only 
linearly, so they can be removed by partial integration. We have
\begin{equation}
 \label{ecuacion 26.4}
 R^{*}\sqrt{} = \left( 
    g^{\mu\nu} \Gamma^\sigma_{\mu\sigma} \sqrt{}
 \right)_{,\nu} -
 \left( 
    g^{\mu\nu} \Gamma^\sigma_{\mu\nu} \sqrt{}
 \right)_{,\sigma}
 -\left(
    g^{\mu\nu}
 \right)_{,\nu} \Gamma^\sigma_{\mu\sigma}
 +\left(
    g^{\mu\nu}
 \right)_{,\sigma} \Gamma^\sigma_{\mu\nu}
\end{equation}
The first two terms are perfect differentials, so they will contribute nothing to $I$. We therefore need to retain only 
the last two terms of (\ref{ecuacion 26.4}). With the help of (\ref{ecuacion 22.5}) and (\ref{ecuacion 22.4}) they 
become
\[
 g^{\nu\beta}\Gamma^\mu_{\beta\nu}\Gamma^\sigma_{\mu\sigma}\sqrt{}
 + \left(
     -2 g^{\nu\beta} \Gamma^\mu_{\beta\sigma} + g^{\mu\nu} \Gamma^\beta_{\sigma\beta}
 \right)
 \Gamma^\sigma_{\mu\nu} \sqrt{}.
\]
This is just $2L\sqrt{}$, from (\ref{ecuacion 26.3}). So (\ref{ecuacion 26.1}) becomes
\[
 I = \int{L\sqrt{} d^4 x},
\]
which involves only the $g^{\mu\nu}$ and their first derivatives. It is homogeneous of the second degree in the first 
derivatives.

Put $\mathcal{L} = L \sqrt{}$. We take it (with a suitable numerical coefficient) as the action density of the 
gravitational field. It is not a scalar density but it is more convenient than $R\sqrt{}$, which is a scalar density, 
but it does not involve second derivatives of the $g_{\mu\nu}$.

According to ordinary ideas of dynamics, the action is the time integral of the Lagrangian. We have(\footnote{Notice 
that when he wrote this equation, the author considered $d x_0$ as a \emph{strictly} temporal coordinates, thus, 
restricting the set of admissible systems of coordinates.})
\[
 I = \int{\mathcal{L} d^4 x} = \int{ d x_0 \int{\mathcal{L} dx^1 dx^2 dx^3}}.
\]
Thus $\mathcal{L}$ may be considered as the Lagrangian density (in three dimensions) as well as the action density (in 
four dimensions). We may look upon the $g_{\mu\nu}$ as dynamical coordinates and their time derivatives as their 
velocities. We then see that the Lagrangian is quadratic (nonhomogeneous) in the velocities, as it usually is in 
ordinary dynamics.

We must now vary $\mathcal{L}$. We have, using (\ref{ecuacion 20.6}),
\begin{equation}
 \label{ecuacion 26.5}
 \begin{array}{rcl}
     \delta \left( \Gamma^\alpha_{\mu\nu} \Gamma^\beta_{\alpha\beta} g^{\mu\nu} \sqrt{}\right) & = &
     \Gamma^\alpha_{\mu\nu} \delta \left( \Gamma^\beta_{\alpha\beta} g^{\mu\nu} \sqrt{}\right) + 
     \Gamma^\beta_{\alpha\beta} g^{\mu\nu} \sqrt{}\delta \left( \Gamma^\alpha_{\mu\nu} \right)\\
     & = & \Gamma^\alpha_{\mu\nu} \delta\left( {g^{\mu\nu}\sqrt{}}_{,\alpha}\right)
        + \Gamma^\beta_{\alpha\beta} \delta\left( \Gamma^\alpha_{\mu\nu} g^{\mu\nu} \sqrt{} \right)
        - \Gamma^\beta_{\alpha\beta} \Gamma^\alpha_{\mu\nu} \delta\left( g^{\mu\nu} \sqrt{} \right)\\
     & = & \Gamma^\alpha_{\mu\nu} \delta\left( {g^{\mu\nu}\sqrt{}}_{,\alpha}\right)
        - \Gamma^\beta_{\alpha\beta} \delta\left(  g^{\alpha\nu} \sqrt{} \right)_{,\nu}
        - \Gamma^\beta_{\alpha\beta} \Gamma^\alpha_{\mu\nu} \delta\left( g^{\mu\nu} \sqrt{} \right)
 \end{array}
\end{equation}
with the help of (\ref{ecuacion 22.5})


