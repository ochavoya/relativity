Introduce the scalar
\begin{equation}
 \label{ecuacion 26.1}
 I = \int{R \sqrt{} d^4 x}
\end{equation}
integrated over a certain four-dimensional volume. Make small variations $\delta g_{\mu\nu}$ in the $g_{\mu\nu}$, 
keeping the $g_{\mu\nu}$ and their first derivatives constant on the boundary. We shall find that putting $\delta I = 
0$ for arbitrary $\delta g_{\mu\nu}$ gives Einstein's vacuum equations.

We have from (\ref{ecuacion 14.4})
\[
 R = g^{\mu\nu} R_{\mu\nu} = R^{*} - L,
\]
where
\begin{equation}
 \label{ecuacion 26.2}
 R^{*} = g^{\mu\nu} \left( \Gamma^\sigma_{\mu\sigma,\nu} - \Gamma^\sigma_{\mu\nu,\sigma} \right).
\end{equation}
and
\begin{equation}
 \label{ecuacion 26.3}
 L = g^{\mu\nu} \left( 
    \Gamma^\sigma_{\mu\nu}\Gamma^\rho_{\sigma\rho} - \Gamma^\rho_{\mu\sigma}\Gamma^\sigma_{\nu\rho}
 \right).
\end{equation}
$I$ involves second derivatives of $g_{\mu\nu}$, since these second derivatives occur in $R^{*}$. But they occur only 
linearly, so they can be removed by partial integration. We have
\begin{equation}
 \label{ecuacion 26.4}
 R^{*}\sqrt{} = \left( 
    g^{\mu\nu} \Gamma^\sigma_{\mu\sigma} \sqrt{}
 \right)_{,\nu} -
 \left( 
    g^{\mu\nu} \Gamma^\sigma_{\mu\nu} \sqrt{}
 \right)_{,\sigma}
 -\left(
    g^{\mu\nu}
 \right)_{,\nu} \Gamma^\sigma_{\mu\sigma}
 +\left(
    g^{\mu\nu}
 \right)_{,\sigma} \Gamma^\sigma_{\mu\nu}
\end{equation}
The first two terms are perfect differentials, so they will contribute nothing to $I$. We therefore need to retain only 
the last two terms of (\ref{ecuacion 26.4}). With the help of (\ref{ecuacion 22.5}) and (\ref{ecuacion 22.4}) they 
become
\[
 g^{\nu\beta}\Gamma^\mu_{\beta\nu}\Gamma^\sigma_{\mu\sigma}\sqrt{}
 + \left(
     -2 g^{\nu\beta} \Gamma^\mu_{\beta\sigma} + g^{\mu\nu}
 \right)
\]



