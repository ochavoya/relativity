\documentclass[12pt]{article}
\usepackage[utf8]{inputenc}
\usepackage{amsmath,latexsym}
\numberwithin{equation}{section}

\title{General Theory of Relativity}
\author{P.A.M. Dirac\\Florida State University}
\date{1975}
\begin{document}
\maketitle
\section{Special relativity}
For the space-time of physics we need four coordinates, the time $t$ and three 
space coordinates $x$, $y$, $z$. We put 
\[
 t = x^0, x = x^1, y = x^2, z = x^3,
\]
so that the four coordinates may be written $x^\mu$, where the suffix $\mu$ 
takes on the values 0, 1, 2, 3. The suffix is written in the upper position so 
that we may maintain a ``balancing'' of the suffixes in all the general 
equations of the theory. The precise meaning of ``balancing'' will become clear 
a little later.

Let us take a point close to the point that we originally considered and let 
its coordinates be $x^\mu + dx^\mu$. The four quantities $dx^\mu$ which form 
the displacement may be considered as the components of a vector. The laws of 
special relativity allow us to make linear nonhomogeneous transformations of the 
coordinates, resulting in linear homogeneous transformations of the $dx^\mu$. 
These are such that, if we choose units of distance and of time such that the 
velocity of light is unity,
\begin{equation}
\label{ecuacion 1.1}
   \left( dx^0 \right)^2 
 - \left( dx^1 \right)^2 
 - \left( dx^2 \right)^2 
 - \left( dx^3 \right)^2 
\end{equation}
is invariant.

Any set of quantities $A^\mu$ that transform under a change of coordinates in 
the same way as the $dx^\mu$ form what is called a \emph{contravariant vector}. 
The invariant quantity 
\begin{equation}
 \label{ecuacion 1.2}
 \left(A^0 \right)^2 -  \left(A^1 \right)^2 -  \left(A^2 \right)^2 - \left(A^3 \right)^2 = \left( A , A \right) 
\end{equation}
may be called the squared length of the vector. With a second contravariant 
vector we have the scalar product invariant
\begin{equation}
 \label{ecuacion 1.3}
 A^0 B^0  - A^1 B^1 - A^2 B^2 - A^3 B^3 = \left( A, B \right)
\end{equation}

In order to get a convenient way of writing such invariants we introduce the 
device of lowering suffixes. Define
\begin{equation}
 \label{ecuacion 1.4}
 A_0 = A^0,\, A_1 = -A^1,\, A_2 = -A^2,\, A_3 = -A^3 .
\end{equation}
Then the expression on the left side of (\ref{ecuacion 1.2}) may be written 
$A_\mu A^\mu$, in which it is understood that a summation is to be taken over 
the four values of $\mu$. With the same notation we can write (\ref{ecuacion 
1.3}) as $A_\mu B^\mu$ or else $A^\mu B_\mu$.

The four quantities $A_\mu$ introduced by equation (\ref{ecuacion 1.4}) may 
also be considered as the components of a vector. Their transformation laws 
under a change of coordinates are somewhat different from those of $A^\mu$, 
because of the differences in sign, and the vector is called a \emph{covariant 
vector}.(\footnote{
In fact, if we set $A_\mu = g_{\mu\nu} A^\nu$ where
\[
  g_{\mu\nu} = 
  g^{\mu\nu} =
  \left(
    \begin{array}{cccc}
        1 &  0 &  0 &  0 \\
        0 & -1 &  0 &  0 \\
        0 &  0 & -1 &  0 \\
        0 &  0 &  0 & -1
    \end{array}
  \right)
\]
we can see that the coefficients of the transformation ${A'_{\mu}} = 
{\Lambda_\mu}^\nu A_\nu$ are linearly related to the coefficients of the 
transformation ${A'}^\mu = {\Lambda^\mu}_\nu A^\nu$.
})

From two contravariant vectors $A^\mu$ and $B^\mu$ we may form the sixteen 
quantities $A^\mu B^\nu$. The suffix $\nu$, like all the Greek suffixes 
appearing in this work, also takes on the four values 0, 1, 2, 3. Those sixteen 
quantities form the components of a tensor of the second rank. It is sometimes 
called the outer product of the vectors $A^\mu$ and $B^\nu$, as distinct of the 
scalar product (\ref{ecuacion 1.3}) which is called the inner product.

The tensor $A^\mu B^\nu$ is rather a special tensor because there are special 
relations between its components. But we can add together several tensors 
constructed in this way to get a general tensor of the second rank; say
\begin{equation}
 \label{ecuacion 1.5}
 T^{\mu\nu} = A^\mu B^\nu + {A'}^\mu {B'}^\nu + {A''}^\mu {B''}^\nu + \cdots
\end{equation}
The important thing about the general tensor is that under a transformation of 
coordinates its components transform in the same way as the components of $A^\mu 
B^\nu$.

We may lower one of the suffixes in $T^{\mu\nu}$ by applying the lowering 
process to each of the terms on the right hand side of (\ref{ecuacion 1.5}). 
Thus we may form ${T_\mu}^\nu$ or ${T^\mu}_\nu$.

In ${T_\mu}^\nu$ we may set $\nu=\mu$ and get ${T_\mu}^\mu$. This is to be 
summed over the four values of $\mu$. A summation is always implied over a 
suffix that occurs twice in a term. Thus ${T_\mu}^\mu$ is scalar. It is equal to 
${T^\mu}_\mu$.

We may continue this process and multiply more than two vectors together, 
taking care that their suffixes are all different. In this way we can construct 
tensors of higher rank. If the vectors are all contravariant, we get a vector 
with all its suffixes upstairs. We may then lower any of the suffixes and so get 
a general tensor with any number of suffixes upstairs and any number downstairs.

We may set a downstairs suffix equal to an upstairs one. We then have to sum 
over all values of this suffix. The suffix becomes a dummy. We are left with a 
tensor having two fewer effective suffixes than the original one. This process 
is called \emph{contraction}. Thus, if we start with the fourth rank tensor 
${{T^\mu}_{\nu\rho}}^{\sigma}$, one way of contracting it is to put $\sigma = 
\rho$, which gives the second rank tensor ${{T^\mu}_{\nu\rho}}^{\rho}$, 
having only sixteen components, arising from the four values of $\mu$ and $\nu$. 
We could contract again to get the scalar ${{T^\mu}_{\mu\rho}}^{\rho}$, with 
just one component.

At this stage one can appreciate the balancing of suffixes. Any effective 
suffix occurring in an equation appears once and only once in each term of the 
equation, and always upstairs or only downstairs. A suffix appearing twice in a 
term is a dummy, and it must occur once upstairs and once downstairs. It may be 
replaced by any other Greek letter not already mentioned in the term. Thus 
${{T^\mu}_{\nu\rho}}^{\rho} = {{T^\mu}_{\nu\alpha}}^{\alpha}$. A suffix must 
never occur more than twice in a term.

\section{Oblique axes}
Before passing to the formalism of general relativity it is convenient to consider an intermediate formalism--special relativity referred to oblique rectilinear axes.

If we make a transformation to oblique axes, each of the $dx^\mu$ mentioned in (\ref{ecuacion 1.1}) becomes a linear function the new $dx^\mu$ and the quadratic form (\ref{ecuacion 1.1}) becomes a general quadratic form in the new $dx^\mu$. We may write it
\begin{equation}
 \label{ecuacion 2.1}
 g_{\mu\nu} dx^\mu dx^\nu
\end{equation}
with summation understood over both $\mu$ and $\nu$. The coefficients $g_{\mu\nu}$ appearing here depend on the system of oblique axes. Of course we take $g_{\mu\nu} = g_{\nu\mu}$ because any difference of $g_{\mu\nu}$ and $g_{\nu\mu}$ would not show up in the quadratic form (\ref{ecuacion 2.1}). There are thus ten independent coefficients $g_{\mu\nu}$.\footnote{In general the number of independent components in a symmetric matrix of $n\times n$ dimensions is equal to the number of unordered pairs of $n$ elements. From the total number of ordered pairs we subtract $n$ pairs with identical elements, and we are left with $n^2 - n$. We divide the last number by two to obtain the number of unordered pairs with distinct elements. The total number of uordered pairs is then \[ T_n = \frac{1}{2} n \cdot (n+1)\] which is the $n$th triangular number. Then we notice that $T_{n-1}$ is the number of independent components of an antisymmetric matrix of $n\times n $ dimensions. In the case of four dimensions this number is $T_3 = 6$.}

A general contravariant vector has four components $A^\mu$ which transform like the $dx^\mu$ under any transformation of the oblique axes. Thus 
\[
 g_{\mu\nu} A^\mu A^\nu
\]
is invariant. It is the squared length of the vector $A^\mu$. 

Let $B^\mu$ be a second contravariant vector; then $A^\mu + \lambda B^\mu$ is still another, for any value of the number $\lambda$. It's squared length is
\[
 g_{\mu\nu}
 \left( A^\mu + \lambda B^\mu \right)
 \left( A^\nu + \lambda B^\nu \right)
 = g_{\mu\nu} A^\mu A^\nu 
 + \lambda \left( g_{\mu\nu} A^\mu B^\nu + g_{\mu\nu} A^\nu B^\mu \right)
 + \lambda^2 g_{\mu\nu} B^\mu B^\nu .
\]
This must be an invariant for all values of $\lambda$. It follows that the term independent of $\lambda$ and the coefficients of $\lambda$ and $\lambda^2$ must separately be invariants. The coefficient of $\lambda$ is
\[
 g_{\mu\nu} A^\mu B^\nu + g_{\mu\nu} A^\nu B^\mu = 2 g_{\mu\nu} A^\mu B^\nu
\]
since in the second term in the left we can interchange $\mu$ and $\nu$ and then set $g_{\mu\nu} = g_{\nu\mu}$. Thus we find that $g_{\mu\nu} A^\mu B^\nu$ is an invariant. It is the scalar product of $A^\mu$ and $B^\mu$.

Let $g$ be the determinant of $g_{\mu\nu}$. It must not vanish; otherwise the four axes would not provide independent directions in space-time and would no be suitable as axes. For the orthogonal axes of the preceding section the diagonal elements of $g_{\mu\nu}$ are $1$, $-1$, $-1$, $-1$ and the nondiagonal elements are zero. Thus $g=-1$. With oblique axes $g$ must still be negative, because the oblique axes can be obtained from the orthogonal ones by a continuous process, resulting ig $g$ varying continuosly, and $g$ cannot pass through the value zero.

Define the covariant vector $A_\mu$, with a downstairs suffix, by
\begin{equation}
 \label{ecuacion 2.2}
 A_\mu = g_{\mu\nu} A^\nu.
\end{equation}
Since the determinant $g$ does not vanish, these equations can be solved for $A^\nu$ in terms of the $A_\mu$. Let the result be
\begin{equation}
 \label{ecuacion 2.3}
 A^\nu = g^{\mu\nu} A_\mu.
\end{equation}
Each $g^{\mu\nu}$ equals the cofactor of the corresponding $g_{\mu\nu}$, divided by the determinant itself. It follows that $g^{\mu\nu} = g^{\nu\mu}$.

Let us substitute for the $A^\nu$ in (\ref{ecuacion 2.2}) sus valores dados por (\ref{ecuacion 2.3}). We must replace the dummy $\mu$ in (\ref{ecuacion 2.3}) by some other Greek letter, say $\rho$, in order not to have three $\mu$'s in the same term. We get
\[
 A\mu = g_{\mu\nu} g^{\nu\rho}A_\rho.
\]
Since this equation must hold for any four quantities $A_\mu$, we can infer 
\begin{equation}
 \label{ecuacion 2.4}
 g_{\mu\nu} g^{\nu\rho} = g_{\mu}^{\rho},
\end{equation}
where
\begin{equation}
 \label{ecuacion 2.5}
 \begin{array}{rclr}
    g_{\mu}^{\rho} & = & 1 & \mbox{ for } \mu = \rho,\\
                   & = & 0 & \mbox{ for } \mu \ne \rho.
 \end{array}
\end{equation}

The formula (\ref{ecuacion 2.2}) may be used to lower any upper suffix occurring in a tensor. Similarly, (\ref{ecuacion 2.3}) can be used to raise any downstairs suffix. If a suffix is lowered and raised agian, he result is the same as the original tensor, on account of (\ref{ecuacion 2.4}) and \ref{ecuacion 2.5}. Note that $g_{\mu}^{\nu}$ just produces a substitution of $\rho$ for $\mu$ or of $\mu$ for $\rho$,
\[
 g_{\mu}^{\rho} A^{\mu} = A^{\rho},
\]
and of $\mu$ for $\rho$,
\[
 g_{\mu}^{\rho} A_{\rho} = A_{\mu}.
\]

if we apply the rule to raising a suffix to the $\mu$ in $g_{\mu\rho}$ we get
\[
 {g^{\alpha}}_{\nu} = g^{\alpha\mu} g_{\mu\nu}.
\]
This agrees with (\ref{ecuacion 2.4}) if we take into accoun that in ${g^{\alpha}}_{\nu}$ we may write the suffixes one above the other because of the symmetry of $g_{\mu\nu}$. Further we may raise the suffix $\nu$ by the same rule and get
\[
 g^{\alpha\beta} = g^{\nu\beta} g^{\alpha}_{\nu},
\]
a result which follows immediately form (\ref{ecuacion 2.5}). The rules for raising and lowering suffixes apply to all the suffixes in $g_{\mu\nu}$, $g^{\mu}_{\nu}$, $g^{\mu\nu}$.









\section{Curvilinear coordinates}
We now pass on to a system of curvilinear coordinates. We shall deal with 
quantities which are located at a point in space. Such a  quantity may have 
various components, which are then referred to the axes at that point. There may 
be a quantity of the same nature at all points in space. It then becomes a field 
quantity.

If we take such quantity $Q$ (or one of its components if it has several), we 
can differentiate it with respect to any of the four coordinates. We write the 
result
\[
 \frac{\partial Q}{\partial x^\mu} = Q_{,\mu}
\]
A downstairs suffix preceded by a comma will always denote a derivative in this 
way. We put the suffix $\mu$ downstairs in order to balance the upstairs $\mu$ 
in the denominator on the left. We can see that the suffixes balance by noting 
that the change in $Q$, when we pass from a point $x^\mu$ to the neighboring 
point $x^\mu + \delta x^\mu$ is
\begin{equation}
 \label{ecuacion 3.1}
 \delta Q = Q_{,\mu} \delta x^\mu .
\end{equation}

We shall have vectors and tensors located at a point, with various components 
referring to the axes at that point. When we change our system of coordinates, 
the components will change according to the same laws as in the preceding 
section, depending on the change of axes at the point concerned. We shall have a 
$g_{\mu\nu}$ and a $g^{\mu\nu}$ to lower and raise suffixes, as before. But 
\emph{they are no longer constants}. They vary from point to point. They are 
field quantities.

Let us see the effect of a particular change in the coordinate system. Take new 
curvilinear coordinates ${x'}^{\mu}$, each a function of the four 
$x^{\mu}$'s. They may be written more conveniently $x^{\mu'}$, with the prime 
attached to the suffix rather than the main symbol.

Making a small variation in the $x^{\mu}$, we get the four quantities $\delta 
x^{\mu}$ forming a contravariant vector. Referred to the new axes, this vector 
has the components
\[
    \delta {x^{\mu'}} 
    = \frac{\partial x^{\mu'}}{\partial x^{\nu}} \delta x^{\nu}
    = x^{\mu'}_{,\nu} \delta x^\nu
\]
with the notation of (\ref{ecuacion 3.1}). This gives the law for he 
transformation of any contravariant vector $A^{\nu}$; namely,
\begin{equation}
 \label{ecuacion 3.2}
 A^{\mu'} = x^{\mu'}_{,\nu} A^\nu
\end{equation}
Interchanging the two systems of axes and changing the suffixes, we get
\begin{equation}
 \label{ecuacion 3.3}
 A^{\lambda} = x^{\lambda}_{,\mu'} A^{\mu'}
\end{equation}

We know from the laws of partial differentiation that
\begin{equation}
 \label{ecuacion 3.4}
 \frac{\partial x^{\lambda}}{\partial x^{\mu'}} 
\frac{\partial x^{\mu'}}{\partial x^{\nu}} = g^{\lambda}_{\nu}.
\end{equation}

To see how a covariant vector $B_{\mu}$ transforms, we use the condition that 
$A^{\mu} B_{\mu}$ is invariant. Thus with the help of (\ref{ecuacion 3.3})
\[
    A^{\mu'} B_{\mu'} 
    = A^{\lambda} B_{\lambda} 
    = x^{\lambda}_{,\mu'}A^{\mu'} B_{\lambda}
\]
This result must hold for all values of the four $A^{\mu'}$; therefore we can 
equate the coefficients of $A^{\mu'}$ and get
\begin{equation}
 \label{ecuacion 3.5}
 B_{\mu'} = x^{\lambda}_{,\mu'} B_{\lambda}
\end{equation}

We can now use the formulas (\ref{ecuacion 3.2}) and (\ref{ecuacion 3.5}) to 
transform any tensor with any upstairs and downstairs suffixes. We just have to 
use coefficients like $x^{\mu'}_{,\nu}$ for each upstairs suffix and 
$x^{\lambda}_{,\nu'}$ for each downstairs suffix and make all the suffixes 
balance. For example
\begin{equation}
 \label{ecuacion 3.6}
 {T^{\alpha'\beta'}}_{\gamma'} =
    x^{\alpha'}_{,\lambda}x^{\beta'}_{,\mu}x^{\nu}_{,\gamma'} 
    {T^{\lambda\mu}}_{\nu}.
\end{equation}
Any quantity that transforms according to this law is a tensor. This may be 
taken as the definition of a tensor.

It should be noted that it has a meaning for a tensor to be symmetrical or 
antisymmetrical between two suffixes like $\lambda$ and $\mu$, because this 
property of symmetry is preserved with the change of coordinates.(\footnote{
In fact if $T^{\mu\nu}$ is tensor and ($T^{\mu\nu} = \pm T^{\nu\mu}$). By 
the transformation law 
\[T^{\mu'\nu'} = x^{\mu'}_{,\alpha} x^{\nu'}_{,\beta}T^{\alpha\beta}.\]
Using the property of symmetry
\[T^{\mu'\nu'} = \pm x^{\mu'}_{,\alpha} x^{\nu'}_{,\beta}T^{\beta\alpha}
               = \pm x^{\nu'}_{,\beta}  x^{\mu'}_{,\alpha}T^{\beta\alpha}\]
and reordering the dummy suffixes
\[T^{\mu'\nu'} = \pm x^{\nu'}_{,\alpha} x^{\mu'}_{,\beta}T^{\alpha\beta}
               = \pm T^{\nu'\mu'}.\]
})

The formula (\ref{ecuacion 3.4}) may be written
\[
    x^{\lambda}_{,\alpha'}x^{\beta'}_{,\nu} 
    g^{\alpha'}_{\beta'} = g^{\lambda}_{\nu}
\]
It just shows that $g^{\lambda}_{\nu}$ is a tensor. We have also, for any 
vectors $A^\mu$, $B^\nu$,
\[
    g_{\alpha'\beta'} A^{\alpha'} A^{\beta'} = g_{\mu\nu} A^\mu B^\nu
    = g_{\mu\nu} x^{\mu}_{,\alpha'} x^{\nu}_{,\beta'} A^{\alpha'} A^{\beta'}.
\]
Since this holds for all values of $A^{\alpha'}$, $B^{\beta'}$, we can infer
\begin{equation}
 \label{ecuacion 3.7}
 g_{\alpha'\beta'} = x^{\mu}_{,\alpha'} x^{\nu}_{,\beta'} g_{\mu\nu}.
\end{equation}
This shows that $g_{\mu\nu}$ is a tensor. Similarly, $g^{\mu\nu}$ is a tensor. 
They are called the \emph{fundamental tensors}.

If $S$ is any scalar field quantity, it can be considered as a function of the 
four $x^\mu$ or the four $x^{\mu'}$. From the laws of partial differentiation 
\[
    S_{,\mu'} = S_{,\lambda} x^{\lambda}_{,\mu'}.
\]
Hence the $S_{,\lambda}$ transform like the $B_{\lambda}$ of equations 
(\ref{ecuacion 3.5}) and thus \emph{the derivative of a scalar field is a 
covariant vector field.}(\footnote{As a matter of fact, a covariant 
(cogradient) vector can be defined as an array of four quantities $A_{\mu}$ that 
transform like the components of the gradient of a scalar.  A contravariant 
(contragradient) vector can be defined as an array of four quantities that 
transform like the differentials $dx^\mu$.})



\section{Nontensors}

We can have a quantity $N^{\mu}_{\nu\rho\cdots}$ with various up and down 
suffixes, which is not a tensor. If it is a tensor, it must transform under a 
change of coordinate system according to the law exemplified by (\ref{ecuacion 
3.6}). With any other law it is a nontensor. A tensor has the property that if 
all the components vanish in a system of coordinates, they vanish in every 
system of coordinates. This may not hold for a nontensor.

For a nontensor we can raise and lower suffixes by the same rules as for a 
tensor. Thus, for example,
\[
    g^{\alpha\nu} {N^{\mu}}_{\nu\rho} = {N^{\mu\alpha}}_{\rho}.
\]
The consistency of these rules is quite independent of the transformation laws 
to a different system of coordinates. Similarly, we can contract a nontensor by 
putting an upper and lower suffix equal.

We may have tensors and nontensors apparing together in the same equation. The 
rules for balancing suffixes apply  equally to tensors and nontensors.

\subsection{The Quotient Theorem}
Suppose that $P_{\lambda\mu\nu}$ is such that $A^\lambda P_{\lambda\mu\nu}$ is 
a tensor \emph{for any vector} $A^\lambda$. Then $P_{\lambda\mu\nu}$ is a 
tensor.

To prove it, write $A^\lambda P_{\lambda\mu\nu} = Q_{\mu\nu}$. We are given 
that this is a tensor; therefore
\[
 Q_{\beta\gamma} = Q_{\mu'\nu'} x^{\mu'}_{,\beta} x^{\nu'}_{,\gamma}.
\]

Thus 
\[
A^\alpha P_{\alpha\beta\gamma} = A^{\lambda'}P_{\lambda'\mu'\nu'}
        x^{\mu'}_{,\beta}x^{\nu'}_{,\gamma}.
\]
Since $A^\lambda$ is a vector, we have from (\ref{ecuacion 3.2}),
\[
 A^{\lambda'} = A^\alpha x^{\lambda'}_{,\alpha'}.
\]
So
\[
A^\alpha P_{\alpha\beta\gamma} = 
A^\alpha 
x^{\lambda'}_{,\alpha}P_{\lambda'\mu'\nu'} x^{\mu'}_{,\beta}x^{\nu'}_{,\gamma}.
\]
This equation must hold for all values of $A^\alpha$, so
\[
P_{\alpha\beta\gamma} = 
P_{\lambda'\mu'\nu'} x^{\lambda'}_{,\alpha} x^{\mu'}_{,\beta}x^{\nu'}_{,\gamma}.
\]
showing that $P_{\alpha\beta\gamma}$ is a tensor.

The theorem also holds is $P_{\alpha\beta\gamma}$ is replaced by a quantity 
with any number of suffixes, and if some of the suffixes are upstairs.


\section{Curved space}
One can easily imagine a curved two-dimensional space as a surface immersed in 
Euclidean three-dimensional space. In the same way, one can have a curved 
four-dimensional space immersed in a flat space of a larger number of 
dimensions. Such a curved space is called a Riemann space. A small region of it 
is approximately flat.

Einstein assumed that physical space is of this nature and thereby laid the 
foundation for his theory of gravitation.

For dealing with curved space one cannot introduce a rectilinear system of 
axes. One has to use curvilinear coordinates, such as those dealt with in 
Section 3. The whole formalism of that section can be applied to curved space, 
because all the equations are local ones which are not disturbed by the 
curvature.

The invariant distance $ds$ between a point $x^\mu$ and a neighboring point 
$x^\mu + dx^\mu$ is given by 
\[
 ds^2 = g_{\mu\nu}dx^\mu dx^\nu
\]
like (\ref{ecuacion 2.1}). $ds$ is real for a timelike interval and imaginary for a spacelike interval.

With a network of curvilinear coordinates the $g_{\mu\nu}$, given as functions 
of the coordinates, fix all the elements of distance; so they fix the metric. 
They determine both the coordinate system and the curvature of the space.
\section{Parallel displacement}

Suppose we have a vector $A^\mu$ located at a point $P$. \textbf{If the space is curved, we cannot give a meaning to a parallel vector at a different point $Q$}, as one can easily see if one thinks of the example of a curved two-dimensional space in a three-dimensional Euclidean space. However, if we take a point $P'$ close to $P$, there is a parallel vector a $P'$, with an uncertainty of the second order, counting the distance from $P$ to $P'$ as the first order. Thus we can give a meaning to displacing the vector $A^\mu$ from $P$ to $P'$ keeping it parallel to itself and keeping the length constant.

We can transfer the vector continuously along a path by this process of parallel displacement. Taking a path from $P$ to $Q$, we end up with a vector at $Q$ which is parallel to the original vector at $P$, \textbf{with respect to this path}. But a different path will give a different result. There is no absolute meaning to a parallel vector at $Q$. If we transport the vector at $P$ by parallel displacement around a closed loop, we shall end up with a vector at $P$ which is usually in a different direction.

We can get equations for the parallel displacement of a vector by supposing our four-dimensional physical space immersed in a flat space of a higher number of dimensions; say $N$. In this $N$-dimensional space we introduce rectilinear coordinates $z^n\,(n=1,\cdots, N)$. Those coordinates do not need to be orthogonal, only rectilinear. Between two neighboring points there is an invariant distance $ds$ given by 
\begin{equation}
 \label{ecuacion 6.1}
 ds^2 = h_{mn} dz^m dz^n ,
\end{equation}
summed for $n,\,m = 1,2,\cdots, N$. The $h_{nm}$ are constants, unlike the $g_{\mu\nu}$. We may use them to lower suffixes in the $N$-dimensional space; thus
\[
 dz_n = h_{mn} dz^m.
\]

Physical space forms a four-dimensional ``surface'' in the flat $N$-dimensional space. Each point $x^\mu$ in the surface determines a definite point $y^n$ in the $N$-dimensional space. Each coordinate $y^n$ is a function of the four $x$'s; say $y^n(x)$. The equations of the surface would be given by eliminating the four $x$'s from the $N$ $y^n(x)$'s. There are $N-4$ such equations.

By differentiating the $y^n(x)$ with respect to the parameters $x^\mu$, we get
\[
 \frac{\partial y^n(x)}{\partial x^\mu} = y^n_{,\mu}.
\]
For two neighboring points in the surface differing by $\delta x^\mu$, we have
\begin{equation}
 \label{ecuacion 6.2}
 \delta y^n = y^n_{,x^\mu} \delta x^\mu
\end{equation}
The squared distance between them is, from (\ref{ecuacion 6.1})
\[
 \delta s^2 = h_{mn}  y^m_{,\mu}  y^n_{,\nu} \delta x^\mu \delta x^\nu.
\]
We may write it 
\[
 \delta s^2 = y^n_{,\mu} y_{n,\nu} \delta x^\mu \delta x^\nu.
\]
Hence
\begin{equation}
 \label{ecuacion 6.3}
 g_{\mu\nu} = y^n_{,\mu} y_{n,\nu}.
\end{equation}

Take a contravariant vector $A^\mu$ in physical space, located at the point x. Its components $A^\mu$ are like the $\delta x^\mu$ of (\ref{ecuacion 6.2}). Thus
\begin{equation}
 \label{ecuacion 6.4}
 A^n = y^n_{,\mu}A^\mu.
\end{equation}

Now, shift the vector $A^n$, keeping it parallel to itself (which means, of course, keeping its components constant), to a neighboring point $x + dx$ in the surface. It will no longer lie in the surface at the new point, on account of the curvature of the surface. But we can project it on to the surface, to get a definite vector lying on the surface.

The projection process consists in splitting the vector into two parts, a tangential part and a normal part, and discarding the normal part. Thus
\begin{equation}
 \label{ecuacion 6.5}
 A^n = A^n_{\mbox{tan}} + A^n_{\mbox{nor}}.
\end{equation}

Now, if $K^\mu$ denotes the components of $A^n_{\mbox{tan}}$ referred to the $x$ coordinate system in the surface, we have, corresponding to (\ref{ecuacion 6.4}),
\begin{equation}
 \label{ecuacion 6.6}
 A^n_{\mbox{tan}} = K^\mu y^n_{,\mu}(x + dx),
\end{equation}
with the coefficients $y^n_{,\mu}$ taken at the new point $x+ dx$.

$A^n_{\mbox{nor}}$ is defined to be orthogonal to every tangential vector at the point $x+dx$, and thus to every vector like the right-hand side of (\ref{ecuacion 6.6}), no matter what the $K^\mu$ are. Thus(\footnote{Here, Dirac assumes that the metric is positive (or negative) which is not the case for the space-time of relativity. Therefore, this mathematical argument supports, at most, an analogy between Einstein's space-time and a Riemann space.})
\[
 A^n_{\mbox{nor}} y_{n,\mu}(x + dx) = 0.
\]
If we now multiply (6.5) by $y_{n,\nu} (x+dx)$ the $A^n_{\mbox{nor}}$ term drops out and we are left with
\[
 \begin{array}{rcl}
  A^n y_{n,\nu}(x + dx ) & = & K^\mu y^n_{,\mu}(x+dx)_{n,\nu}(x + dx)\\
                         & = & K^\mu g_{\mu\nu}(x + dx)
 \end{array}
\]
from (\ref{ecuacion 6.3}). Thus, to the first order in $dx$
\[
 \begin{array}{rcl}
  K_{\nu}(x + dx) & = & A^n\left[ y_{n,\nu}(x) + y_{n,\nu,\sigma} dx^\sigma \right]\\
                  & = & A^\mu y^n_{,\mu} \left[ y_{n,\nu} + y_{n,\nu,\sigma} d x^\sigma \right]\\
                  & = & A_\nu + A^\mu y^n_{,\mu} y_{n,\nu,\sigma} d x^\sigma
 \end{array}
\]
This $K_\nu$ is the result of parallel displacement of $A_\nu$ to the point $x+ dx$. We may put
\[
 K_\nu - A_\nu = d A_\nu,
\]
so $dA_\nu$ denotes the change in $A_\nu$ under parallel displacement. Then we have
\begin{equation}
 \label{ecuacion 6.7}
 d A_\nu = A^\mu y^n_{,\mu} y_{n,\nu,\sigma} d x^\sigma
\end{equation}


\section{Christoffel symbols}
By differentiating (\ref{ecuacion 6.3}) we get (\textbf{omitting the second 
comma with two differentiations})
\begin{equation}
 \label{ecuacion 7.1}
 \begin{array}{rcl}
g_{\mu\nu,\sigma} & = & y^n_{,\mu\sigma} y_{n,\nu} +
                    y^n_{,\mu} y_{n,\nu\sigma} \\
                  & = & y_{n,\mu\sigma} y^n_{,\nu} +
                     y_{n,\nu\sigma} y^n_{,\mu}
\end{array}
\end{equation}

since we can move the suffix $n$ freely up and down, on account of the 
constancy of the $h_{mn}$. Interchanging $\mu$ and $\sigma$ in (\ref{ecuacion 
7.1}) we get 
\begin{equation}
 \label{ecuacion 7.2}
 g_{\sigma\nu,\mu} = y_{n,\mu\sigma} y^n_{,\nu} + y_{n,\nu\mu} y^n_{,\sigma} 
\end{equation}

Interchanging $\nu$ and $\sigma$ in (\ref{ecuacion 7.1})
\begin{equation}
 \label{ecuacion 7.3}
 g_{\mu\sigma,\nu} = y_{n,\mu\nu} y^n_{,\sigma} + y_{n,\sigma\nu} y^n_{,\mu} 
\end{equation}
Now take (\ref{ecuacion 7.1})$+$(\ref{ecuacion 7.3})$-$(\ref{ecuacion 7.2}) and 
divide by $2$. The result is
\begin{equation}
 \label{ecuacion 7.4}
 \frac{1}{2} \left( g_{\mu\nu,\sigma} + g_{\mu\sigma,\nu} - g_{\nu\sigma,\mu}  
\right) = y_{n,\nu\sigma}y^n_{,\mu}.
\end{equation}

Put
\begin{equation}
 \label{ecuacion 7.5}
 \Gamma_{\mu\nu\sigma} = \frac{1}{2}\left(
    g_{\mu\nu,\sigma} + g_{\mu\sigma,\nu} - g_{\nu \sigma,\mu}
 \right)
\end{equation}
It is called a Christoffel symbol of the first kind. It is symmetrical with 
respect to the last two suffixes. It is a nontensor. A simple consequence of 
(\ref{ecuacion 7.5}) is
\begin{equation}
 \label{ecuacion 7.6}
 \Gamma_{\mu\nu\sigma} + \Gamma_{\nu\mu\sigma} = g_{\mu\nu,\sigma}.
\end{equation}
We see now that (\ref{ecuacion 6.7}) can be written as
\begin{equation}
 \label{ecuacion 7.7}
 d A_{\nu} = A^\mu \Gamma_{\mu\nu\sigma} dx^\sigma.
\end{equation}
Al reference to the $N$-dimensional space has now disappeared, as the 
Christoffel symbol involves only the metric $g_{\mu\nu}$ of physical space.

We can infer that the length of a vector is unchanged by parallel displacement. 
We have
\begin{equation}
 \label{ecuacion 7.8}
 \begin{array}{rcl}
 d\left( g^{\mu\nu} A_\mu A_\nu \right) & = & 
    g^{\mu\nu} A_\nu d A_\mu + g^{\mu\nu} A_\mu d A_\nu + A_\mu A_\nu  
    {g^{\mu\nu}}_{,\sigma} dx^\sigma \\
    & = & A^\mu d A_\mu + A^\nu d A_\nu 
      + A_\mu A_\nu {g^{\mu\nu}}_{,\sigma} dx^\sigma \\
    & = & A^\mu A^\nu \Gamma_{\nu\mu\sigma} dx^\sigma 
    + A^\nu A^\mu \Gamma_{\mu\nu\sigma} dx^\sigma
    + A_\mu A_\nu {g^{\mu\nu}}_{,\sigma} dx^\sigma \\
    & = & A^\mu A^\nu g_{\mu\nu,\sigma} dx^\sigma
    + A_\mu A_\nu {g^{\mu\nu}}_{,\sigma} dx^\sigma
 \end{array}
\end{equation}

Now ${g^{\alpha\mu}}_{,\sigma} g_{\mu\nu} + g^{\alpha\mu} g_{\mu\nu,\sigma} = 
\left(g^{\alpha\mu}g_{\mu\nu}\right)_{,\sigma} = g^{\alpha}_{\nu,\sigma} = 0$. 
Multiplying by $g^{\beta\nu}$, we get
\begin{equation}
 \label{ecuacion 7.9}
 {g^{\alpha\beta}}_{,\sigma} = - g^{\alpha\mu} g^{\beta\nu}g_{\mu\nu,\sigma}.
\end{equation}
This is a useful formula giving the derivative of $g^{\alpha\beta}$ in terms 
of the derivative of $g_{\mu\nu}$. It allows us to infer
\[
 A_\alpha A_{\beta} {g^{\alpha\beta}}_{,\sigma} = - A^\mu A^\nu 
g_{\mu\nu,\sigma}
\]
and so the expression (\ref{ecuacion 7.8}) vanishes. Thus the length of a 
vector is constant. In particular, a null vector (i.e. a vector of zero length) 
remains a null vector under parallel displacement.

The constancy of the length of the vector follows also from geometrical 
arguments. When we split up the vector $A^\mu$ into tangential and normal parts 
according to (\ref{ecuacion 6.5}), the normal part is infinitesimal and is 
orthogonal to the tangential part. It follows that, to the first order, the 
length of the whole vector equals that of its tangential part.

The constancy of the length of any vector requires the constancy of the scalar 
product $g^{\mu\nu} A_\mu B_\nu$ of any two vectors $A$ and $B$. This can be 
inferred from the constancy of the length of $A + \lambda B$ for any value of 
the parameter $\lambda$.

It is frequently useful to raise the first suffix of the Christoffel symbol so 
as to form
\[
 \Gamma^\mu_{\nu\sigma} = g^{\mu\lambda}\Gamma_{\lambda\mu\sigma}.
\]
It is then called a Christoffel symbol of the second kind. It is symmetrical 
between its two lower suffixes. As explained in Section 4, this raising is 
quite permissible, even for a nontensor.

The formula (\ref{ecuacion 7.7}) may be written
\begin{equation}
 \label{ecuacion 7.10}
 dA_{\nu} = \Gamma^\mu_{\nu\sigma} A_\mu dx^\sigma
\end{equation}
It is the standard formula referring to covariant components. For a second 
vector $B^\nu$ we have
\[
\begin{array}{rcl}
d(A_\nu B^\nu) & = & 0 \\
A_\nu d B^\nu  & = & - B^\nu d A_\nu 
                 = - B^\nu \Gamma^\mu_{\nu\sigma} A_\mu dx^\sigma \\
               & = & - B^\mu \Gamma^\nu_{\mu\sigma} A_\nu dx^\sigma
\end{array}
\]
This must hold for any $A_\nu$, so we get
\begin{equation}
 \label{ecuacion 7.11}
 d B^\nu = - \Gamma^\nu_{\mu\sigma} B^\mu dx^\sigma.
\end{equation}
This is the standard formula for parallel displacement referring to 
contravariant components.










\section{Geodesics}
Take a point with coordinates $z^\mu$ and suppose it moves along a track; we 
then have a function of some parameter $\tau$. Put $dz^\mu / d\tau = 
u^\mu$.

There is a vector $u^\mu$ at each point of the track. Suppose that as we go 
along the track the vector $u^\mu$ gets shifted by parallel displacement. Then 
the whole track is determined if we are given the initial point and the initial 
value of the vector $u^\mu$. We just have to shift the initial point from 
$z^\mu$ to $z^\mu + u^\mu d\tau$, then shift the vector $u^\mu$ to this new 
point by parallel displacement, then shift the point again in the direction 
fixed by the new  $u^\mu$, and so on. Not only is the track determined, but 
also the parameter $\tau$ along it. A track produced this way is called a 
geodesic.

If the vector $u^\mu$ is a null vector, it always remains a null vector and the 
track is called a null geodesic. If the vector $u^\mu$ is initially timelike 
(i.e., $u^\mu u_\mu > 0$), it is always timelike and we have a timelike 
geodesic. Similarly if $u^\mu$ is initially spacelike ($u^\mu u_\mu < 0$), it 
is always spacelike and we have a spacelike geodesic.

We get the equations of a geodesic by applying (\ref{ecuacion 7.11}) with 
$B^\nu = u^\nu$ and $dx^\sigma = dz^\sigma$. Thus 
\begin{equation}
 \label{ecuacion 8.1}
 \frac{d u^\nu}{d\tau} 
       + \Gamma^\nu_{\mu\sigma} u^\mu \frac{dz^\sigma}{d\tau} = 0
\end{equation}
or
\begin{equation}
 \label{ecuacion 8.2}
 \frac{d^2 z }{d \tau^2} =
      + \Gamma^\nu_{\mu\sigma} \frac{dz^\mu}{d\tau} \frac{dz^\sigma}{d\tau}  
      = 0
\end{equation}

For a timelike geodesic we may multiply the initial $u^\nu$ by a factor so as 
to make its length unity. This merely requires a change in the scale of $\tau$. 
The vector $u^\mu$ now always has the length unity. It is just the velocity 
vector $v^\mu = d z^\mu / ds$, and the parameter $\tau$ has become the proper 
time $s$.

Equation (\ref{ecuacion 8.1}) then becomes
\begin{equation}
 \label{ecuacion 8.3}
 \frac{dv^\mu}{ds} + \Gamma^\mu_{\nu\sigma} u^\nu u^\sigma = 0.
\end{equation}
Equation (\ref{ecuacion 8.2}) becomes 
\begin{equation}
 \label{ecuacion 8.4}
 \frac{d^2 z^\mu}{d s^2} + 
      \Gamma^\mu_{\nu\sigma}\frac{dz^\nu}{ds}\frac{dz^\sigma}{ds} = 0.
\end{equation}

We make the physical assumption that the world line of a particle not acted on 
by any forces, except gravitational, is a timelike geodesic. This replaces 
Newton's first law of motion. Equation (\ref{ecuacion 8.4}) fixes the 
acceleration and provides the equations of motion.

We also make the assumption that the path of a ray of light is a null geodesic. 
It is fixed by equation (\ref{ecuacion 8.2}) referring to some parameter along 
the path. The proper time $s$ cannot now be used because $ds$ vanishes.


\section{The stationary property of geodesics}
A gepdesic that is not a null geodesic has the property that $\int ds$, taken 
along a sectio of the track with the end points $P$ and $Q$, is stationary if 
one makes a small variation of the tack keeping the end points fixed.

Let us suppose that each point of the track, with the coordinates $z^\mu$ is 
shifted so that its coordinates become $z^\mu + \delta z^\mu$. If $dx^\mu$ 
denotes an element along the track,
\[
 ds^2 = g_{ \mu \nu } dx^\mu dx^\nu
\]
Thus
\[
\begin{array}{rcl}
 2 ds \delta ds & = & d x ^\mu dx^\nu \delta g_{\mu\nu} 
    + g_{\mu\nu} dx^\mu \delta dx^\nu
    + g_{\mu\nu} dx^\nu \delta dx^\mu  \\
    & = & d x ^\mu dx^\nu \delta g_{\mu\nu}  
    + 2 g_{\mu\lambda}dx^\mu \delta dx^\lambda .
\end{array}
\]
Now
\[
\delta dx^\lambda = d (\delta x^\lambda).
\]
Thus, with the help of $dx^\mu = v^\mu ds$,
\[
    \delta(ds) = \left(
    \frac{1}{2} g_{\mu\nu,\lambda}v^\mu v^\nu \delta x^\lambda  
    + g_{\mu\lambda}v^\mu \frac{d\delta x^\lambda}{ds}
    \right) ds
\]
Hence
\[
\delta \int ds = \int{\delta (ds)}
 = \int \left[
    \frac{1}{2} g_{\mu\nu,\lambda}v^\mu v^\nu \delta x^\lambda  
    + g_{\mu\lambda}v^\mu \frac{d\delta x^\lambda}{ds}
    \right] ds
\]
By partial integration (of the second term), using the condition that $\delta 
x^\lambda = 0$ at the end points $P$ and $Q$, we get
\begin{equation}
 \label{ecuacion 9.1}
 \delta \int ds 
 = \int \left[
    \frac{1}{2} g_{\mu\nu,\lambda}v^\mu v^\nu 
    -   \frac{d}{ds} \left(g_{\mu\lambda}v^\mu\right)
    \right] \delta x^\lambda   ds
\end{equation}
The condition for this to vanish with arbitrary $\delta x^\lambda$ is
\begin{equation}
 \label{ecuacion 9.2}
 \frac{d}{ds}\left(g_{\mu\lambda} v^\mu\right) 
 - \frac{1}{2}g_{\mu\nu,\lambda} = 0
\end{equation}
Now
\[
\begin{array}{rcl}
 \frac{d}{ds}\left(g_{\mu\lambda} v^\mu\right) & = &  
  g_{\mu\lambda}\frac{dv^\mu}{ds} + g_{\mu\lambda,\nu}v^\mu v^\nu \\
  & = & g_{\mu\lambda}\frac{dv^\mu}{ds} + 
  \frac{1}{2}\left( g_{\lambda\mu,\nu} + g_{\lambda\nu,\mu} \right) v^\mu v^\nu 
.
\end{array}
\]
Thus the condition (\ref{ecuacion 9.2}) becomes
\[
 g_{\mu\lambda}\frac{dv^\mu}{ds} + \Gamma_{\lambda\mu\nu}v^\mu v^\nu = 0.
\]
Multiplying this by $g^{\lambda\sigma}$, it becomes
\[
 \frac{dv^\sigma}{ds} + \Gamma^\sigma_{\mu\nu}v^\mu v^\nu = 0,
\]
which is just the condition (\ref{ecuacion 8.3}) for the geodesic.

This work shows that for a geodesic, (\ref{ecuacion 9.1}) vanishes and $\int 
ds$ is stationary. Conversely, if we assume that $\int ds$ is stationary, we 
can infer that the track is a geodesic. Thus we may use the stationary 
condition as the definition of a geodesic, except in the case of a null 
geodesic.

\section{Covariant differentiation}
Let $S$ be a scalar field. Its derivative $S,y$ is a covariant vector, as we 
saw in Section 3. Now let $A_\mu$ be a vector field. Is its derivative 
$A_{\mu,\nu}$ a tensor?

We must examine how $A_{\nu,\nu}$ transforms under a change of coordinate 
system. With the notation in Section 3, $A_\mu$ transforms to
\[
 A_{\mu'} = A_\rho x^\rho _{,\mu'}
\]
like equation (\ref{ecuacion 3.5}), and hence
\[
 \begin{array}{rcl}
  A_{\mu',\nu'} & = & \left( A_\rho x^\rho _{,\mu'} \right)_{,\nu'}\\
                & = & A_{\rho,\sigma} x^\sigma_{,\nu'} x^\rho_{,\mu'}
                + A_\rho x^\rho_{,\mu'\nu'} .
 \end{array}
\]
The last term should not be here if we were to have the correct transformation 
law for a tensor. Thus $A_{\mu,\nu}$ is not a tensor.

We can, however, modify the process of differentiation so as to get a tensor. 
Let us take the vector $A_\mu$ at the point $x$ and shift it to $x+dx$ by 
parallel displacement. It is still a vector. We may subtract it from the vector 
$A_\mu$ at $x+dx$ and the difference will be a vector. It is, to the first order
\[
 A_{\mu}(x + dx) - \left[ A_{\mu}(x) + \Gamma^\alpha_{\mu\nu}A_\alpha 
dx^\nu\right] = \left(A_{\mu,\nu} - \Gamma^\alpha_{\mu\nu} A_\alpha 
\right)dx^\nu .
\]
This quantity is a vector, for any vector $dx^\nu$; hence, by the quotient 
theorem of Section 4, the coefficient
\[
 A_{\mu\nu} - \Gamma^\alpha_{\mu\nu} A_\alpha
\]
is a tensor. One can easily verify directly that it transforms correctly under 
a change of coordinate system.

It is called the covariant derivative of $A_\mu$ and written
\begin{equation}
 \label{ecuacion 10.1}
 A_{\mu:\nu} = A_{\mu,\nu} - \Gamma^\alpha_{\mu\nu}A_\alpha. 
\end{equation}

The sign $:$ before a lower suffix will always denote a covariant derivative, 
just as the comma denotes an ordinary derivative.

Let $B_{\nu}$ be a second vector. We define the outer product $A_{\mu}B_{\nu}$ 
to have the covariant derivative(\footnote{Notice that the definition of the  
covariant derivative of a second rank tensor is not motivated because the 
notion of \emph{parallel displacement of tensor} doesn't make sense. })

\begin{equation}
 \label{ecuacion 10.2}
 \left(A_{\mu} B_{\nu} \right)_{\sigma} =
     A_{\mu:\sigma}B_{\nu} + A_{\mu} B_{\nu:\sigma}
\end{equation}

Evidently it is a tensor with three suffixes. It has the value
\[
  \left(A_{\mu} B_{\nu} \right)_{\sigma} =
  \left(A_{\mu,\sigma} - \Gamma^{\alpha}_{\mu\sigma} A_{\alpha}\right)B_{\nu} 
+ A_{\mu} \left(B_{\nu,\sigma} - \Gamma^{\alpha}_{\nu\sigma}B_{\alpha}\right)
\]

Let $T_{\mu\nu}$ be a tensor with two suffixes. It is expressible as a sum of 
terms like $A_{\mu}B_{\nu}$, so, its covariant derivative is
\begin{equation}
 \label{ecuacion 10.3}
 T_{\mu\nu:\sigma} = T_{\mu\nu,\sigma}
 - \Gamma^{\alpha}_{\mu\sigma} T_{\alpha\nu}
 - \Gamma^{\alpha}_{\nu\sigma} T_{\mu\alpha}.
\end{equation}


The rule can be extended to the covariant derivative of a tensor 
$Y_{\mu\nu\cdots}$ with any number of suffixes downstairs
\begin{equation}
 \label{ecuacion 10.4}
 Y_{\mu\nu\cdots:\sigma} = Y_{\mu\nu\cdots,\sigma} - \mbox{ a }
 \Gamma \mbox{ term for each suffix}.
\end{equation}
In each of these $\Gamma$ terms we must make the suffixes balance, which is 
sufficient to fix how the suffixes go.

The case of a scalar is included in the general formula (\ref{ecuacion 10.4}) 
with the number of suffixes in $Y$ zero.
\begin{equation}
 \label{ecuacion 10.5}
 Y_{:\sigma} = Y_{,\sigma} .
\end{equation}




\section{The curvature tensor}
With the product law (\ref{ecuacion 10.8}) we see that covariant 
differentiation is very similar to ordinary differentiation. But there is an 
important property of ordinary differentiation, that if we perform two 
differentiations in succession their order does not matter, which does not, in 
general, hold for covariant differentiation.

Let us first consider a scalar field $S$. We have from the formula 
(\ref{ecuacion 10.1}),
\begin{equation}
 \label{ecuacion 11.1}
 \begin{array}{rcl}
 S_{:\mu:\nu} & = & S_{:\mu,\nu} - \Gamma^{\alpha}_{\mu\nu} S_{\alpha} \\
      & = & S_{,\mu\nu} - \Gamma^{\alpha}_{\mu\nu} S_{,\alpha}.
 \end{array}
\end{equation}
\textbf{This is symmetrical between $\mu$ and $\nu$, so in this 
case the order of the covariant differentiation does not matter.}

Now let us take a vector $A_{\nu}$ and apply two covariant differentiations to 
it. From the formula (\ref{ecuacion 10.3}) with $A_{\nu:\rho}$ for 
$T_{\nu\rho}$ we get
\[
 \begin{array}{rcl}
  A_{\nu:\rho:\sigma} & = & A_{\nu:\rho,\sigma} 
  - \Gamma^{\alpha}_{\nu\sigma} A_{\alpha:\rho}
  - \Gamma^{\alpha}_{\rho\sigma} A_{\nu:\alpha} \\
  & = &  \left(A_{\nu,\rho} - 
              \Gamma^{\alpha}_{\nu\rho} A_{\alpha} \right)_{,\sigma}
  - \Gamma^{\alpha}_{\nu\sigma} \left( 
      A_{\alpha,\rho} - \Gamma^{\beta}_{\alpha\rho} A_{\beta}
  \right)
  - \Gamma^{\alpha}_{\rho\sigma}  \left( 
      A_{\nu,\alpha} - \Gamma^{\beta}_{\nu\alpha} A_{\beta}
  \right)\\
  & = & A_{\nu,\rho\sigma} 
      - \Gamma^{\alpha}_{\nu\rho} A_{\alpha,\sigma}
      - \Gamma^{\alpha}_{\nu\sigma} A_{\alpha,\rho}
      - \Gamma^{\alpha}_{\rho\sigma} A_{\nu,\alpha}\\
      & & 
      -A_{\beta}\left(
        \Gamma^{\beta}_{\nu\rho,\sigma}
      - \Gamma^{\alpha}_{\nu\sigma}\Gamma^{\beta}_{\alpha\rho}
      - \Gamma^{\alpha}_{\rho\sigma}\Gamma^{\beta}_{\nu\alpha}
      \right)
 \end{array}
\]
Interchange $\rho$ and $\sigma$ here and subtract from the previous expression. 
The result is
\begin{equation}
 \label{ecuacion 11.2}
 A_{\nu:\rho:\sigma} - A_{\nu:\sigma:\rho} =
 A_{\beta}R^{\beta}_{\nu\rho\sigma}
\end{equation}
where
\begin{equation}
 \label{ecuacion 11.3}
 R^{\beta}_{\nu\rho\sigma} = 
 \Gamma^{\beta}_{\nu\sigma,\rho} - \Gamma^{\beta}_{\nu\rho,\sigma}
 + \Gamma^{\alpha}_{\nu\sigma}\Gamma^{\beta}_{\alpha\rho}
 - \Gamma^{\alpha}_{\nu\rho}\Gamma^{\beta}_{\alpha\sigma}
\end{equation}

The left hand of (\ref{ecuacion 11.2}) is a tensor. It follows that the 
right-hand side of (\ref{ecuacion 11.2}) is a tensor. This holds for any vector 
$A_{\beta}$; therefore, by the quotient theorem in Section 4, 
$R^{\beta}_{\nu\rho\sigma}$ is a tensor. It is called the Riemann-Christoffel 
tensor or the curvature tensor.

It has the obvious property 
\begin{equation}
 \label{ecuacion 11.4}
 R^{\beta}_{\nu\rho\sigma} = - R^{\beta}_{\nu\sigma\rho}
\end{equation}
Also we easily see from (\ref{ecuacion 11.3}) that
\begin{equation}
 \label{ecuacion 11.5}
   R^{\beta}_{\nu\rho\sigma} 
 + R^{\beta}_{\rho\sigma\nu} 
 + R^{\beta}_{\sigma\nu\rho} = 0
\end{equation}

Let us lower the suffix $\beta$ and put it as the first suffix. We get
\[
 R_{\mu\nu\sigma\rho} = g_{\mu\beta} R^{\beta}_{\nu\sigma\rho} =
 g_{\mu\beta} \Gamma^{\beta}_{\nu\sigma,\rho} 
 + \Gamma^{\alpha}_{\nu\alpha} \Gamma_{\mu\alpha\rho} - \langle \rho\sigma 
\rangle,
\]
where the symbol $\langle \rho \sigma \rangle$ is used to denote the preceding 
terms with $\rho$ and $\sigma$ interchanged. Thus
\[
 \begin{array}{rcl}
  R_{\mu\nu\sigma\rho} & = & R_{\mu\nu\sigma,\rho} - 
g_{\mu\beta,\rho}\Gamma^{\beta}_{\nu\sigma} + 
\Gamma_{\mu\beta\rho}\Gamma^{\beta}_{\nu\sigma} - \langle \rho\sigma \rangle \\
& = & \Gamma_{\mu\nu\sigma,\rho} - 
\Gamma_{\beta\mu\rho}\Gamma^{\beta}_{\nu\sigma} - \langle \rho\sigma \rangle
 \end{array}
\]
from (\ref{ecuacion 7.6}). So from (\ref{ecuacion 7.5})
\begin{equation}
 \label{ecuacion 11.6}
 R_{\mu\nu\sigma\rho} = \frac{1}{2}
 \left(
       g_{\mu\sigma,\nu\rho} 
     - g_{\nu\sigma,\mu\rho} 
     - g_{\mu\rho,\nu\sigma}
     - g_{\nu\rho,\nu\sigma}
 \right)
 + \Gamma_{\beta\mu\sigma}\Gamma^{\beta}_{\nu\rho}
 - \Gamma_{\beta\mu\rho}\Gamma^{\beta}_{\nu\sigma}
\end{equation}

Some further symmetries now show up; namely,
\begin{equation}
 \label{ecuacion 11.7}
 R_{\mu\nu\rho\sigma} = - R_{\nu\mu\rho\sigma}
\end{equation}
and
\begin{equation}
 \label{ecuacion 11.8}
 R_{\mu\nu\rho\sigma} = R_{\rho\sigma\mu\nu} = R_{\sigma\rho\nu\mu}
\end{equation}
The result of all these symmetries is that, of the 256 components of 
$R_{\mu\nu\rho\sigma}$, only 20 are independent.



\section{The condition for flat Space}
If space is flat, we may choose a system of coordinates that is rectilinear and then the $g_{\mu\nu}$ are constant. The 
tensor $R_{\mu\nu\rho\sigma}$ then vanishes.

Conversely, if $R_{\mu\nu\rho\sigma}$ vanishes, one can prove that the space is flat. Let us take a vector $A_{\mu}$ 
situated at the point $x$ and shift it by parallel displacement to the point $x+dx$. Then shift it by parallel 
displacement to the point $x+dx+\delta x$. If $R_{\mu\nu\rho\sigma}$ vanishes, he result must be the same as if we had 
shifted it first from $x$ to $x+\delta x$, then to $x+\delta x+ dx$. Thus we can shift the vector to a distant point and 
the result we get is independent of the path to the distant point. Therefore, if we shift the original vector $A_{\mu}$ 
at $x$ to all points by parallel displacement, we get a vector field that satisfies $A_{\mu:\nu}=0$, or
\begin{equation}
 \label{ecuacion 12.1}
 A_{\mu,\nu} = \Gamma^{\alpha}_{\mu\nu}.
\end{equation}

Can such a vector field be the gradient of a scalar? Let us put $A_{\mu} = 
S_{\mu}$ in (\ref{ecuacion 12.1}). We get 
\begin{equation}
 \label{ecuacion 12.2}
 S_{,\mu\nu} = \Gamma^{\alpha}_{\mu\nu} S_{\alpha}
\end{equation}
By virtue of the symmetry of $\Gamma^{\alpha}_{\mu\nu}$ in the lower suffixes, we have the same value for $S_{,\mu\nu}$ 
as $S_{,\nu\mu}$ and the equations (\ref{ecuacion 12.2}) are integrable.

Let us take four independent scalars satisfying (\ref{ecuacion 12.2}) and let them to be the coordinates $x^{\alpha'}$ 
of a new system of coordinates. Then
\[
 x^{\alpha'}_{,\mu\nu} = \Gamma^{\sigma}_{\mu\nu} x^{\alpha'}_{,\sigma}.
\]

According the transformation law (\ref{ecuacion 3.7}),
\[
g_{\mu\lambda} = g_{\alpha'\beta'}x^{\alpha'}_{,\mu}x^{\beta'}_{,\lambda}.
\]
Differentiating this equation with respect to $x^{\nu}$, we get
\[
\begin{array}{rcl}
 g_{\mu\lambda,\nu} - g_{\alpha'\beta',\nu}x^{\alpha'}_{,\mu}x^{\beta'}_{,\lambda} & = & 
    g_{\alpha'\beta'}\left(x^{\alpha'}_{,\mu\nu}x^{\beta'}_{,\lambda} 
    + x^{\alpha'}_{,\mu}x^{\beta'}_{,\lambda\nu} \right) \\
    & = & g_{\alpha'\beta'}\left(
    \Gamma^{\sigma}_{\mu\nu} x^{\alpha'}_{,\sigma} x^{\beta'}_{,\lambda} 
    + x^{\alpha'}_{,\mu} \Gamma^{\sigma}_{\lambda\nu} x^{\beta'}_{,\sigma}
    \right)\\
    & = & g_{\sigma\lambda} \Gamma^{\sigma}_{\mu\nu} + g_{\mu\sigma} \Gamma^{\sigma}_{\lambda\nu}\\
    & = & \Gamma_{\lambda\mu\nu} + \Gamma_{\mu\lambda\nu} = g_{\mu\lambda,\nu}
\end{array}
\]
from (\ref{ecuacion 7.6}). Thus
\[
g_{\alpha'\beta',\nu}x^{\alpha'}_{,\mu}x^{\beta'}_{,\lambda} = 0.
\]
It follows that $g_{\alpha'\beta',\nu}=0$. Referred to the new system of coordinates, the fundamental tensor is 
constant. Thus we have flat space referred to rectilinear coordinates.

\section{The Bianci relations}
To deal with the second covariant derivative of a tensor, take first the case in which the tensor is the outer product 
of two vectors $A_{\mu} B_{\tau}$. We have
\[
\begin{array}{rcl}
 \left( A_{\mu} B_{\tau}\right)_{:\rho:\sigma} & = & \left( A_{\mu:\rho} B_{\tau} + A_{\mu} B_{\tau:\rho} \right)\\
      & = & A_{\mu:\rho:\sigma} B_{\tau} + A_{\mu:\rho} B_{\tau:\sigma} + A_{\mu:\sigma} B_{\tau:\rho} 
      + A_{\mu} B_{\tau:\rho:\sigma}.
\end{array}
\]
Now interchange $\rho$ and $\sigma$ and subtract. We get from (\ref{ecuacion 11.2})
\[
\left( A_{\mu} B_{\tau}\right)_{:\rho:\sigma} - \left( A_{\mu} B_{\tau}\right)_{:\sigma:\rho} = 
   A_{\alpha}R^{\alpha}_{\mu\rho\alpha} B_{\tau} + A_{\mu}R^{\alpha}_{\tau\rho\sigma} B_{\alpha}.
\]
A general tensor $T_{\mu\tau}$ is expressible as a sum of terms like $A_{\mu} B_{\tau}$, so it must satisfy
\begin{equation}
 \label{ecuacion 13.1}
 T_{\mu\tau:\rho:\sigma} - T_{\mu\tau:\sigma:\rho} =
 T_{\alpha\tau}R^{\alpha}_{\mu\rho\sigma} - T_{\mu\alpha}R^{\alpha}_{\tau\rho\sigma}.
\end{equation}

Now take $T_{\mu\tau}$ to be the covariant derivative of a vector $A_{\mu:\tau}$. We get
\[
A_{\mu:\tau:\rho:\sigma} - A_{\mu:\tau:\sigma:\rho}
= A_{\alpha:\tau} R^{\alpha}_{\mu\rho\sigma} + A_{\mu:\alpha} R^{\alpha}_{\tau\rho\sigma}.
\]
In this formula make cyclic permutations of $\tau$, $\rho$, $\sigma$ and add the three equations so obtained. The left 
hand side gives
\begin{equation}
 \label{ecuacion 13.2}
    \begin{array}{cl}
          & A_{\mu:\tau:\rho:\sigma} - A_{\mu:\tau:\sigma:\rho} + \mbox{ cyc perm} \\
        = & \left( A_{\alpha} R^{\alpha}_{\mu\rho\sigma} \right)_{:\tau} + \mbox{ cyc perm} \\
        = & A_{\alpha:\tau} R^{\alpha}_{\mu\rho\sigma} + A_{\mu:\alpha} R^{\alpha}_{\tau\rho\sigma}
        + \mbox{ cyc perm}.
    \end{array}
\end{equation}
The right-hand side gives 
\begin{equation}
 \label{ecuacion 13.3}
    A_{\alpha:\tau} R^{\alpha}_{\mu\rho\sigma} + \mbox{ cyc perm},
\end{equation}
as the remaining terms cancel from (\ref{ecuacion 11.5}). The first term of (\ref{ecuacion 13.2}) cancels with 
(\ref{ecuacion 13.3}) and we are left with
\[
   A_{\alpha} R^{\alpha}_{\mu\rho\sigma:\tau} + \mbox{ cyc perm} = 0.
\]
The factor $A_{\alpha}$ occurs throughout this equation and may be canceled out. We are left with
\begin{equation}
 \label{ecuacion 13.4}
 R^{\alpha}_{\mu\rho\sigma:\tau} + 
 R^{\alpha}_{\mu\sigma\tau:\rho} + 
 R^{\alpha}_{\mu\tau\rho:\sigma} = 0
\end{equation}

The curvature tensor satisfies these differential equations as well as other symmetry relations in Section 11. They are 
known as the Bianci relations.

\section{The Ricci tensor}
Let us contract two of the suffixes in $R_{\mu\nu\rho\sigma}$. If we take two with respect to which it is 
antisymmetrical, we get zero, of course. If we take any other two we get the same result, apart from the sign because 
of the symmetries (\ref{ecuacion 11.4}), (\ref{ecuacion 11.7}), and (\ref{ecuacion 11.8}). Let us take the first and 
last and put
\[
R^{\alpha}_{\nu\rho\alpha} = R_{\nu\rho}.
\]
It is called the Ricci tensor.

By multiplying (\ref{ecuacion 11.8}) by $g^{\mu\nu}$ we get 
\begin{equation}
 \label{ecuacion 14.1}
 R_{\nu\rho} = R_{\rho\nu}.
\end{equation}
The Ricci tensor is symmetrical.

We may contract again and form
\[
g^{\nu\rho}R_{\nu\rho} = R^{\nu}_{\nu} = R,
\]
say. This $R$ is a scalar and is called the scalar curvature or total curvature. It is defined in such way that it is 
positive for the surface of a sphere in three dimensions, as one can check by a straightforward calculation.

The Bianci relation (\ref{ecuacion 13.4}) involves five suffixes. Let us contract it twice and get a relation with one 
nondummy suffix. Put $\tau = \alpha$ and multiply by $g^{\mu\rho}$. The result is
\[
g^{\mu\rho}\left(
R^{\alpha}_{\mu\rho\sigma:\alpha} +
R^{\alpha}_{\mu\sigma\alpha:\rho} +
R^{\alpha}_{\mu\alpha\rho:\sigma}
\right) = 0
\]
or
\begin{equation}
 \label{ecuacion 14.2}
 \left( g^{\mu\rho} R^{\alpha}_{\mu\rho\sigma}\right)_{:\alpha} +
 \left( g^{\mu\rho} R^{\alpha}_{\mu\sigma\alpha}\right)_{:\rho} +
 \left( g^{\mu\rho} R^{\alpha}_{\mu\alpha\rho}\right)_{:\sigma}
 = 0 .
\end{equation}
Now
\[
 g^{\mu\rho} R^{\alpha}_{\mu\rho\sigma} =
  g^{\mu\rho}g^{\alpha\beta}R_{\beta\mu\rho\sigma} =
  g^{\mu\rho}g^{\alpha\beta}R_{\mu\beta\sigma\rho} =
  g^{\alpha\beta}R_{\beta\sigma} = R^{\alpha}_{\sigma}.
\]
One can write $R^{\alpha}_{\sigma}$ with the suffixes one over the other on account of $R_{\alpha\sigma}$ being 
symmetrical. Equation (\ref{ecuacion 14.2}) now becomes
\[
 R^{\alpha}_{\sigma:\alpha} + \left( g^{\mu\rho} R_{\mu\sigma} \right)_{:\rho} - R_{:\sigma} = 0
\]
or
\[
 2 R^{\alpha}_{\sigma:\alpha} - R_{:\sigma} = 0
\]
which is the Bianci relation for the Ricci tensor. If we raise the suffix $\sigma$, we get
\begin{equation}
 \label{ecuacion 14.3}
 \left( R^{\sigma \alpha} - \frac{1}{2} g^{\sigma \alpha} R\right)_{:\alpha} = 0
\end{equation}

The explicit expression for the Ricci tensor is, from (\ref{ecuacion 11.3})
\begin{equation}
 \label{ecuacion 14.4}
 R_{\mu\nu} = \Gamma^{\alpha}_{\mu\alpha,\nu} - \Gamma^{\alpha}_{\mu\nu,\alpha}
 + \Gamma^{\alpha}_{\mu\beta}\Gamma^{\beta}_{\nu\alpha} - \Gamma^{\alpha}_{\mu\nu}\Gamma^{\beta}_{\alpha\beta}
\end{equation}
The first term here does not appear to be symmetrical in $\mu$ and $\nu$, although the other three terms evidently are. 
To establish that the first term really is symmetrical we need a little calculation.

To differentiate the determinant $g$ we must differentiate each element $g_{\lambda\mu}$ and then multiply for the 
cofactor $g g^{\lambda\mu}$. Thus
\begin{equation}
 \label{ecuacion 14.5}
 g_{,\nu} = g g^{\lambda\mu} g_{\lambda\mu,\nu}.
\end{equation}
Hence 
\begin{equation}
 \label{ecuacion 14.6}
 \begin{array}{rcl}
    \Gamma^{\mu}_{\lambda\mu} & = & g^{\lambda\mu}\Gamma_{\lambda\nu\mu} =
    \frac{1}{2}\left(
        g_{\lambda\nu,\mu} + g_{\lambda\mu,\nu} - g_{\mu\nu,\lambda}
    \right)\\
    & = & \frac{1}{2} g^{\lambda\mu} g_{\lambda\mu,\nu} = \frac{1}{2} g^{-1}g_{\nu}
    = \frac{1}{2} \left( log(g) \right)_{,\nu}.
 \end{array}
\end{equation}
This makes it evident that the first term of (\ref{ecuacion 14.4}) is symmetrical.



\section{Einstein's Law of gravitation}
Up to the present our workhas all been pure mathematics (apart from the physical assumption that the track of a 
particle is a geodesic). It was done mainly in the last entury and applies to curved spaces in any number of 
dimensions. The only place where the number of dimensions would appear in the formalism is in the equation 
\[g^{\mu}_{\nu} = \mbox{ number of dimensions}\]

    Einstein made the assumption that in empty spaces
\begin{equation}
    \label{ecuacion 15.1}
    R_{\mu\nu} = 0.
\end{equation}
It constitutes his law of gravitation. ``Empty'' here means that htere is no matter present and no physical fields 
except the gravitational field. The gravitational field does not disturb the emptyness. Other fields do.(\footnote{If 
the gravitational field does not disturb the \emph{emptyness} of space, then the gravitational field is nothing, which 
begs the questions: ``What is a quantum of \emph{nothing}>'', ``How \emph{nothing} can be the substrate of 
gravitational waves?'', etcetera}). The conditions for empty space hold in a good approximation for the space between 
ghe planets in the solar system, and equation (15.1) applies there. 

Flat space obviously satisfies (\ref{ecuacion 15.1}). The geodesics are then straight lines and so particles move along 
straight lines. 


\section{The Newtonian approximation}
Let us consider a static gravitational field and refer it to a static coordinate system. The $g_{\mu\nu}$ are then 
constant in time, $g_{\mu\nu,0} = 0$. Further, we must have 
\[
 g_{m0} = 0, \, \left( m=1,\cdots,3 \right) .
\]
This leads to
\[
 g^{m0} = 0, \, g^{00} = (g_{00})^{-1},
\]
and $g^{mn}$ is the reciprocal matrix to $g_{mn}$. Roman suffixes like $m$ and $n$ always take on the values 1, 2, 3. 
We find that $\Gamma_{m0n} = 0$ and hence also $\Gamma^{m}_{0n}=0$.

Let us take a particle that is moving slowly, compared with the velocity of light. Then $v^m$ is a small quantity, of 
the first order. With neglect of second order quantities,
\begin{equation}
 \label{ecuacion 16.1}
 g_{00} {v^{0}}^2 = 1.
\end{equation}

\section{The gravitational red shift}
Let us take again a static gravitational field and consider an atom at rest emitting monochromatic radiation. The 
wavelength of the light will correspond to a definite $\Delta s$. Since the atom is at rest we have, for a static 
system of coordinates such as we used in Section 16,
\[
    {\Delta s}^2 = g_{00}{\Delta x^0}^2,
\]
where $\Delta x^0$ is the period, that is, the time between successive crests referred to our static coordinate system.

If the light travels to another place, $Delta x^0$ will remain constant. This $Delta x^0$ will not be the same as the 
period of the same spectral line emitted by a local atom, which would be $\Delta s$ again. The period is thus dependent 
on the gravitational potential $g_{00}$ at the place where the light was emitted:
\[
    \Delta x^0 :: {g_{00}}^{-1/2}.
\]
The spectral line will be shifted by this factor ${g_{00}}^{-1/2}$

If we use the Newtonian approximation (\ref{ecuacion 16.6}), we have 
\[
    \Delta x^0 :: 1 - V.
\]

$V$ will be negative at a place with a strong gravitational field, such as the surface of the sun, so, light emitted 
there will be red-shifted when compared with the corresponding light emitted on earth.

\section{The Schwarzschild solution}
The Einstein equations for empty space are nonlinear and are therefore very complicated, and it is difficult to get 
accurate solutions of them. There is, however, a special case which can be solved without too much trouble; namely the 
static spherically symmetric field produced by a spherically symmetric body at rest.

The static condition means that, with a static coordinate system, the $g_{\mu\nu}$ are independent of the time $x^0$ or 
$t$ and also $g_{0m}$(\footnote{Making the assumption that all intervals of the form $(\delta x^0,0,0,0)$ are temporal 
and all interval of the form $(0,\delta x^1, \delta x^2, \delta x^3)$ are spatial.}). The spatial coordinates may be 
taken to be spherical polar coordinates $x^1 = r,\,x^2=\theta,\,x^3=\phi$. The most general form of $ds^2$ compatible 
with spherical symmetry is
\[
 ds^2 = U dt^2 - V dr^2 -W r^2(d\theta^2 + \sin^2\theta d\phi^2),
\]
where $U$, $V$, and $W$ are functions of $r$ only. We may replace $r$ by any function of $r$ without disturbing the 
spherical symmetry. We use this freedom to simplify things as much as possible, and the most convenient arrangement is 
to have $W=1$. The expression for $ds^2$ may then be written as 
\begin{equation}
 \label{ecuacion 18.1}
 ds^2 = e^{2\nu} dt^2 - e^{2\lambda} dr^2 - r^2 d\theta^2 - r^2 \sin^2 \theta d\phi^2,
\end{equation}
with $\nu$ and $\lambda$ functions of $r$ only. They must be chosen to satisfy the Einstein equations. We can read off 
the values of the $g_{\mu\nu}$ from (\ref{ecuacion 18.1}), namely,
\[
\begin{array}{cccc}
 g_00 = e^{2\nu}, & g_{11} = e^{-2\lambda}, & g_{22} = -r^2, & g_{33} = - r^2 \sin^2\theta,
\end{array}
\]
and 
\[
g_{\mu\nu} = 0 \mbox{ for } \mu \ne \nu .
\]
We find
\[
   \begin{array}{cccc}
   g^{00} = e^{-2\nu}, & g_{11} = e^{-2\lambda}, & g^{22} = -r^2, &  g^{33} = -r^{-2} \sin^{-2} 
\theta,
   \end{array}
\]
and
\[
 g^{\mu\nu} = 0 \mbox{ for } \mu \ne \nu .
\]

It is now necessary to calculate all the Christoffel symbols $\Gamma^\sigma_{\mu\nu}$. Many of them vanish. The ones 
that do not are(\footnote{With primes denoting differentiation with respect to $r$.}):
\[
 \begin{array}{ll}
   \Gamma^1_{00} = \nu' e^{2\nu - 2 \lambda}       & \Gamma^0_{10} =  \nu' \\
   \Gamma^1_{11} = \lambda'                        & \Gamma^2_{12} = \Gamma^3_{13} = r^{-1}\\
   \Gamma^1_{22} = - r e^{-2\lambda}               & \Gamma^2_{23} = \cot \theta \\
   \Gamma^1_{33} = - r \sin^2 \theta e^{-2\lambda} & \Gamma^2_{33} = - \sin \theta \cos \theta
 \end{array}
\]
These expressions are to be substituted in (\ref{ecuacion 14.4}). The results are
\begin{equation}
 \label{ecuacion 18.2}
 R_{00} = \left( - \nu'' + \lambda' \nu' - {\nu'}^2 - \frac{2 \nu'}{r} \right) e^{2 \nu - 2 \lambda},
\end{equation}
\begin{equation}
 \label{ecuacion 18.3}
 R_{11} = \nu'' - \lambda' \nu' + {\nu'}^2 - \frac{2 \lambda'}{r},
\end{equation}
\begin{equation}
 \label{ecuacion 18.4}
 R_{22} = (1 + r \nu' - r \lambda' ) e^{-2\lambda} -1,
\end{equation}
\[
 R_{33} = R_{22} \sin^2 \theta,
\]
with the other components of $R_{\mu\nu}$ vanishing.

Einstein's law of gravitation requires these expressions to vanish. The vanishing of (\ref{ecuacion 18.2}) and 
(\ref{ecuacion 18.3}) leads to
\[
 \lambda' + \nu' = 0. 
\]
For large values of $r$ the space must approximate to being flat, so that $\lambda$ and $\nu$ both tend to zero as 
$r\rightarrow \infty$. It follows that
\[
 \lambda + \nu = 0.
\]
The vanishing of (\ref{ecuacion 18.4}) now gives
\[
 (1 + 2 r \nu') e^{2\nu} = 1
\]
or
\[
 \left( r e^{2\nu}\right)' = 1.
\]
Thus 
\[
 r e^{2\nu} = r - 2m,
\]
where $m$ is a constant of integration. This also makes (\ref{ecuacion 18.2}) and (\ref{ecuacion 18.3}) vanish. We now 
get
\begin{equation}
 \label{ecuacion 18.5}
 g_{00} = 1 - \frac{2 m}{r}
\end{equation}

The Newtonian approximation must hold for large values of $r$. Comparing (\ref{ecuacion 18.5}) with \ref{ecuacion 
16.6}), we see that the constant of integration that has appeared in (\ref{ecuacion 18.5}) is just the mass of the 
central body which is producing the gravitational field.

Te complete solution is
\begin{equation}
\label{ecuacion 18.6}
 ds^2 = \left( 1 - \frac{2 m }{r} \right)
 -  \left( 1 - \frac{2 m }{r} \right)^{-1} dr^2 - r^2 d\theta^2 - r^2 d\theta^2 - r^2 \sin^2 \theta d\phi^2.
\end{equation}
It is known as the Schwarzschild solution. It holds outside the surface of the body that is producing the field, where 
there is no matter. Thus it holds fairly accurately outside the surface of a star.

The solution (\ref{ecuacion 18.6}) leads to small corrections in the Newtonian theory for the motions of the planets 
around the Sun. These corrections are appreciable only in the case of Mercury, the nearest planet, and they explain the 
discrepancy of the motion of this planet with the Newtonian theory. Thus they provide a striking confirmation of the 
Einstein theory.\footnote{As we know that the mass of the Sun is not constant and this fact must have its effect on the 
displacement of the perihelion of the orbit of Mercury. I have attempted to compute this but the methods of classical 
perturbation theory cannot be applied because, when we consider a non constant solar mass, the resulting dynamical 
system stops being autonomous. Also, the displacement of the orbit of Mercury is small $39''$ per century. } 







\section{Black holes}
The solution (\ref{ecuacion 18.6}) becomes singular at $r = 2 m$, because then $g_{00}=0$ and $g_{11}=-\infty$. It 
would seem that $r=2m$ gives a minimum radius for a body of mass $m$. But a close investigation shows that this is not 
so.

Consider a particle falling into the central body and let its velocity vector be $v^\mu = d z^\mu / ds$. Let us suppose 
that it falls in radially, so that $v^2=v^3=0$. The motion is determined by the geodesic equation (\ref{ecuacion 8.3}):
\[
 \begin{array}{rcl}
  \frac{dv^0}{ds} & = & - \Gamma^0_{\mu\mu} v^\mu v^\nu = -g^{00} \Gamma_{0\mu\nu} v^\mu v^\nu\\
         & = & -g^{00} g_{00,1} v^0 v^1 = - g^{00} \frac{dg_{00}}{ds} v^0 .
 \end{array}
\]

Now $g^{00}= \frac{1}{g_{00}}$, so we get
\[
  g_{00} \frac{dv^0}{ds} + \frac{dg_{00}}{ds} v^0 = 0.
\]
This integrates to
\[
 g_{00} v^0 = k,
\]
with $k$ a constant. It is the value of $g_{00}$ where the particle starts to fall.

Again, we have
\[
 1 = g_{\mu\nu} v^\mu v^\nu = g_{00} {v^0}^2 + g_{11} {v^1}^2 .
\]
Multiplying this equation by $g_{00}$ and using $g_{00} g_{11} = -1$, which we obtained in the last section, we find
\[
 k^2 - {v^1}^2 = g_{00} = 1 - \frac{2 m }{r}.
\]
For a falling body $v^1<0$ and hence
\[
 v^1 = - \left( k^2 - 1 + \frac{2m}{r} \right)^{\frac{1}{2}}.
\]

Now
\[
 \frac{dt}{dr} = \frac{v^0}{v^1} = - k \left( 1 - \frac{2 m}{r} \right)^{-1}
 \left( k^2 - 1 + \frac{2m}{r} \right)^{-\frac{1}{2}}.
\]
Let us suppose the particle is close to the critical radius, so $r = 2 m + \epsilon$ with $\epsilon$ small, and let us 
neglect $\epsilon^2$. Then 
\[
 \frac{dt}{dr} \approx - \frac{2 m }{\epsilon} = - \frac{2 m}{r - 2m}.
\]
This integrates to
\[
 t = - 2 m \log (r - 2 m) + \mbox{ constant}.
\]
Thus, as $r\rightarrow 2m$ $t\rightarrow \infty$. The particle takes an infinite time to reach the critical radius $r = 
2 m$.

Let us suppose the particle is emitting light of a certain spectral line, and is being observed by someone at a large 
value of $r$. The light is red-shifted by a factor of ${g_{00}}^{-1/2}$. This factor becomes infinite as the particle 
approaches the critical radius. All physical processes on the particle will be observed to be going more and more 
slowly as it approaches $r=2m$.

Now consider an observer traveling with the particle. His time scale is measured by $ds$. Now 
\[
 \frac{ds}{dr} = \frac{1}{v^1} = - \left( k^2 - 1 + \frac{2 m }{r} \right) ^{-1/2},
\]
and this tends to $-k^{-1}$ as $r$ tends to $2m$. Thus, the particle reaches $r = 2m$ after a lapse of finite proper 
time for the observer. The traveling observer has aged only a finite amount when he reaches $r = 2m$. What will happen 
to him afterwards? He may continue sailing through empty space into smaller values of $r$.

To examine the continuation of the Schwarzschild solution for values of $r < 2 m$, it is necessary to use a non static 
system of coordinates, so that we have  the $g_{\mu\nu}$ varying with the time coordinate. We keep the coordinates 
$\theta$ and $\phi$ unchanged, but instead of $t$ and $r$ we use $\tau$ and $\rho$, defined by
\begin{equation}
 \label{ecuacion 19.1}
 \tau = t + f(r)\, \rho = t + g(r),
\end{equation}
where the functions $f$ and $g$ are at our disposal.

We have, using the prime again to denote derivative with respect to $r$,
\begin{equation}
 \label{ecuacion 19.2}
 \begin{array}{rcl}
   d\tau^2 - \frac{2 m}{3} d\rho^2 & = & (d t + f' dr)^2 - \frac{2 m}{r}(dt + g' dr)^2 \\
   & = & \left( 1 - \frac{2m}{r} \right)dt^2 + 2 \left( f' - \frac{2m}{r} g' \right) dt dr   
   + \left( f'^2 - \frac{2m}{r} g'^2 \right)dr^2\\
   & = & \left(1 - \frac{2m}{r}\right)dt^2 - \left(1 - \frac{2m}{r}\right)^{-1} dr^2
 \end{array}
\end{equation}
provided we choose the functions $f$ and $g$ to satisfy
\begin{equation}
 \label{ecuacion 19.3}
 f' = \frac{2 m}{r} g'
\end{equation}
and
\begin{equation}
 \label{ecuacion 19.4}
 \frac{2m}{r}g'^2 - f'^2 = \left(1 - \frac{2m}{r}\right)^{-1}.
\end{equation}
Elimination of $f$ from these equations gives
\begin{equation}
 \label{ecuacion 19.5}
 g' = \left( \frac{r}{2 m} \right)^{1/2} \left(
     1 = \frac{2 m}{r}
 \right)^{-1}
\end{equation}
To integrate this equation, put $r = y^2$ and $2 m = a^2$. With $r > 2m$ we have $y > a$. We now have
\[
  \frac{dg}{dy} = 2 y \frac{dg}{dr} = \frac{2 y^4}{a} \frac{1}{y^2 - a^2},
\]
\begin{equation}
 \label{ecuacion 19.6}
 g = \frac{2}{3a} y^3 + 2 a y - a^2 \log \frac{y + 1}{y - a}.
\end{equation}
Finally we get from (\ref{ecuacion 19.3}) and (\ref{ecuacion 19.5})
\[
g'-f' = \left( 1 - \frac{2 m }{r} \right) g' = \left(\frac{r}{2 m}\right)^{1/2},
\]
which integrates to 
\begin{equation}
 \label{ecuacion 19.7}
 \frac{2}{3} \frac{1}{\sqrt{2m}} r^{3/2} = g - f = \rho - \tau .
\end{equation}
Thus
\begin{equation}
 \label{ecuacion 19.8}
 r = \mu (\rho - \tau)^{2/3},
\end{equation}
with 
\[
\mu = \left(\frac{3}{2} \sqrt{2 m}\right)^{2/3}.
\]

In this way we see that we can satisfy the conditions (\ref{ecuacion 19.3}) and (\ref{ecuacion 19.4}) and so we can use 
(\ref{ecuacion 19.2}). Substituting into the Schwarzschild solution (\ref{ecuacion 18.6}), we get
\begin{equation}
 \label{ecuacion 19.9}
 ds^2 = d\tau^2 - \frac{2 m}{\mu (\rho - \tau)^{2/3}} d\rho^2 - \mu^2 (\rho - \tau)^{4/3}
 \left( d\theta^2 + \sin^2 \theta d\phi^2 \right).
\end{equation}
The critical value $r = 2 m$ corresponds, from (\ref{ecuacion 19.7}), to $\rho - \tau = 4 m/3$. There is no singularity 
here in the metric (\ref{ecuacion 19.9}).

We know that the metric (\ref{ecuacion 19.9}) satisfies the Einstein equations for empty space in the region $r > 2m$, 
because it can be transformed to the Schwarzschild solution by a mere change of coordinates. We can  infer that it 
satisfies the Einstein equations for $r \le 2m $ from analytic continuity, because it doesn't involve any singularity 
at $r=2m$. It may continue to hold down to $r=0$ or $\rho - \tau = 0$. The singularity appears in the connection 
between the new coordinates and the original ones, equation (\ref{ecuacion 19.1}). But once we have established the 
coordinate system we can disregard the previous one and the singularity no longer appears.

We see that the Schwarzschild solution for empty space can be extended to the region $r<2m$. ut this region cannot 
communicate with the space for which $r>m$. Any signal, even a light signal, will take an infinite time to reach the 
boundary $r=2m$, as we can easily check. Thus we cannot have direct observational knowledge of the region $r < 2m$. 
Such a region is called a black hole, because things can fall into it (taking and infinite time by our clocks, to do 
so) but nothing can come out.

The question arises whether such a region can actually exist. All we can say is that definitely the Einstein equations 
allow it. A massive stellar object can collapse to a very small radius and the gravitational forces then become so 
strong that no physical forces can hold them in check and prevent further collapse. It would seem that it would have to 
collapse into a black hole. It would take an infinite time to do so by our clocks, but only a finite time relative to 
the collapsing matter itself.







\section{Tensor densities}
With a transformation of coordinates, an element of four-dimensional volume transforms according to the law
\begin{equation}
 \label{ecuacion 20.1}
 {dx^0}'{dx^1}'{dx^2}'{dx^3}' = {dx^0}{dx^1}{dx^2}{dx^3} J,
\end{equation}
where $J$ is the Jacobian
\[
 J = \frac{\partial({x^0}',{x^1}',{x^2}',{x^3}')}{\partial({x^0},{x^1},{x^2},{x^3})}
 = \mbox{ determinant of } x^{\mu'}_{,\nu}
\]
We may write (20.1)
\[
 d^4 x' = J d^4 x
\]
for brevity.

Now
\[
   g_{\alpha\beta} = x^{\mu'}_{,\alpha} g_{\mu'\nu'} x^{\nu'}_{,\beta}.
\]
We can look upon the right side as the product of three matrices, the first matrix having its rows specified by 
$\alpha$ and columns specified by $\mu'$, the second having its rows specified by $\mu'$ and columns by $\nu'$ and the 
third having its rows specified by $\nu'$ and columns by $\beta$. This product equals the matrix $g_{\alpha\beta}$ on 
the left. The corresponding equation must hold between the determinants; therefore
\[
 g = J g' J
\]
or
\[
 g = J^2 g'.
\]

Now, $g$ is a negative quantity, so we may form $\sqrt{-g}$, taking the positive value for he square root. Thus
\begin{equation}
 \label{ecuacion 20.3}
 \sqrt{-g} = J \sqrt{- g'}.
\end{equation}

Suppose $S$ is a scalar field quantity, $S = S'$. Then
\[
\int{S \sqrt{-g} d^4 x} = \int{S \sqrt{-g'} J d^4 x} = \int{S' \sqrt{-g'} d^4 x'},
\]
if the region of integration for the $x'$ corresponds to that for the $x$. Thus
\begin{equation}
 \label{ecuacion 20.4}
 \int{S \sqrt{-g} d^4 x} = \mbox{ invariant.}
\end{equation}
We call $S \sqrt{-g}$ a scalar density, meaning a quantity whose integral is invariant.

Similarly, for any tensor field $T^{\mu\nu\cdots}$ we may call $T^{\mu\nu\cdots}\sqrt{-g}$ a tensor density. The 
integral
\[
   \int{T^{\mu\nu\cdots}\sqrt{-g} d^4 x}
\]
is a tensor if the domain of integration is small. It is not a tensor if the domain of integration is not small, 
because it then consists of a sum of tensors located at different points and it does not transform in any simple way in 
a transformation of coordinates.

The quantity $\sqrt{-g}$ will be very much used in the future. For brevity we shall write it simply as $\sqrt{}$. We 
have



\section{Gauss and Stokes theorems}
The vector $A^\mu$ has the covariant divergence $A^\mu_{;\mu}$, which is a scalar. We have
\[
    {A^\mu}_{;\mu} = A^\mu_{,\mu} + \Gamma^\mu_{\nu\mu} A^\nu = A^\mu_{,\mu}+\sqrt{}^{-1}\sqrt{}_{,\nu}A^\nu.
\]
Thus
\begin{equation}
 \label{ecuacion 21.1}
 A^\mu_{;\mu} \sqrt{} = \left( A^\mu \sqrt{} \right)_{,\mu}.
\end{equation}

\section{Harmonic coordinates}
The d'Alembert equation for a scalar $V$, namely $\Box V = 0$, gives, from (\ref{ecuacion 10.9}),
\begin{equation}
 \label{ecuacion 22.1}
 g^{\mu\nu}\left(
    V_{\mu\nu} - \Gamma^\alpha_{\mu\nu} V_{,\alpha} = 0.
 \right)
\end{equation}
If we are using rectilinear axes in flat space, each of the four coordinates $x^\lambda$ satisfies $\Box x^\lambda = 
0$. We might substitute $x^\lambda$ for $V$ in (\ref{ecuacion 22.1}). The result, of course, is not a tensor equation, 
because $x^\lambda$ is not a scalar like $V$, so it holds only in certain coordinate systems. It imposes a restriction 
on the coordinates.

If we substitute $x^\lambda$ for $V$, then for $V_{,\alpha}$ we must substitute $x^\lambda_{,\alpha} = 
g^\lambda_{\alpha}$(\footnote{Which is a constant and therefore $x^\lambda_{,\alpha\beta}\equiv 0$.}). The equation 
(\ref{ecuacion 22.1}) becomes
\begin{equation}
 \label{ecuacion 22.2}
 g^{\mu\nu} \Gamma^\lambda_{\mu\nu} = 0.
\end{equation}

Coordinates that satisfy this condition are called \emph{harmonic coordinates}. They provide the closest approximation 
to rectilinear coordinates that we can have in curve space. They provide the closes approximation to rectilinear 
coordinates that we can have in curved space. We may use them in any problem if we wish to, but very often they are not 
worthwhile because the tensor formalism with general coordinates is really quite convenient. For the discussion of 
gravitational waves, however, harmonic coordinates are very useful.

We have in general coordinates, from (\ref{ecuacion 7.9}) and (\ref{ecuacion 7.6}),
\begin{equation}
 \label{ecuacion 22.3}
 \begin{array}{rcl}
  {g^{\mu\nu}}_{,\sigma} & = & - g^{\mu\alpha}g^{\nu\beta}\left(
    \Gamma_{\alpha\beta\sigma} + \Gamma_{\beta\alpha\sigma}
  \right)\\
  & = & - g^{\nu\beta} \Gamma^\mu_{\beta\sigma} - g^{\mu\alpha} \Gamma^\nu_{\alpha\sigma}.
 \end{array}
\end{equation}
Thus, with the help of (\ref{ecuacion 20.6}),
\begin{equation}
 \label{ecuacion 22.4}
 \left(g^{\mu\nu}\sqrt{}\right)_{,\sigma} = \left( 
 - g^{\nu\beta}  \Gamma^{\mu}_{\beta\sigma}
 - g^{\mu\alpha} \Gamma^{\nu}_{\alpha\sigma}
 + g^{\mu\nu} \Gamma^\beta_{\sigma\beta}
 \right) \sqrt{}.
\end{equation}
Contracting by putting $\sigma = \nu$, we get
\begin{equation}
 \label{ecuacion 22.5}
 \left(g^{\mu\nu} \sqrt{} \right)_{,\nu} = - g^{\nu\beta}\Gamma^\mu_{\beta\nu}\sqrt{}.
\end{equation}

We see now that an alternative form for the harmonic condition is
\begin{equation}
 \label{ecuacion 22.6}
 \left(
 - g^{\mu\nu} \sqrt{}
 \right)_{,\nu} = 0.
\end{equation}



\end{document}
